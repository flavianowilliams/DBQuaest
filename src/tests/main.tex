
\documentclass[12pt, addpoints]{exam}
\usepackage[utf8]{inputenc}
\usepackage[portuguese]{babel}
\usepackage{multicol}
\usepackage{graphicx}
\usepackage{amsmath}
\usepackage{xcolor}
\usepackage{tikz,pgfplots,tikz-3dplot,bm}
\usepackage{circuitikz}
\usepackage{tkz-base}
\usepackage{tkz-fct}
\usepackage{tkz-euclide}
\usepackage[a4paper, portrait, margin=2cm]{geometry}

\usetikzlibrary{arrows,3d,calc,automata,positioning,shadows,math,fit,shapes}
\usetikzlibrary{patterns,hobby,optics,calc}
\tikzset{>=stealth, thick, global scale/.style={scale=#1,every node/.style={scale=#1}}}
\setlength{\columnsep}{1cm}
\renewcommand{\choiceshook}{\setlength{\leftmargin}{0pt}}

\begin{document}

    \begin{minipage}[b]{0.75\linewidth}
        \begin{flushleft}
            {\bf \large Prova bimestral}
        \end{flushleft}
        \begin{flushleft}
            {\bf \large LQ2N (2B), 31 de outubro de 2022}
        \end{flushleft}
    \end{minipage}
    \begin{minipage}[b]{0.20\linewidth}
        \begin{flushright}
            {\bf \large Code: 0}
        \end{flushright}
    \end{minipage}
    \vspace{0.5cm} \hrule \vspace{0.5cm}
    \begin{minipage}{0.55\linewidth}
        \begin{flushleft}
            Student: Flaviano W. Fernandes
        \end{flushleft}
    \end{minipage}
    \begin{minipage}{0.20\linewidth}
        \begin{center}
            Date: 2022-11-14
        \end{center}
    \end{minipage}
    \begin{minipage}{0.20\linewidth}
        \begin{flushright}
            Class: LQ2N
        \end{flushright}
    \end{minipage}
    \vspace{0.5cm} \hrule \vspace{0.5cm}
    \begin{questions}
\begin{multicols*}{2}
\question Considere a figura abaixo onde as linhas trajeçadas representam superfícies equipotenciais Se colocarmos um elétron próximo a carga Q, quais trechos possíveis o elétron poderá se deslocar?
        
        \begin{center}
            \begin{minipage}[c]{0.5\linewidth}
                \begin{tikzpicture}[scale=0.5,transform shape, font=\Large]

                \tkzInit[xmin=-4,xmax=4,ymin=-4,ymax=4]
                \tkzClip[space=0.5]

                \tkzDefPoints{0/0/O,4/0/P}

                \foreach \x in {0.5,1.25,2.25,3,4}{
                    \tkzDrawCircle[R,dashed,color=gray!50](O,\x)
                }

                \foreach \y in {0,1,...,11}{
                    \tkzDefPointsBy[rotation= center O angle 30*\y](O,P){P1,P2}
                \draw[->, line width=1.0pt] (O) -- (P2);}

                \tkzDefPoints{3/0/a,4/0/b,0/4/c,0/3/d}

                \tkzDrawPoints[color=red,fill=red,size=0.3cm](a,b,c,d)

                \tkzDrawPoints(O)
                \tkzLabelPoints[above right,font=\Large](a,b,c,d)

                \node[circle, radius=0.25, ball color=gray!50] (n1) at (0,0) {Q};

                \end{tikzpicture}
            \end{minipage}
        \end{center}
        
        

\begin{choices}
\choice $b\rightarrow c$ ou $a\rightarrow d$ 
\choice $b\rightarrow a$ ou $c\rightarrow d$ 
\choice $b\rightarrow a\rightarrow d\rightarrow c$ ou $c\rightarrow d\rightarrow a\rightarrow b$ 
\choice $a\rightarrow b$ ou $d\rightarrow c$ 
\choice $c\rightarrow b$ ou $d\rightarrow a$ 
\end{choices}
\question Uma corrente elétrica de    2.00 A percorre um fio de cobre. Sabendo-se que a carga de um elétron é igual a $1,6\times 10^{-19}\;C$, qual é o número de elétrons que atravessa, por minuto, a seção reta desse fio?

\begin{oneparchoices}
\choice 1.1e+19 \choice 1.3e+19 \choice 5.5e+19 \choice 3.1e+19 \choice 3.1e+19 \choice 3.5e+19 \choice 8.3e+19 \choice 3.2e-19 \choice 1.9e-17 \choice 7.5e+20 
\end{oneparchoices}\question Uma diferença de potencial de 120 V é aplicada a uma bomba d’água. Sabe-se que em funcionamento, o motor da bomba é percorrido por uma corrente de    1.87 A. Qual é a potência desenvolvida nesse motor?

\begin{oneparchoices}
\choice   0.016 W\choice 1.8e+04 W\choice  64.327 W\choice 417.603 W\choice 320.921 W\choice 2.3e+04 W\choice 1.4e+04 W\choice 223.858 W\choice 9.5e+03 W\choice 2.4e+04 W
\end{oneparchoices}\question A figura abaixo mostra a trajetória de uma partícula eletricamente carregada. $\vec{{v}}$ representa a velocidade atravessando um campo magnético $\vec{{B}}$. Determine a sua trajetória devido a ação da força magnética atuando sobre ela.
        
        \begin{center}
            \begin{minipage}[c]{0.5\linewidth}
                \begin{tikzpicture}[scale=0.5,transform shape, font=\Large]

                    \tkzInit[xmin=-3,xmax=3,ymin=-3,ymax=3]
                %	\tkzGrid[color=gray!20]
                    \tkzClip[space=1.0]

                    \tkzDefPoints{0/0/O,4/0/P}

                    \foreach \x in {-2.5,-1.5,...,2.5}{
                        \foreach \y in {-2.5,-1.5,...,2.5}{
                        \tkzDefPoint(\x,\y){B}
                        \tkzText(B){x}
                }
                }

                \draw[->, line width=1pt, color=red] (0,0) --++ (0,1.5) node [above] {$\vec{v}$};

                    \node[circle, radius=0.25, ball color=gray!50] (n1) at (0,0) {+};

                    \tkzText[above right=0.25cm](B){$\vec{B}$}

                \end{tikzpicture}
            \end{minipage}
        \end{center}

        

\begin{choices}
\choice Paralelo ao papel e circular no sentido horário. 
\choice Paralelo ao papel e na vertical. 
\choice Paralelo ao papel e circular no sentido anti-horário. 
\choice Paralelo ao papel e da direita para a esquerda. 
\choice Paralelo ao papel e da esquerda para a direita. 
\end{choices}
\question Uma partícula de carga 7.03e-06 C é lançada em um campo magnético uniforme de    0.78 T , com uma velocidade de 249.89 m/s. Calcule o valor da força magnética atuando na carga se o ângulo entre a velocidade e o campo magnético for   52.65 graus.

\begin{oneparchoices}
\choice 8.3e-04 N\choice 1.0e-04 N\choice 5.8e-03 N\choice 4.8e-04 N\choice 9.4e-04 N\choice 5.2e-04 N\choice 6.3e-04 N\choice   0.072 N\choice 1.1e-03 N\choice 2.3e-03 N
\end{oneparchoices}\end{multicols*}
\end{questions}
\newpage

    \begin{minipage}[b]{0.75\linewidth}
        \begin{flushleft}
            {\bf \large Prova bimestral}
        \end{flushleft}
        \begin{flushleft}
            {\bf \large LQ2N (2B), 31 de outubro de 2022}
        \end{flushleft}
    \end{minipage}
    \begin{minipage}[b]{0.20\linewidth}
        \begin{flushright}
            {\bf \large Code: 1}
        \end{flushright}
    \end{minipage}
    \vspace{0.5cm} \hrule \vspace{0.5cm}
    \begin{minipage}{0.55\linewidth}
        \begin{flushleft}
            Student: Flaviano - IFPR
        \end{flushleft}
    \end{minipage}
    \begin{minipage}{0.20\linewidth}
        \begin{center}
            Date: 2022-11-14
        \end{center}
    \end{minipage}
    \begin{minipage}{0.20\linewidth}
        \begin{flushright}
            Class: LQ2N
        \end{flushright}
    \end{minipage}
    \vspace{0.5cm} \hrule \vspace{0.5cm}
    \begin{questions}
\begin{multicols*}{2}
\question Considere a figura abaixo onde as linhas trajeçadas representam superfícies equipotenciais Se colocarmos um elétron próximo a carga Q, quais trechos possíveis o elétron poderá se deslocar?
        
        \begin{center}
            \begin{minipage}[c]{0.5\linewidth}
                \begin{tikzpicture}[scale=0.5,transform shape, font=\Large]

                \tkzInit[xmin=-4,xmax=4,ymin=-4,ymax=4]
                \tkzClip[space=0.5]

                \tkzDefPoints{0/0/O,4/0/P}

                \foreach \x in {0.5,1.25,2.25,3,4}{
                    \tkzDrawCircle[R,dashed,color=gray!50](O,\x)
                }

                \foreach \y in {0,1,...,11}{
                    \tkzDefPointsBy[rotation= center O angle 30*\y](O,P){P1,P2}
                \draw[->, line width=1.0pt] (O) -- (P2);}

                \tkzDefPoints{3/0/a,4/0/b,0/4/c,0/3/d}

                \tkzDrawPoints[color=red,fill=red,size=0.3cm](a,b,c,d)

                \tkzDrawPoints(O)
                \tkzLabelPoints[above right,font=\Large](a,b,c,d)

                \node[circle, radius=0.25, ball color=gray!50] (n1) at (0,0) {Q};

                \end{tikzpicture}
            \end{minipage}
        \end{center}
        
        

\begin{choices}
\choice $b\rightarrow a$ ou $c\rightarrow d$ 
\choice $b\rightarrow a\rightarrow d\rightarrow c$ ou $c\rightarrow d\rightarrow a\rightarrow b$ 
\choice $b\rightarrow c$ ou $a\rightarrow d$ 
\choice $c\rightarrow b$ ou $d\rightarrow a$ 
\choice $a\rightarrow b$ ou $d\rightarrow c$ 
\end{choices}
\question Uma corrente elétrica de    4.11 A percorre um fio de cobre. Sabendo-se que a carga de um elétron é igual a $1,6\times 10^{-19}\;C$, qual é o número de elétrons que atravessa, por minuto, a seção reta desse fio?

\begin{oneparchoices}
\choice 3.0e+19 \choice 5.9e+19 \choice 1.5e+21 \choice 3.9e-17 \choice 6.6e-19 \choice 9.9e+19 \choice 2.6e+19 \choice 7.8e+19 \choice 8.2e+19 \choice 3.7e+19 
\end{oneparchoices}\question Uma diferença de potencial de 120 V é aplicada a uma bomba d’água. Sabe-se que em funcionamento, o motor da bomba é percorrido por uma corrente de    3.31 A. Qual é a potência desenvolvida nesse motor?

\begin{oneparchoices}
\choice 2.3e+04 W\choice 1.3e+03 W\choice 1.9e+04 W\choice 2.4e+04 W\choice  36.210 W\choice 1.2e+04 W\choice 397.678 W\choice 3.0e+04 W\choice 3.5e+04 W\choice   0.028 W
\end{oneparchoices}\question A figura abaixo mostra a trajetória de uma partícula eletricamente carregada. $\vec{{v}}$ representa a velocidade atravessando um campo magnético $\vec{{B}}$. Determine a sua trajetória devido a ação da força magnética atuando sobre ela.
        
        \begin{center}
            \begin{minipage}[c]{0.5\linewidth}
                \begin{tikzpicture}[scale=0.5,transform shape, font=\Large]

                    \tkzInit[xmin=-3,xmax=3,ymin=-3,ymax=3]
                %	\tkzGrid[color=gray!20]
                    \tkzClip[space=1.0]

                    \tkzDefPoints{0/0/O,4/0/P}

                    \foreach \x in {-2.5,-1.5,...,2.5}{
                        \foreach \y in {-2.5,-1.5,...,2.5}{
                        \tkzDefPoint(\x,\y){B}
                        \tkzText(B){x}
                }
                }

                \draw[->, line width=1pt, color=red] (0,0) --++ (0,1.5) node [above] {$\vec{v}$};

                    \node[circle, radius=0.25, ball color=gray!50] (n1) at (0,0) {+};

                    \tkzText[above right=0.25cm](B){$\vec{B}$}

                \end{tikzpicture}
            \end{minipage}
        \end{center}

        

\begin{choices}
\choice Paralelo ao papel e na vertical. 
\choice Paralelo ao papel e da direita para a esquerda. 
\choice Paralelo ao papel e da esquerda para a direita. 
\choice Paralelo ao papel e circular no sentido horário. 
\choice Paralelo ao papel e circular no sentido anti-horário. 
\end{choices}
\question Uma partícula de carga 2.49e-06 C é lançada em um campo magnético uniforme de    0.95 T , com uma velocidade de 292.98 m/s. Calcule o valor da força magnética atuando na carga se o ângulo entre a velocidade e o campo magnético for   28.40 graus.

\begin{oneparchoices}
\choice 1.2e-03 N\choice 2.0e-03 N\choice 6.1e-04 N\choice 6.6e-04 N\choice   0.020 N\choice 9.2e-04 N\choice 1.2e-04 N\choice 7.8e-04 N\choice 3.3e-04 N\choice -8.4e-05 N
\end{oneparchoices}\end{multicols*}
\end{questions}
\newpage

    \begin{minipage}[b]{0.75\linewidth}
        \begin{flushleft}
            {\bf \large Prova bimestral}
        \end{flushleft}
        \begin{flushleft}
            {\bf \large LQ2N (2B), 31 de outubro de 2022}
        \end{flushleft}
    \end{minipage}
    \begin{minipage}[b]{0.20\linewidth}
        \begin{flushright}
            {\bf \large Code: 0}
        \end{flushright}
    \end{minipage}
    \vspace{0.5cm} \hrule \vspace{0.5cm}
    \begin{minipage}{0.55\linewidth}
        \begin{flushleft}
            Student: Flaviano W. Fernandes
        \end{flushleft}
    \end{minipage}
    \begin{minipage}{0.20\linewidth}
        \begin{center}
            Date: 2022-11-14
        \end{center}
    \end{minipage}
    \begin{minipage}{0.20\linewidth}
        \begin{flushright}
            Class: LQ2N
        \end{flushright}
    \end{minipage}
    \vspace{0.5cm} \hrule \vspace{0.5cm}
    \begin{questions}
\begin{multicols*}{2}
\question Considere a figura abaixo onde as linhas trajeçadas representam superfícies equipotenciais Se colocarmos um elétron próximo a carga Q, quais trechos possíveis o elétron poderá se deslocar?
        
        \begin{center}
            \begin{minipage}[c]{0.5\linewidth}
                \begin{tikzpicture}[scale=0.5,transform shape, font=\Large]

                \tkzInit[xmin=-4,xmax=4,ymin=-4,ymax=4]
                \tkzClip[space=0.5]

                \tkzDefPoints{0/0/O,4/0/P}

                \foreach \x in {0.5,1.25,2.25,3,4}{
                    \tkzDrawCircle[R,dashed,color=gray!50](O,\x)
                }

                \foreach \y in {0,1,...,11}{
                    \tkzDefPointsBy[rotation= center O angle 30*\y](O,P){P1,P2}
                \draw[->, line width=1.0pt] (O) -- (P2);}

                \tkzDefPoints{3/0/a,4/0/b,0/4/c,0/3/d}

                \tkzDrawPoints[color=red,fill=red,size=0.3cm](a,b,c,d)

                \tkzDrawPoints(O)
                \tkzLabelPoints[above right,font=\Large](a,b,c,d)

                \node[circle, radius=0.25, ball color=gray!50] (n1) at (0,0) {Q};

                \end{tikzpicture}
            \end{minipage}
        \end{center}
        
        

\begin{choices}
\choice $a\rightarrow b$ ou $d\rightarrow c$ 
\choice $b\rightarrow c$ ou $a\rightarrow d$ 
\choice $b\rightarrow a$ ou $c\rightarrow d$ 
\choice $c\rightarrow b$ ou $d\rightarrow a$ 
\choice $b\rightarrow a\rightarrow d\rightarrow c$ ou $c\rightarrow d\rightarrow a\rightarrow b$ 
\end{choices}
\question Uma corrente elétrica de    3.74 A percorre um fio de cobre. Sabendo-se que a carga de um elétron é igual a $1,6\times 10^{-19}\;C$, qual é o número de elétrons que atravessa, por minuto, a seção reta desse fio?

\begin{oneparchoices}
\choice 2.6e+19 \choice 6.0e-19 \choice 1.0e+20 \choice 3.9e+19 \choice 2.3e+19 \choice 3.6e-17 \choice 1.4e+21 \choice 4.6e+19 \choice 9.8e+19 \choice 4.1e+19 
\end{oneparchoices}\question Uma diferença de potencial de 120 V é aplicada a uma bomba d’água. Sabe-se que em funcionamento, o motor da bomba é percorrido por uma corrente de    2.32 A. Qual é a potência desenvolvida nesse motor?

\begin{oneparchoices}
\choice  51.619 W\choice 8.5e+03 W\choice 2.4e+04 W\choice 8.5e+03 W\choice 648.517 W\choice 7.6e+03 W\choice 278.966 W\choice 8.5e+03 W\choice   0.019 W\choice 6.0e+03 W
\end{oneparchoices}\question A figura abaixo mostra a trajetória de uma partícula eletricamente carregada. $\vec{{v}}$ representa a velocidade atravessando um campo magnético $\vec{{B}}$. Determine a sua trajetória devido a ação da força magnética atuando sobre ela.
        
        \begin{center}
            \begin{minipage}[c]{0.5\linewidth}
                \begin{tikzpicture}[scale=0.5,transform shape, font=\Large]

                    \tkzInit[xmin=-3,xmax=3,ymin=-3,ymax=3]
                %	\tkzGrid[color=gray!20]
                    \tkzClip[space=1.0]

                    \tkzDefPoints{0/0/O,4/0/P}

                    \foreach \x in {-2.5,-1.5,...,2.5}{
                        \foreach \y in {-2.5,-1.5,...,2.5}{
                        \tkzDefPoint(\x,\y){B}
                        \tkzText(B){x}
                }
                }

                \draw[->, line width=1pt, color=red] (0,0) --++ (0,1.5) node [above] {$\vec{v}$};

                    \node[circle, radius=0.25, ball color=gray!50] (n1) at (0,0) {+};

                    \tkzText[above right=0.25cm](B){$\vec{B}$}

                \end{tikzpicture}
            \end{minipage}
        \end{center}

        

\begin{choices}
\choice Paralelo ao papel e circular no sentido horário. 
\choice Paralelo ao papel e da direita para a esquerda. 
\choice Paralelo ao papel e da esquerda para a direita. 
\choice Paralelo ao papel e na vertical. 
\choice Paralelo ao papel e circular no sentido anti-horário. 
\end{choices}
\question Uma partícula de carga 4.43e-06 C é lançada em um campo magnético uniforme de    0.99 T , com uma velocidade de 224.35 m/s. Calcule o valor da força magnética atuando na carga se o ângulo entre a velocidade e o campo magnético for   63.51 graus.

\begin{oneparchoices}
\choice 1.2e-03 N\choice 6.1e-04 N\choice 7.3e-05 N\choice 9.6e-05 N\choice 5.7e-04 N\choice 4.9e-04 N\choice   0.062 N\choice 4.4e-04 N\choice 8.8e-04 N\choice 3.5e-04 N
\end{oneparchoices}\end{multicols*}
\end{questions}
\newpage

    \begin{minipage}[b]{0.75\linewidth}
        \begin{flushleft}
            {\bf \large Prova bimestral}
        \end{flushleft}
        \begin{flushleft}
            {\bf \large LQ2N (2B), 31 de outubro de 2022}
        \end{flushleft}
    \end{minipage}
    \begin{minipage}[b]{0.20\linewidth}
        \begin{flushright}
            {\bf \large Code: 1}
        \end{flushright}
    \end{minipage}
    \vspace{0.5cm} \hrule \vspace{0.5cm}
    \begin{minipage}{0.55\linewidth}
        \begin{flushleft}
            Student: Flaviano - IFPR
        \end{flushleft}
    \end{minipage}
    \begin{minipage}{0.20\linewidth}
        \begin{center}
            Date: 2022-11-14
        \end{center}
    \end{minipage}
    \begin{minipage}{0.20\linewidth}
        \begin{flushright}
            Class: LQ2N
        \end{flushright}
    \end{minipage}
    \vspace{0.5cm} \hrule \vspace{0.5cm}
    \begin{questions}
\begin{multicols*}{2}
\question Considere a figura abaixo onde as linhas trajeçadas representam superfícies equipotenciais Se colocarmos um elétron próximo a carga Q, quais trechos possíveis o elétron poderá se deslocar?
        
        \begin{center}
            \begin{minipage}[c]{0.5\linewidth}
                \begin{tikzpicture}[scale=0.5,transform shape, font=\Large]

                \tkzInit[xmin=-4,xmax=4,ymin=-4,ymax=4]
                \tkzClip[space=0.5]

                \tkzDefPoints{0/0/O,4/0/P}

                \foreach \x in {0.5,1.25,2.25,3,4}{
                    \tkzDrawCircle[R,dashed,color=gray!50](O,\x)
                }

                \foreach \y in {0,1,...,11}{
                    \tkzDefPointsBy[rotation= center O angle 30*\y](O,P){P1,P2}
                \draw[->, line width=1.0pt] (O) -- (P2);}

                \tkzDefPoints{3/0/a,4/0/b,0/4/c,0/3/d}

                \tkzDrawPoints[color=red,fill=red,size=0.3cm](a,b,c,d)

                \tkzDrawPoints(O)
                \tkzLabelPoints[above right,font=\Large](a,b,c,d)

                \node[circle, radius=0.25, ball color=gray!50] (n1) at (0,0) {Q};

                \end{tikzpicture}
            \end{minipage}
        \end{center}
        
        

\begin{choices}
\choice $c\rightarrow b$ ou $d\rightarrow a$ 
\choice $a\rightarrow b$ ou $d\rightarrow c$ 
\choice $b\rightarrow a$ ou $c\rightarrow d$ 
\choice $b\rightarrow c$ ou $a\rightarrow d$ 
\choice $b\rightarrow a\rightarrow d\rightarrow c$ ou $c\rightarrow d\rightarrow a\rightarrow b$ 
\end{choices}
\question Uma corrente elétrica de    3.31 A percorre um fio de cobre. Sabendo-se que a carga de um elétron é igual a $1,6\times 10^{-19}\;C$, qual é o número de elétrons que atravessa, por minuto, a seção reta desse fio?

\begin{oneparchoices}
\choice 1.2e+21 \choice 9.2e+19 \choice 5.3e-19 \choice 7.0e+19 \choice 3.9e+19 \choice 4.7e+19 \choice 2.1e+19 \choice 1.9e+19 \choice 3.2e-17 \choice 2.3e+19 
\end{oneparchoices}\question Uma diferença de potencial de 120 V é aplicada a uma bomba d’água. Sabe-se que em funcionamento, o motor da bomba é percorrido por uma corrente de    4.10 A. Qual é a potência desenvolvida nesse motor?

\begin{oneparchoices}
\choice 3.5e+04 W\choice 2.0e+03 W\choice 2.6e+04 W\choice 2.2e+03 W\choice 3.7e+03 W\choice  29.263 W\choice 946.740 W\choice   0.034 W\choice 1.7e+04 W\choice 492.085 W
\end{oneparchoices}\question A figura abaixo mostra a trajetória de uma partícula eletricamente carregada. $\vec{{v}}$ representa a velocidade atravessando um campo magnético $\vec{{B}}$. Determine a sua trajetória devido a ação da força magnética atuando sobre ela.
        
        \begin{center}
            \begin{minipage}[c]{0.5\linewidth}
                \begin{tikzpicture}[scale=0.5,transform shape, font=\Large]

                    \tkzInit[xmin=-3,xmax=3,ymin=-3,ymax=3]
                %	\tkzGrid[color=gray!20]
                    \tkzClip[space=1.0]

                    \tkzDefPoints{0/0/O,4/0/P}

                    \foreach \x in {-2.5,-1.5,...,2.5}{
                        \foreach \y in {-2.5,-1.5,...,2.5}{
                        \tkzDefPoint(\x,\y){B}
                        \tkzText(B){x}
                }
                }

                \draw[->, line width=1pt, color=red] (0,0) --++ (0,1.5) node [above] {$\vec{v}$};

                    \node[circle, radius=0.25, ball color=gray!50] (n1) at (0,0) {+};

                    \tkzText[above right=0.25cm](B){$\vec{B}$}

                \end{tikzpicture}
            \end{minipage}
        \end{center}

        

\begin{choices}
\choice Paralelo ao papel e circular no sentido horário. 
\choice Paralelo ao papel e circular no sentido anti-horário. 
\choice Paralelo ao papel e da direita para a esquerda. 
\choice Paralelo ao papel e da esquerda para a direita. 
\choice Paralelo ao papel e na vertical. 
\end{choices}
\question Uma partícula de carga 7.56e-06 C é lançada em um campo magnético uniforme de    0.65 T , com uma velocidade de 968.51 m/s. Calcule o valor da força magnética atuando na carga se o ângulo entre a velocidade e o campo magnético for   43.64 graus.

\begin{oneparchoices}
\choice   0.207 N\choice 8.0e-04 N\choice -1.6e-03 N\choice 2.9e-05 N\choice 7.8e-04 N\choice 6.3e-04 N\choice 1.7e-04 N\choice 3.3e-03 N\choice 3.4e-03 N\choice 5.7e-04 N
\end{oneparchoices}\end{multicols*}
\end{questions}
\newpage

    \begin{minipage}[b]{0.75\linewidth}
        \begin{flushleft}
            {\bf \large Prova bimestral}
        \end{flushleft}
        \begin{flushleft}
            {\bf \large LQ2N (2B), 31 de outubro de 2022}
        \end{flushleft}
    \end{minipage}
    \begin{minipage}[b]{0.20\linewidth}
        \begin{flushright}
            {\bf \large Code: 0}
        \end{flushright}
    \end{minipage}
    \vspace{0.5cm} \hrule \vspace{0.5cm}
    \begin{minipage}{0.55\linewidth}
        \begin{flushleft}
            Student: Flaviano W. Fernandes
        \end{flushleft}
    \end{minipage}
    \begin{minipage}{0.20\linewidth}
        \begin{center}
            Date: 2022-11-14
        \end{center}
    \end{minipage}
    \begin{minipage}{0.20\linewidth}
        \begin{flushright}
            Class: LQ2N
        \end{flushright}
    \end{minipage}
    \vspace{0.5cm} \hrule \vspace{0.5cm}
    \begin{questions}
\begin{multicols*}{2}
\question Considere a figura abaixo onde as linhas trajeçadas representam superfícies equipotenciais Se colocarmos um elétron próximo a carga Q, quais trechos possíveis o elétron poderá se deslocar?
        
        \begin{center}
            \begin{minipage}[c]{0.5\linewidth}
                \begin{tikzpicture}[scale=0.5,transform shape, font=\Large]

                \tkzInit[xmin=-4,xmax=4,ymin=-4,ymax=4]
                \tkzClip[space=0.5]

                \tkzDefPoints{0/0/O,4/0/P}

                \foreach \x in {0.5,1.25,2.25,3,4}{
                    \tkzDrawCircle[R,dashed,color=gray!50](O,\x)
                }

                \foreach \y in {0,1,...,11}{
                    \tkzDefPointsBy[rotation= center O angle 30*\y](O,P){P1,P2}
                \draw[->, line width=1.0pt] (O) -- (P2);}

                \tkzDefPoints{3/0/a,4/0/b,0/4/c,0/3/d}

                \tkzDrawPoints[color=red,fill=red,size=0.3cm](a,b,c,d)

                \tkzDrawPoints(O)
                \tkzLabelPoints[above right,font=\Large](a,b,c,d)

                \node[circle, radius=0.25, ball color=gray!50] (n1) at (0,0) {Q};

                \end{tikzpicture}
            \end{minipage}
        \end{center}
        
        

\begin{choices}
\choice $b\rightarrow c$ ou $a\rightarrow d$ 
\choice $c\rightarrow b$ ou $d\rightarrow a$ 
\choice $b\rightarrow a$ ou $c\rightarrow d$ 
\choice $a\rightarrow b$ ou $d\rightarrow c$ 
\choice $b\rightarrow a\rightarrow d\rightarrow c$ ou $c\rightarrow d\rightarrow a\rightarrow b$ 
\end{choices}
\question Uma corrente elétrica de    8.88 A percorre um fio de cobre. Sabendo-se que a carga de um elétron é igual a $1,6\times 10^{-19}\;C$, qual é o número de elétrons que atravessa, por minuto, a seção reta desse fio?

\begin{oneparchoices}
\choice 8.5e-17 \choice 3.9e+19 \choice 5.6e+19 \choice 7.7e+19 \choice 3.3e+21 \choice 8.0e+19 \choice 1.4e-18 \choice 6.0e+19 \choice 9.2e+19 \choice 9.1e+19 
\end{oneparchoices}\question Uma diferença de potencial de 120 V é aplicada a uma bomba d’água. Sabe-se que em funcionamento, o motor da bomba é percorrido por uma corrente de    4.71 A. Qual é a potência desenvolvida nesse motor?

\begin{oneparchoices}
\choice  25.459 W\choice 2.6e+04 W\choice 1.7e+04 W\choice 2.6e+04 W\choice   0.039 W\choice 3.4e+04 W\choice 1.5e+04 W\choice 565.619 W\choice 2.7e+03 W\choice 1.1e+04 W
\end{oneparchoices}\question A figura abaixo mostra a trajetória de uma partícula eletricamente carregada. $\vec{{v}}$ representa a velocidade atravessando um campo magnético $\vec{{B}}$. Determine a sua trajetória devido a ação da força magnética atuando sobre ela.
        
        \begin{center}
            \begin{minipage}[c]{0.5\linewidth}
                \begin{tikzpicture}[scale=0.5,transform shape, font=\Large]

                    \tkzInit[xmin=-3,xmax=3,ymin=-3,ymax=3]
                %	\tkzGrid[color=gray!20]
                    \tkzClip[space=1.0]

                    \tkzDefPoints{0/0/O,4/0/P}

                    \foreach \x in {-2.5,-1.5,...,2.5}{
                        \foreach \y in {-2.5,-1.5,...,2.5}{
                        \tkzDefPoint(\x,\y){B}
                        \tkzText(B){x}
                }
                }

                \draw[->, line width=1pt, color=red] (0,0) --++ (0,1.5) node [above] {$\vec{v}$};

                    \node[circle, radius=0.25, ball color=gray!50] (n1) at (0,0) {+};

                    \tkzText[above right=0.25cm](B){$\vec{B}$}

                \end{tikzpicture}
            \end{minipage}
        \end{center}

        

\begin{choices}
\choice Paralelo ao papel e circular no sentido horário. 
\choice Paralelo ao papel e da direita para a esquerda. 
\choice Paralelo ao papel e da esquerda para a direita. 
\choice Paralelo ao papel e na vertical. 
\choice Paralelo ao papel e circular no sentido anti-horário. 
\end{choices}
\question Uma partícula de carga 9.68e-06 C é lançada em um campo magnético uniforme de    0.93 T , com uma velocidade de 977.53 m/s. Calcule o valor da força magnética atuando na carga se o ângulo entre a velocidade e o campo magnético for   24.15 graus.

\begin{oneparchoices}
\choice 8.0e-03 N\choice 2.1e-03 N\choice   0.212 N\choice 2.4e-03 N\choice 3.6e-03 N\choice -7.3e-03 N\choice 4.3e-04 N\choice 8.4e-04 N\choice 3.2e-05 N\choice 1.2e-03 N
\end{oneparchoices}\end{multicols*}
\end{questions}
\newpage

    \begin{minipage}[b]{0.75\linewidth}
        \begin{flushleft}
            {\bf \large Prova bimestral}
        \end{flushleft}
        \begin{flushleft}
            {\bf \large LQ2N (2B), 31 de outubro de 2022}
        \end{flushleft}
    \end{minipage}
    \begin{minipage}[b]{0.20\linewidth}
        \begin{flushright}
            {\bf \large Code: 1}
        \end{flushright}
    \end{minipage}
    \vspace{0.5cm} \hrule \vspace{0.5cm}
    \begin{minipage}{0.55\linewidth}
        \begin{flushleft}
            Student: Flaviano - IFPR
        \end{flushleft}
    \end{minipage}
    \begin{minipage}{0.20\linewidth}
        \begin{center}
            Date: 2022-11-14
        \end{center}
    \end{minipage}
    \begin{minipage}{0.20\linewidth}
        \begin{flushright}
            Class: LQ2N
        \end{flushright}
    \end{minipage}
    \vspace{0.5cm} \hrule \vspace{0.5cm}
    \begin{questions}
\begin{multicols*}{2}
\question Considere a figura abaixo onde as linhas trajeçadas representam superfícies equipotenciais Se colocarmos um elétron próximo a carga Q, quais trechos possíveis o elétron poderá se deslocar?
        
        \begin{center}
            \begin{minipage}[c]{0.5\linewidth}
                \begin{tikzpicture}[scale=0.5,transform shape, font=\Large]

                \tkzInit[xmin=-4,xmax=4,ymin=-4,ymax=4]
                \tkzClip[space=0.5]

                \tkzDefPoints{0/0/O,4/0/P}

                \foreach \x in {0.5,1.25,2.25,3,4}{
                    \tkzDrawCircle[R,dashed,color=gray!50](O,\x)
                }

                \foreach \y in {0,1,...,11}{
                    \tkzDefPointsBy[rotation= center O angle 30*\y](O,P){P1,P2}
                \draw[->, line width=1.0pt] (O) -- (P2);}

                \tkzDefPoints{3/0/a,4/0/b,0/4/c,0/3/d}

                \tkzDrawPoints[color=red,fill=red,size=0.3cm](a,b,c,d)

                \tkzDrawPoints(O)
                \tkzLabelPoints[above right,font=\Large](a,b,c,d)

                \node[circle, radius=0.25, ball color=gray!50] (n1) at (0,0) {Q};

                \end{tikzpicture}
            \end{minipage}
        \end{center}
        
        

\begin{choices}
\choice $b\rightarrow a\rightarrow d\rightarrow c$ ou $c\rightarrow d\rightarrow a\rightarrow b$ 
\choice $b\rightarrow c$ ou $a\rightarrow d$ 
\choice $c\rightarrow b$ ou $d\rightarrow a$ 
\choice $b\rightarrow a$ ou $c\rightarrow d$ 
\choice $a\rightarrow b$ ou $d\rightarrow c$ 
\end{choices}
\question Uma corrente elétrica de    5.18 A percorre um fio de cobre. Sabendo-se que a carga de um elétron é igual a $1,6\times 10^{-19}\;C$, qual é o número de elétrons que atravessa, por minuto, a seção reta desse fio?

\begin{oneparchoices}
\choice 8.9e+19 \choice 3.2e+19 \choice 1.4e+19 \choice 8.7e+19 \choice 8.3e-19 \choice 9.8e+19 \choice 1.7e+19 \choice 1.9e+21 \choice 5.0e-17 \choice 9.6e+19 
\end{oneparchoices}\question Uma diferença de potencial de 120 V é aplicada a uma bomba d’água. Sabe-se que em funcionamento, o motor da bomba é percorrido por uma corrente de    3.14 A. Qual é a potência desenvolvida nesse motor?

\begin{oneparchoices}
\choice 376.317 W\choice 1.1e+04 W\choice 3.3e+04 W\choice 1.1e+04 W\choice  38.266 W\choice   0.026 W\choice 1.1e+04 W\choice 1.5e+04 W\choice 2.5e+04 W\choice 1.2e+03 W
\end{oneparchoices}\question A figura abaixo mostra a trajetória de uma partícula eletricamente carregada. $\vec{{v}}$ representa a velocidade atravessando um campo magnético $\vec{{B}}$. Determine a sua trajetória devido a ação da força magnética atuando sobre ela.
        
        \begin{center}
            \begin{minipage}[c]{0.5\linewidth}
                \begin{tikzpicture}[scale=0.5,transform shape, font=\Large]

                    \tkzInit[xmin=-3,xmax=3,ymin=-3,ymax=3]
                %	\tkzGrid[color=gray!20]
                    \tkzClip[space=1.0]

                    \tkzDefPoints{0/0/O,4/0/P}

                    \foreach \x in {-2.5,-1.5,...,2.5}{
                        \foreach \y in {-2.5,-1.5,...,2.5}{
                        \tkzDefPoint(\x,\y){B}
                        \tkzText(B){x}
                }
                }

                \draw[->, line width=1pt, color=red] (0,0) --++ (0,1.5) node [above] {$\vec{v}$};

                    \node[circle, radius=0.25, ball color=gray!50] (n1) at (0,0) {+};

                    \tkzText[above right=0.25cm](B){$\vec{B}$}

                \end{tikzpicture}
            \end{minipage}
        \end{center}

        

\begin{choices}
\choice Paralelo ao papel e circular no sentido horário. 
\choice Paralelo ao papel e na vertical. 
\choice Paralelo ao papel e da direita para a esquerda. 
\choice Paralelo ao papel e circular no sentido anti-horário. 
\choice Paralelo ao papel e da esquerda para a direita. 
\end{choices}
\question Uma partícula de carga 3.40e-06 C é lançada em um campo magnético uniforme de    0.82 T , com uma velocidade de 863.85 m/s. Calcule o valor da força magnética atuando na carga se o ângulo entre a velocidade e o campo magnético for    1.87 graus.

\begin{oneparchoices}
\choice 7.8e-05 N\choice 3.8e-04 N\choice 5.1e-05 N\choice 1.1e-04 N\choice 9.8e-04 N\choice 2.3e-03 N\choice 4.3e-04 N\choice 4.5e-03 N\choice 2.4e-03 N\choice 2.3e-04 N
\end{oneparchoices}\end{multicols*}
\end{questions}
\newpage

    \begin{minipage}[b]{0.75\linewidth}
        \begin{flushleft}
            {\bf \large Prova bimestral}
        \end{flushleft}
        \begin{flushleft}
            {\bf \large LQ2N (2B), 31 de outubro de 2022}
        \end{flushleft}
    \end{minipage}
    \begin{minipage}[b]{0.20\linewidth}
        \begin{flushright}
            {\bf \large Code: 0}
        \end{flushright}
    \end{minipage}
    \vspace{0.5cm} \hrule \vspace{0.5cm}
    \begin{minipage}{0.55\linewidth}
        \begin{flushleft}
            Student: Flaviano W. Fernandes
        \end{flushleft}
    \end{minipage}
    \begin{minipage}{0.20\linewidth}
        \begin{center}
            Date: 2022-11-14
        \end{center}
    \end{minipage}
    \begin{minipage}{0.20\linewidth}
        \begin{flushright}
            Class: LQ2N
        \end{flushright}
    \end{minipage}
    \vspace{0.5cm} \hrule \vspace{0.5cm}
    \begin{questions}
\begin{multicols*}{2}
\question Considere a figura abaixo onde as linhas trajeçadas representam superfícies equipotenciais Se colocarmos um elétron próximo a carga Q, quais trechos possíveis o elétron poderá se deslocar?
        
        \begin{center}
            \begin{minipage}[c]{0.5\linewidth}
                \begin{tikzpicture}[scale=0.5,transform shape, font=\Large]

                \tkzInit[xmin=-4,xmax=4,ymin=-4,ymax=4]
                \tkzClip[space=0.5]

                \tkzDefPoints{0/0/O,4/0/P}

                \foreach \x in {0.5,1.25,2.25,3,4}{
                    \tkzDrawCircle[R,dashed,color=gray!50](O,\x)
                }

                \foreach \y in {0,1,...,11}{
                    \tkzDefPointsBy[rotation= center O angle 30*\y](O,P){P1,P2}
                \draw[->, line width=1.0pt] (O) -- (P2);}

                \tkzDefPoints{3/0/a,4/0/b,0/4/c,0/3/d}

                \tkzDrawPoints[color=red,fill=red,size=0.3cm](a,b,c,d)

                \tkzDrawPoints(O)
                \tkzLabelPoints[above right,font=\Large](a,b,c,d)

                \node[circle, radius=0.25, ball color=gray!50] (n1) at (0,0) {Q};

                \end{tikzpicture}
            \end{minipage}
        \end{center}
        
        

\begin{choices}
\choice $b\rightarrow c$ ou $a\rightarrow d$ 
\choice $a\rightarrow b$ ou $d\rightarrow c$ 
\choice $b\rightarrow a\rightarrow d\rightarrow c$ ou $c\rightarrow d\rightarrow a\rightarrow b$ 
\choice $c\rightarrow b$ ou $d\rightarrow a$ 
\choice $b\rightarrow a$ ou $c\rightarrow d$ 
\end{choices}
\question Uma corrente elétrica de    5.69 A percorre um fio de cobre. Sabendo-se que a carga de um elétron é igual a $1,6\times 10^{-19}\;C$, qual é o número de elétrons que atravessa, por minuto, a seção reta desse fio?

\begin{oneparchoices}
\choice 6.7e+19 \choice 7.0e+19 \choice 5.5e-17 \choice 8.0e+19 \choice 8.9e+19 \choice 3.6e+19 \choice 2.6e+19 \choice 2.1e+21 \choice 9.1e-19 \choice 6.2e+19 
\end{oneparchoices}\question Uma diferença de potencial de 120 V é aplicada a uma bomba d’água. Sabe-se que em funcionamento, o motor da bomba é percorrido por uma corrente de    4.51 A. Qual é a potência desenvolvida nesse motor?

\begin{oneparchoices}
\choice 158.614 W\choice 541.606 W\choice  35.169 W\choice 1.7e+04 W\choice  26.588 W\choice 2.0e+04 W\choice   0.038 W\choice 2.5e+04 W\choice 682.883 W\choice 2.4e+03 W
\end{oneparchoices}\question A figura abaixo mostra a trajetória de uma partícula eletricamente carregada. $\vec{{v}}$ representa a velocidade atravessando um campo magnético $\vec{{B}}$. Determine a sua trajetória devido a ação da força magnética atuando sobre ela.
        
        \begin{center}
            \begin{minipage}[c]{0.5\linewidth}
                \begin{tikzpicture}[scale=0.5,transform shape, font=\Large]

                    \tkzInit[xmin=-3,xmax=3,ymin=-3,ymax=3]
                %	\tkzGrid[color=gray!20]
                    \tkzClip[space=1.0]

                    \tkzDefPoints{0/0/O,4/0/P}

                    \foreach \x in {-2.5,-1.5,...,2.5}{
                        \foreach \y in {-2.5,-1.5,...,2.5}{
                        \tkzDefPoint(\x,\y){B}
                        \tkzText(B){x}
                }
                }

                \draw[->, line width=1pt, color=red] (0,0) --++ (0,1.5) node [above] {$\vec{v}$};

                    \node[circle, radius=0.25, ball color=gray!50] (n1) at (0,0) {+};

                    \tkzText[above right=0.25cm](B){$\vec{B}$}

                \end{tikzpicture}
            \end{minipage}
        \end{center}

        

\begin{choices}
\choice Paralelo ao papel e circular no sentido horário. 
\choice Paralelo ao papel e da esquerda para a direita. 
\choice Paralelo ao papel e circular no sentido anti-horário. 
\choice Paralelo ao papel e da direita para a esquerda. 
\choice Paralelo ao papel e na vertical. 
\end{choices}
\question Uma partícula de carga 8.58e-06 C é lançada em um campo magnético uniforme de    0.85 T , com uma velocidade de 948.86 m/s. Calcule o valor da força magnética atuando na carga se o ângulo entre a velocidade e o campo magnético for   74.64 graus.

\begin{oneparchoices}
\choice 7.8e-04 N\choice 3.4e-05 N\choice 3.8e-03 N\choice 1.8e-03 N\choice 1.0e-03 N\choice -4.7e-03 N\choice 6.1e-04 N\choice   0.515 N\choice 6.6e-03 N\choice 7.4e-04 N
\end{oneparchoices}\end{multicols*}
\end{questions}
\newpage

    \begin{minipage}[b]{0.75\linewidth}
        \begin{flushleft}
            {\bf \large Prova bimestral}
        \end{flushleft}
        \begin{flushleft}
            {\bf \large LQ2N (2B), 31 de outubro de 2022}
        \end{flushleft}
    \end{minipage}
    \begin{minipage}[b]{0.20\linewidth}
        \begin{flushright}
            {\bf \large Code: 1}
        \end{flushright}
    \end{minipage}
    \vspace{0.5cm} \hrule \vspace{0.5cm}
    \begin{minipage}{0.55\linewidth}
        \begin{flushleft}
            Student: Flaviano - IFPR
        \end{flushleft}
    \end{minipage}
    \begin{minipage}{0.20\linewidth}
        \begin{center}
            Date: 2022-11-14
        \end{center}
    \end{minipage}
    \begin{minipage}{0.20\linewidth}
        \begin{flushright}
            Class: LQ2N
        \end{flushright}
    \end{minipage}
    \vspace{0.5cm} \hrule \vspace{0.5cm}
    \begin{questions}
\begin{multicols*}{2}
\question Considere a figura abaixo onde as linhas trajeçadas representam superfícies equipotenciais Se colocarmos um elétron próximo a carga Q, quais trechos possíveis o elétron poderá se deslocar?
        
        \begin{center}
            \begin{minipage}[c]{0.5\linewidth}
                \begin{tikzpicture}[scale=0.5,transform shape, font=\Large]

                \tkzInit[xmin=-4,xmax=4,ymin=-4,ymax=4]
                \tkzClip[space=0.5]

                \tkzDefPoints{0/0/O,4/0/P}

                \foreach \x in {0.5,1.25,2.25,3,4}{
                    \tkzDrawCircle[R,dashed,color=gray!50](O,\x)
                }

                \foreach \y in {0,1,...,11}{
                    \tkzDefPointsBy[rotation= center O angle 30*\y](O,P){P1,P2}
                \draw[->, line width=1.0pt] (O) -- (P2);}

                \tkzDefPoints{3/0/a,4/0/b,0/4/c,0/3/d}

                \tkzDrawPoints[color=red,fill=red,size=0.3cm](a,b,c,d)

                \tkzDrawPoints(O)
                \tkzLabelPoints[above right,font=\Large](a,b,c,d)

                \node[circle, radius=0.25, ball color=gray!50] (n1) at (0,0) {Q};

                \end{tikzpicture}
            \end{minipage}
        \end{center}
        
        

\begin{choices}
\choice $b\rightarrow a$ ou $c\rightarrow d$ 
\choice $c\rightarrow b$ ou $d\rightarrow a$ 
\choice $b\rightarrow c$ ou $a\rightarrow d$ 
\choice $b\rightarrow a\rightarrow d\rightarrow c$ ou $c\rightarrow d\rightarrow a\rightarrow b$ 
\choice $a\rightarrow b$ ou $d\rightarrow c$ 
\end{choices}
\question Uma corrente elétrica de    8.87 A percorre um fio de cobre. Sabendo-se que a carga de um elétron é igual a $1,6\times 10^{-19}\;C$, qual é o número de elétrons que atravessa, por minuto, a seção reta desse fio?

\begin{oneparchoices}
\choice 6.8e+19 \choice 6.2e+19 \choice 8.1e+19 \choice 8.7e+19 \choice 1.4e-18 \choice 8.5e-17 \choice 3.3e+21 \choice 7.2e+19 \choice 4.1e+19 \choice 5.5e+19 
\end{oneparchoices}\question Uma diferença de potencial de 120 V é aplicada a uma bomba d’água. Sabe-se que em funcionamento, o motor da bomba é percorrido por uma corrente de    3.48 A. Qual é a potência desenvolvida nesse motor?

\begin{oneparchoices}
\choice 2.7e+03 W\choice  34.519 W\choice   0.029 W\choice 2.3e+04 W\choice 1.6e+03 W\choice 417.156 W\choice 2.8e+03 W\choice 1.4e+04 W\choice 3.1e+04 W\choice 1.5e+03 W
\end{oneparchoices}\question A figura abaixo mostra a trajetória de uma partícula eletricamente carregada. $\vec{{v}}$ representa a velocidade atravessando um campo magnético $\vec{{B}}$. Determine a sua trajetória devido a ação da força magnética atuando sobre ela.
        
        \begin{center}
            \begin{minipage}[c]{0.5\linewidth}
                \begin{tikzpicture}[scale=0.5,transform shape, font=\Large]

                    \tkzInit[xmin=-3,xmax=3,ymin=-3,ymax=3]
                %	\tkzGrid[color=gray!20]
                    \tkzClip[space=1.0]

                    \tkzDefPoints{0/0/O,4/0/P}

                    \foreach \x in {-2.5,-1.5,...,2.5}{
                        \foreach \y in {-2.5,-1.5,...,2.5}{
                        \tkzDefPoint(\x,\y){B}
                        \tkzText(B){x}
                }
                }

                \draw[->, line width=1pt, color=red] (0,0) --++ (0,1.5) node [above] {$\vec{v}$};

                    \node[circle, radius=0.25, ball color=gray!50] (n1) at (0,0) {+};

                    \tkzText[above right=0.25cm](B){$\vec{B}$}

                \end{tikzpicture}
            \end{minipage}
        \end{center}

        

\begin{choices}
\choice Paralelo ao papel e da esquerda para a direita. 
\choice Paralelo ao papel e da direita para a esquerda. 
\choice Paralelo ao papel e na vertical. 
\choice Paralelo ao papel e circular no sentido horário. 
\choice Paralelo ao papel e circular no sentido anti-horário. 
\end{choices}
\question Uma partícula de carga 2.72e-06 C é lançada em um campo magnético uniforme de    0.93 T , com uma velocidade de 426.99 m/s. Calcule o valor da força magnética atuando na carga se o ângulo entre a velocidade e o campo magnético for   26.78 graus.

\begin{oneparchoices}
\choice 1.1e-03 N\choice 7.4e-04 N\choice 1.5e-04 N\choice   0.029 N\choice 4.9e-04 N\choice 7.5e-04 N\choice 1.0e-03 N\choice 9.6e-04 N\choice 1.3e-03 N\choice 9.0e-05 N
\end{oneparchoices}\end{multicols*}
\end{questions}
\newpage

    \begin{minipage}[b]{0.75\linewidth}
        \begin{flushleft}
            {\bf \large Prova bimestral}
        \end{flushleft}
        \begin{flushleft}
            {\bf \large LQ2N (2B), 31 de outubro de 2022}
        \end{flushleft}
    \end{minipage}
    \begin{minipage}[b]{0.20\linewidth}
        \begin{flushright}
            {\bf \large Code: 0}
        \end{flushright}
    \end{minipage}
    \vspace{0.5cm} \hrule \vspace{0.5cm}
    \begin{minipage}{0.55\linewidth}
        \begin{flushleft}
            Student: Flaviano W. Fernandes
        \end{flushleft}
    \end{minipage}
    \begin{minipage}{0.20\linewidth}
        \begin{center}
            Date: 2022-11-14
        \end{center}
    \end{minipage}
    \begin{minipage}{0.20\linewidth}
        \begin{flushright}
            Class: LQ2N
        \end{flushright}
    \end{minipage}
    \vspace{0.5cm} \hrule \vspace{0.5cm}
    \begin{questions}
\begin{multicols*}{2}
\question Considere a figura abaixo onde as linhas trajeçadas representam superfícies equipotenciais Se colocarmos um elétron próximo a carga Q, quais trechos possíveis o elétron poderá se deslocar?
        
        \begin{center}
            \begin{minipage}[c]{0.5\linewidth}
                \begin{tikzpicture}[scale=0.5,transform shape, font=\Large]

                \tkzInit[xmin=-4,xmax=4,ymin=-4,ymax=4]
                \tkzClip[space=0.5]

                \tkzDefPoints{0/0/O,4/0/P}

                \foreach \x in {0.5,1.25,2.25,3,4}{
                    \tkzDrawCircle[R,dashed,color=gray!50](O,\x)
                }

                \foreach \y in {0,1,...,11}{
                    \tkzDefPointsBy[rotation= center O angle 30*\y](O,P){P1,P2}
                \draw[->, line width=1.0pt] (O) -- (P2);}

                \tkzDefPoints{3/0/a,4/0/b,0/4/c,0/3/d}

                \tkzDrawPoints[color=red,fill=red,size=0.3cm](a,b,c,d)

                \tkzDrawPoints(O)
                \tkzLabelPoints[above right,font=\Large](a,b,c,d)

                \node[circle, radius=0.25, ball color=gray!50] (n1) at (0,0) {Q};

                \end{tikzpicture}
            \end{minipage}
        \end{center}
        
        

\begin{choices}
\choice $b\rightarrow c$ ou $a\rightarrow d$ 
\choice $b\rightarrow a\rightarrow d\rightarrow c$ ou $c\rightarrow d\rightarrow a\rightarrow b$ 
\choice $c\rightarrow b$ ou $d\rightarrow a$ 
\choice $b\rightarrow a$ ou $c\rightarrow d$ 
\choice $a\rightarrow b$ ou $d\rightarrow c$ 
\end{choices}
\question Uma corrente elétrica de    3.04 A percorre um fio de cobre. Sabendo-se que a carga de um elétron é igual a $1,6\times 10^{-19}\;C$, qual é o número de elétrons que atravessa, por minuto, a seção reta desse fio?

\begin{oneparchoices}
\choice 3.6e+19 \choice 2.9e-17 \choice 4.0e+19 \choice 2.4e+19 \choice 1.1e+21 \choice 4.9e+19 \choice 1.9e+19 \choice 4.9e-19 \choice 3.1e+19 \choice 3.1e+19 
\end{oneparchoices}\question Uma diferença de potencial de 120 V é aplicada a uma bomba d’água. Sabe-se que em funcionamento, o motor da bomba é percorrido por uma corrente de    4.12 A. Qual é a potência desenvolvida nesse motor?

\begin{oneparchoices}
\choice 2.0e+03 W\choice 3.3e+04 W\choice 1.1e+04 W\choice 2.2e+04 W\choice 3.1e+04 W\choice 1.3e+04 W\choice 493.810 W\choice   0.034 W\choice  29.161 W\choice 2.1e+04 W
\end{oneparchoices}\question A figura abaixo mostra a trajetória de uma partícula eletricamente carregada. $\vec{{v}}$ representa a velocidade atravessando um campo magnético $\vec{{B}}$. Determine a sua trajetória devido a ação da força magnética atuando sobre ela.
        
        \begin{center}
            \begin{minipage}[c]{0.5\linewidth}
                \begin{tikzpicture}[scale=0.5,transform shape, font=\Large]

                    \tkzInit[xmin=-3,xmax=3,ymin=-3,ymax=3]
                %	\tkzGrid[color=gray!20]
                    \tkzClip[space=1.0]

                    \tkzDefPoints{0/0/O,4/0/P}

                    \foreach \x in {-2.5,-1.5,...,2.5}{
                        \foreach \y in {-2.5,-1.5,...,2.5}{
                        \tkzDefPoint(\x,\y){B}
                        \tkzText(B){x}
                }
                }

                \draw[->, line width=1pt, color=red] (0,0) --++ (0,1.5) node [above] {$\vec{v}$};

                    \node[circle, radius=0.25, ball color=gray!50] (n1) at (0,0) {+};

                    \tkzText[above right=0.25cm](B){$\vec{B}$}

                \end{tikzpicture}
            \end{minipage}
        \end{center}

        

\begin{choices}
\choice Paralelo ao papel e circular no sentido horário. 
\choice Paralelo ao papel e da esquerda para a direita. 
\choice Paralelo ao papel e na vertical. 
\choice Paralelo ao papel e circular no sentido anti-horário. 
\choice Paralelo ao papel e da direita para a esquerda. 
\end{choices}
\question Uma partícula de carga 3.54e-06 C é lançada em um campo magnético uniforme de    0.64 T , com uma velocidade de 310.68 m/s. Calcule o valor da força magnética atuando na carga se o ângulo entre a velocidade e o campo magnético for   55.32 graus.

\begin{oneparchoices}
\choice 5.8e-04 N\choice 2.2e-03 N\choice 2.2e-04 N\choice 4.0e-04 N\choice 1.1e-04 N\choice 7.4e-04 N\choice 4.0e-03 N\choice -6.7e-04 N\choice 3.2e-03 N\choice   0.039 N
\end{oneparchoices}\end{multicols*}
\end{questions}
\newpage

    \begin{minipage}[b]{0.75\linewidth}
        \begin{flushleft}
            {\bf \large Prova bimestral}
        \end{flushleft}
        \begin{flushleft}
            {\bf \large LQ2N (2B), 31 de outubro de 2022}
        \end{flushleft}
    \end{minipage}
    \begin{minipage}[b]{0.20\linewidth}
        \begin{flushright}
            {\bf \large Code: 1}
        \end{flushright}
    \end{minipage}
    \vspace{0.5cm} \hrule \vspace{0.5cm}
    \begin{minipage}{0.55\linewidth}
        \begin{flushleft}
            Student: Flaviano - IFPR
        \end{flushleft}
    \end{minipage}
    \begin{minipage}{0.20\linewidth}
        \begin{center}
            Date: 2022-11-14
        \end{center}
    \end{minipage}
    \begin{minipage}{0.20\linewidth}
        \begin{flushright}
            Class: LQ2N
        \end{flushright}
    \end{minipage}
    \vspace{0.5cm} \hrule \vspace{0.5cm}
    \begin{questions}
\begin{multicols*}{2}
\question Considere a figura abaixo onde as linhas trajeçadas representam superfícies equipotenciais Se colocarmos um elétron próximo a carga Q, quais trechos possíveis o elétron poderá se deslocar?
        
        \begin{center}
            \begin{minipage}[c]{0.5\linewidth}
                \begin{tikzpicture}[scale=0.5,transform shape, font=\Large]

                \tkzInit[xmin=-4,xmax=4,ymin=-4,ymax=4]
                \tkzClip[space=0.5]

                \tkzDefPoints{0/0/O,4/0/P}

                \foreach \x in {0.5,1.25,2.25,3,4}{
                    \tkzDrawCircle[R,dashed,color=gray!50](O,\x)
                }

                \foreach \y in {0,1,...,11}{
                    \tkzDefPointsBy[rotation= center O angle 30*\y](O,P){P1,P2}
                \draw[->, line width=1.0pt] (O) -- (P2);}

                \tkzDefPoints{3/0/a,4/0/b,0/4/c,0/3/d}

                \tkzDrawPoints[color=red,fill=red,size=0.3cm](a,b,c,d)

                \tkzDrawPoints(O)
                \tkzLabelPoints[above right,font=\Large](a,b,c,d)

                \node[circle, radius=0.25, ball color=gray!50] (n1) at (0,0) {Q};

                \end{tikzpicture}
            \end{minipage}
        \end{center}
        
        

\begin{choices}
\choice $b\rightarrow c$ ou $a\rightarrow d$ 
\choice $b\rightarrow a$ ou $c\rightarrow d$ 
\choice $c\rightarrow b$ ou $d\rightarrow a$ 
\choice $a\rightarrow b$ ou $d\rightarrow c$ 
\choice $b\rightarrow a\rightarrow d\rightarrow c$ ou $c\rightarrow d\rightarrow a\rightarrow b$ 
\end{choices}
\question Uma corrente elétrica de    6.82 A percorre um fio de cobre. Sabendo-se que a carga de um elétron é igual a $1,6\times 10^{-19}\;C$, qual é o número de elétrons que atravessa, por minuto, a seção reta desse fio?

\begin{oneparchoices}
\choice 9.6e+19 \choice 1.1e-18 \choice 5.8e+19 \choice 8.5e+19 \choice 7.7e+19 \choice 6.6e-17 \choice 4.3e+19 \choice 3.1e+19 \choice 1.4e+19 \choice 2.6e+21 
\end{oneparchoices}\question Uma diferença de potencial de 120 V é aplicada a uma bomba d’água. Sabe-se que em funcionamento, o motor da bomba é percorrido por uma corrente de    2.96 A. Qual é a potência desenvolvida nesse motor?

\begin{oneparchoices}
\choice 2.2e+04 W\choice  40.585 W\choice 2.7e+04 W\choice 2.4e+04 W\choice 3.4e+04 W\choice 2.1e+04 W\choice 2.7e+04 W\choice   0.025 W\choice 1.0e+03 W\choice 354.815 W
\end{oneparchoices}\question A figura abaixo mostra a trajetória de uma partícula eletricamente carregada. $\vec{{v}}$ representa a velocidade atravessando um campo magnético $\vec{{B}}$. Determine a sua trajetória devido a ação da força magnética atuando sobre ela.
        
        \begin{center}
            \begin{minipage}[c]{0.5\linewidth}
                \begin{tikzpicture}[scale=0.5,transform shape, font=\Large]

                    \tkzInit[xmin=-3,xmax=3,ymin=-3,ymax=3]
                %	\tkzGrid[color=gray!20]
                    \tkzClip[space=1.0]

                    \tkzDefPoints{0/0/O,4/0/P}

                    \foreach \x in {-2.5,-1.5,...,2.5}{
                        \foreach \y in {-2.5,-1.5,...,2.5}{
                        \tkzDefPoint(\x,\y){B}
                        \tkzText(B){x}
                }
                }

                \draw[->, line width=1pt, color=red] (0,0) --++ (0,1.5) node [above] {$\vec{v}$};

                    \node[circle, radius=0.25, ball color=gray!50] (n1) at (0,0) {+};

                    \tkzText[above right=0.25cm](B){$\vec{B}$}

                \end{tikzpicture}
            \end{minipage}
        \end{center}

        

\begin{choices}
\choice Paralelo ao papel e da direita para a esquerda. 
\choice Paralelo ao papel e na vertical. 
\choice Paralelo ao papel e da esquerda para a direita. 
\choice Paralelo ao papel e circular no sentido anti-horário. 
\choice Paralelo ao papel e circular no sentido horário. 
\end{choices}
\question Uma partícula de carga 6.88e-06 C é lançada em um campo magnético uniforme de    0.54 T , com uma velocidade de 626.70 m/s. Calcule o valor da força magnética atuando na carga se o ângulo entre a velocidade e o campo magnético for   83.65 graus.

\begin{oneparchoices}
\choice 5.1e-04 N\choice 2.6e-04 N\choice   0.194 N\choice 2.9e-04 N\choice 2.3e-03 N\choice 4.9e-04 N\choice 2.6e-04 N\choice 2.1e-03 N\choice 1.4e-04 N\choice 8.7e-04 N
\end{oneparchoices}\end{multicols*}
\end{questions}
\newpage

    \begin{minipage}[b]{0.75\linewidth}
        \begin{flushleft}
            {\bf \large Prova bimestral}
        \end{flushleft}
        \begin{flushleft}
            {\bf \large LQ2N (2B), 31 de outubro de 2022}
        \end{flushleft}
    \end{minipage}
    \begin{minipage}[b]{0.20\linewidth}
        \begin{flushright}
            {\bf \large Code: 0}
        \end{flushright}
    \end{minipage}
    \vspace{0.5cm} \hrule \vspace{0.5cm}
    \begin{minipage}{0.55\linewidth}
        \begin{flushleft}
            Student: Flaviano W. Fernandes
        \end{flushleft}
    \end{minipage}
    \begin{minipage}{0.20\linewidth}
        \begin{center}
            Date: 2022-11-14
        \end{center}
    \end{minipage}
    \begin{minipage}{0.20\linewidth}
        \begin{flushright}
            Class: LQ2N
        \end{flushright}
    \end{minipage}
    \vspace{0.5cm} \hrule \vspace{0.5cm}
    \begin{questions}
\begin{multicols*}{2}
\question Considere a figura abaixo onde as linhas trajeçadas representam superfícies equipotenciais Se colocarmos um elétron próximo a carga Q, quais trechos possíveis o elétron poderá se deslocar?
        
        \begin{center}
            \begin{minipage}[c]{0.5\linewidth}
                \begin{tikzpicture}[scale=0.5,transform shape, font=\Large]

                \tkzInit[xmin=-4,xmax=4,ymin=-4,ymax=4]
                \tkzClip[space=0.5]

                \tkzDefPoints{0/0/O,4/0/P}

                \foreach \x in {0.5,1.25,2.25,3,4}{
                    \tkzDrawCircle[R,dashed,color=gray!50](O,\x)
                }

                \foreach \y in {0,1,...,11}{
                    \tkzDefPointsBy[rotation= center O angle 30*\y](O,P){P1,P2}
                \draw[->, line width=1.0pt] (O) -- (P2);}

                \tkzDefPoints{3/0/a,4/0/b,0/4/c,0/3/d}

                \tkzDrawPoints[color=red,fill=red,size=0.3cm](a,b,c,d)

                \tkzDrawPoints(O)
                \tkzLabelPoints[above right,font=\Large](a,b,c,d)

                \node[circle, radius=0.25, ball color=gray!50] (n1) at (0,0) {Q};

                \end{tikzpicture}
            \end{minipage}
        \end{center}
        
        

\begin{choices}
\choice $a\rightarrow b$ ou $d\rightarrow c$ 
\choice $b\rightarrow a\rightarrow d\rightarrow c$ ou $c\rightarrow d\rightarrow a\rightarrow b$ 
\choice $c\rightarrow b$ ou $d\rightarrow a$ 
\choice $b\rightarrow c$ ou $a\rightarrow d$ 
\choice $b\rightarrow a$ ou $c\rightarrow d$ 
\end{choices}
\question Uma corrente elétrica de    2.24 A percorre um fio de cobre. Sabendo-se que a carga de um elétron é igual a $1,6\times 10^{-19}\;C$, qual é o número de elétrons que atravessa, por minuto, a seção reta desse fio?

\begin{oneparchoices}
\choice 5.6e+19 \choice 1.4e+19 \choice 2.1e-17 \choice 9.6e+19 \choice 1.7e+19 \choice 3.2e+19 \choice 4.7e+19 \choice 3.6e-19 \choice 8.4e+20 \choice 7.1e+19 
\end{oneparchoices}\question Uma diferença de potencial de 120 V é aplicada a uma bomba d’água. Sabe-se que em funcionamento, o motor da bomba é percorrido por uma corrente de    1.07 A. Qual é a potência desenvolvida nesse motor?

\begin{oneparchoices}
\choice 2.2e+04 W\choice 8.9e-03 W\choice 4.5e+03 W\choice 2.4e+04 W\choice 3.5e+04 W\choice 3.3e+04 W\choice 137.087 W\choice 128.259 W\choice 112.273 W\choice 2.7e+04 W
\end{oneparchoices}\question A figura abaixo mostra a trajetória de uma partícula eletricamente carregada. $\vec{{v}}$ representa a velocidade atravessando um campo magnético $\vec{{B}}$. Determine a sua trajetória devido a ação da força magnética atuando sobre ela.
        
        \begin{center}
            \begin{minipage}[c]{0.5\linewidth}
                \begin{tikzpicture}[scale=0.5,transform shape, font=\Large]

                    \tkzInit[xmin=-3,xmax=3,ymin=-3,ymax=3]
                %	\tkzGrid[color=gray!20]
                    \tkzClip[space=1.0]

                    \tkzDefPoints{0/0/O,4/0/P}

                    \foreach \x in {-2.5,-1.5,...,2.5}{
                        \foreach \y in {-2.5,-1.5,...,2.5}{
                        \tkzDefPoint(\x,\y){B}
                        \tkzText(B){x}
                }
                }

                \draw[->, line width=1pt, color=red] (0,0) --++ (0,1.5) node [above] {$\vec{v}$};

                    \node[circle, radius=0.25, ball color=gray!50] (n1) at (0,0) {+};

                    \tkzText[above right=0.25cm](B){$\vec{B}$}

                \end{tikzpicture}
            \end{minipage}
        \end{center}

        

\begin{choices}
\choice Paralelo ao papel e circular no sentido horário. 
\choice Paralelo ao papel e circular no sentido anti-horário. 
\choice Paralelo ao papel e na vertical. 
\choice Paralelo ao papel e da direita para a esquerda. 
\choice Paralelo ao papel e da esquerda para a direita. 
\end{choices}
\question Uma partícula de carga 9.49e-06 C é lançada em um campo magnético uniforme de    0.91 T , com uma velocidade de 297.15 m/s. Calcule o valor da força magnética atuando na carga se o ângulo entre a velocidade e o campo magnético for   36.05 graus.

\begin{oneparchoices}
\choice 5.3e-04 N\choice   0.093 N\choice 2.3e-03 N\choice 9.6e-06 N\choice 1.5e-03 N\choice 2.1e-03 N\choice 1.6e-03 N\choice 4.0e-04 N\choice 3.0e-03 N\choice -2.6e-03 N
\end{oneparchoices}\end{multicols*}
\end{questions}
\newpage

    \begin{minipage}[b]{0.75\linewidth}
        \begin{flushleft}
            {\bf \large Prova bimestral}
        \end{flushleft}
        \begin{flushleft}
            {\bf \large LQ2N (2B), 31 de outubro de 2022}
        \end{flushleft}
    \end{minipage}
    \begin{minipage}[b]{0.20\linewidth}
        \begin{flushright}
            {\bf \large Code: 1}
        \end{flushright}
    \end{minipage}
    \vspace{0.5cm} \hrule \vspace{0.5cm}
    \begin{minipage}{0.55\linewidth}
        \begin{flushleft}
            Student: Flaviano - IFPR
        \end{flushleft}
    \end{minipage}
    \begin{minipage}{0.20\linewidth}
        \begin{center}
            Date: 2022-11-14
        \end{center}
    \end{minipage}
    \begin{minipage}{0.20\linewidth}
        \begin{flushright}
            Class: LQ2N
        \end{flushright}
    \end{minipage}
    \vspace{0.5cm} \hrule \vspace{0.5cm}
    \begin{questions}
\begin{multicols*}{2}
\question Considere a figura abaixo onde as linhas trajeçadas representam superfícies equipotenciais Se colocarmos um elétron próximo a carga Q, quais trechos possíveis o elétron poderá se deslocar?
        
        \begin{center}
            \begin{minipage}[c]{0.5\linewidth}
                \begin{tikzpicture}[scale=0.5,transform shape, font=\Large]

                \tkzInit[xmin=-4,xmax=4,ymin=-4,ymax=4]
                \tkzClip[space=0.5]

                \tkzDefPoints{0/0/O,4/0/P}

                \foreach \x in {0.5,1.25,2.25,3,4}{
                    \tkzDrawCircle[R,dashed,color=gray!50](O,\x)
                }

                \foreach \y in {0,1,...,11}{
                    \tkzDefPointsBy[rotation= center O angle 30*\y](O,P){P1,P2}
                \draw[->, line width=1.0pt] (O) -- (P2);}

                \tkzDefPoints{3/0/a,4/0/b,0/4/c,0/3/d}

                \tkzDrawPoints[color=red,fill=red,size=0.3cm](a,b,c,d)

                \tkzDrawPoints(O)
                \tkzLabelPoints[above right,font=\Large](a,b,c,d)

                \node[circle, radius=0.25, ball color=gray!50] (n1) at (0,0) {Q};

                \end{tikzpicture}
            \end{minipage}
        \end{center}
        
        

\begin{choices}
\choice $b\rightarrow a\rightarrow d\rightarrow c$ ou $c\rightarrow d\rightarrow a\rightarrow b$ 
\choice $b\rightarrow c$ ou $a\rightarrow d$ 
\choice $c\rightarrow b$ ou $d\rightarrow a$ 
\choice $a\rightarrow b$ ou $d\rightarrow c$ 
\choice $b\rightarrow a$ ou $c\rightarrow d$ 
\end{choices}
\question Uma corrente elétrica de    1.38 A percorre um fio de cobre. Sabendo-se que a carga de um elétron é igual a $1,6\times 10^{-19}\;C$, qual é o número de elétrons que atravessa, por minuto, a seção reta desse fio?

\begin{oneparchoices}
\choice 4.6e+19 \choice 2.2e-19 \choice 1.3e-17 \choice 5.2e+20 \choice 9.9e+19 \choice 7.6e+19 \choice 8.2e+19 \choice 8.6e+18 \choice 9.6e+19 \choice 6.4e+19 
\end{oneparchoices}\question Uma diferença de potencial de 120 V é aplicada a uma bomba d’água. Sabe-se que em funcionamento, o motor da bomba é percorrido por uma corrente de    2.67 A. Qual é a potência desenvolvida nesse motor?

\begin{oneparchoices}
\choice 1.2e+04 W\choice 1.8e+04 W\choice 3.1e+04 W\choice 2.6e+04 W\choice 855.979 W\choice   0.022 W\choice 320.496 W\choice  44.930 W\choice 3.1e+04 W\choice 3.4e+04 W
\end{oneparchoices}\question A figura abaixo mostra a trajetória de uma partícula eletricamente carregada. $\vec{{v}}$ representa a velocidade atravessando um campo magnético $\vec{{B}}$. Determine a sua trajetória devido a ação da força magnética atuando sobre ela.
        
        \begin{center}
            \begin{minipage}[c]{0.5\linewidth}
                \begin{tikzpicture}[scale=0.5,transform shape, font=\Large]

                    \tkzInit[xmin=-3,xmax=3,ymin=-3,ymax=3]
                %	\tkzGrid[color=gray!20]
                    \tkzClip[space=1.0]

                    \tkzDefPoints{0/0/O,4/0/P}

                    \foreach \x in {-2.5,-1.5,...,2.5}{
                        \foreach \y in {-2.5,-1.5,...,2.5}{
                        \tkzDefPoint(\x,\y){B}
                        \tkzText(B){x}
                }
                }

                \draw[->, line width=1pt, color=red] (0,0) --++ (0,1.5) node [above] {$\vec{v}$};

                    \node[circle, radius=0.25, ball color=gray!50] (n1) at (0,0) {+};

                    \tkzText[above right=0.25cm](B){$\vec{B}$}

                \end{tikzpicture}
            \end{minipage}
        \end{center}

        

\begin{choices}
\choice Paralelo ao papel e da direita para a esquerda. 
\choice Paralelo ao papel e circular no sentido horário. 
\choice Paralelo ao papel e na vertical. 
\choice Paralelo ao papel e da esquerda para a direita. 
\choice Paralelo ao papel e circular no sentido anti-horário. 
\end{choices}
\question Uma partícula de carga 9.61e-06 C é lançada em um campo magnético uniforme de    0.24 T , com uma velocidade de 633.29 m/s. Calcule o valor da força magnética atuando na carga se o ângulo entre a velocidade e o campo magnético for   60.65 graus.

\begin{oneparchoices}
\choice 1.4e-04 N\choice 4.2e-04 N\choice 7.2e-04 N\choice 3.6e-03 N\choice 3.3e-05 N\choice -1.2e-03 N\choice 2.0e-04 N\choice 1.3e-03 N\choice 5.3e-04 N\choice   0.089 N
\end{oneparchoices}\end{multicols*}
\end{questions}
\newpage

    \begin{minipage}[b]{0.75\linewidth}
        \begin{flushleft}
            {\bf \large Prova bimestral}
        \end{flushleft}
        \begin{flushleft}
            {\bf \large LQ2N (2B), 31 de outubro de 2022}
        \end{flushleft}
    \end{minipage}
    \begin{minipage}[b]{0.20\linewidth}
        \begin{flushright}
            {\bf \large Code: 0}
        \end{flushright}
    \end{minipage}
    \vspace{0.5cm} \hrule \vspace{0.5cm}
    \begin{minipage}{0.55\linewidth}
        \begin{flushleft}
            Student: Flaviano W. Fernandes
        \end{flushleft}
    \end{minipage}
    \begin{minipage}{0.20\linewidth}
        \begin{center}
            Date: 2022-11-14
        \end{center}
    \end{minipage}
    \begin{minipage}{0.20\linewidth}
        \begin{flushright}
            Class: LQ2N
        \end{flushright}
    \end{minipage}
    \vspace{0.5cm} \hrule \vspace{0.5cm}
    \begin{questions}
\begin{multicols*}{2}
\question Considere a figura abaixo onde as linhas trajeçadas representam superfícies equipotenciais Se colocarmos um elétron próximo a carga Q, quais trechos possíveis o elétron poderá se deslocar?
        
        \begin{center}
            \begin{minipage}[c]{0.5\linewidth}
                \begin{tikzpicture}[scale=0.5,transform shape, font=\Large]

                \tkzInit[xmin=-4,xmax=4,ymin=-4,ymax=4]
                \tkzClip[space=0.5]

                \tkzDefPoints{0/0/O,4/0/P}

                \foreach \x in {0.5,1.25,2.25,3,4}{
                    \tkzDrawCircle[R,dashed,color=gray!50](O,\x)
                }

                \foreach \y in {0,1,...,11}{
                    \tkzDefPointsBy[rotation= center O angle 30*\y](O,P){P1,P2}
                \draw[->, line width=1.0pt] (O) -- (P2);}

                \tkzDefPoints{3/0/a,4/0/b,0/4/c,0/3/d}

                \tkzDrawPoints[color=red,fill=red,size=0.3cm](a,b,c,d)

                \tkzDrawPoints(O)
                \tkzLabelPoints[above right,font=\Large](a,b,c,d)

                \node[circle, radius=0.25, ball color=gray!50] (n1) at (0,0) {Q};

                \end{tikzpicture}
            \end{minipage}
        \end{center}
        
        

\begin{choices}
\choice $b\rightarrow a$ ou $c\rightarrow d$ 
\choice $c\rightarrow b$ ou $d\rightarrow a$ 
\choice $b\rightarrow a\rightarrow d\rightarrow c$ ou $c\rightarrow d\rightarrow a\rightarrow b$ 
\choice $a\rightarrow b$ ou $d\rightarrow c$ 
\choice $b\rightarrow c$ ou $a\rightarrow d$ 
\end{choices}
\question Uma corrente elétrica de    6.66 A percorre um fio de cobre. Sabendo-se que a carga de um elétron é igual a $1,6\times 10^{-19}\;C$, qual é o número de elétrons que atravessa, por minuto, a seção reta desse fio?

\begin{oneparchoices}
\choice 4.4e+19 \choice 4.2e+19 \choice 9.2e+19 \choice 4.2e+19 \choice 5.8e+19 \choice 6.4e-17 \choice 1.1e-18 \choice 2.5e+21 \choice 8.5e+19 \choice 5.5e+19 
\end{oneparchoices}\question Uma diferença de potencial de 120 V é aplicada a uma bomba d’água. Sabe-se que em funcionamento, o motor da bomba é percorrido por uma corrente de    4.92 A. Qual é a potência desenvolvida nesse motor?

\begin{oneparchoices}
\choice 1.6e+04 W\choice 2.9e+03 W\choice 8.5e+03 W\choice   0.041 W\choice 1.7e+04 W\choice 2.9e+03 W\choice  24.408 W\choice 2.5e+04 W\choice 589.965 W\choice 2.6e+04 W
\end{oneparchoices}\question A figura abaixo mostra a trajetória de uma partícula eletricamente carregada. $\vec{{v}}$ representa a velocidade atravessando um campo magnético $\vec{{B}}$. Determine a sua trajetória devido a ação da força magnética atuando sobre ela.
        
        \begin{center}
            \begin{minipage}[c]{0.5\linewidth}
                \begin{tikzpicture}[scale=0.5,transform shape, font=\Large]

                    \tkzInit[xmin=-3,xmax=3,ymin=-3,ymax=3]
                %	\tkzGrid[color=gray!20]
                    \tkzClip[space=1.0]

                    \tkzDefPoints{0/0/O,4/0/P}

                    \foreach \x in {-2.5,-1.5,...,2.5}{
                        \foreach \y in {-2.5,-1.5,...,2.5}{
                        \tkzDefPoint(\x,\y){B}
                        \tkzText(B){x}
                }
                }

                \draw[->, line width=1pt, color=red] (0,0) --++ (0,1.5) node [above] {$\vec{v}$};

                    \node[circle, radius=0.25, ball color=gray!50] (n1) at (0,0) {+};

                    \tkzText[above right=0.25cm](B){$\vec{B}$}

                \end{tikzpicture}
            \end{minipage}
        \end{center}

        

\begin{choices}
\choice Paralelo ao papel e da esquerda para a direita. 
\choice Paralelo ao papel e da direita para a esquerda. 
\choice Paralelo ao papel e circular no sentido horário. 
\choice Paralelo ao papel e na vertical. 
\choice Paralelo ao papel e circular no sentido anti-horário. 
\end{choices}
\question Uma partícula de carga 5.66e-06 C é lançada em um campo magnético uniforme de    0.50 T , com uma velocidade de 268.57 m/s. Calcule o valor da força magnética atuando na carga se o ângulo entre a velocidade e o campo magnético for   17.06 graus.

\begin{oneparchoices}
\choice   0.013 N\choice 2.6e-04 N\choice 2.2e-04 N\choice 7.3e-04 N\choice 2.4e-03 N\choice 8.8e-05 N\choice 2.2e-03 N\choice 1.9e-03 N\choice 2.4e-04 N\choice -7.5e-04 N
\end{oneparchoices}\end{multicols*}
\end{questions}
\newpage

    \begin{minipage}[b]{0.75\linewidth}
        \begin{flushleft}
            {\bf \large Prova bimestral}
        \end{flushleft}
        \begin{flushleft}
            {\bf \large LQ2N (2B), 31 de outubro de 2022}
        \end{flushleft}
    \end{minipage}
    \begin{minipage}[b]{0.20\linewidth}
        \begin{flushright}
            {\bf \large Code: 1}
        \end{flushright}
    \end{minipage}
    \vspace{0.5cm} \hrule \vspace{0.5cm}
    \begin{minipage}{0.55\linewidth}
        \begin{flushleft}
            Student: Flaviano - IFPR
        \end{flushleft}
    \end{minipage}
    \begin{minipage}{0.20\linewidth}
        \begin{center}
            Date: 2022-11-14
        \end{center}
    \end{minipage}
    \begin{minipage}{0.20\linewidth}
        \begin{flushright}
            Class: LQ2N
        \end{flushright}
    \end{minipage}
    \vspace{0.5cm} \hrule \vspace{0.5cm}
    \begin{questions}
\begin{multicols*}{2}
\question Considere a figura abaixo onde as linhas trajeçadas representam superfícies equipotenciais Se colocarmos um elétron próximo a carga Q, quais trechos possíveis o elétron poderá se deslocar?
        
        \begin{center}
            \begin{minipage}[c]{0.5\linewidth}
                \begin{tikzpicture}[scale=0.5,transform shape, font=\Large]

                \tkzInit[xmin=-4,xmax=4,ymin=-4,ymax=4]
                \tkzClip[space=0.5]

                \tkzDefPoints{0/0/O,4/0/P}

                \foreach \x in {0.5,1.25,2.25,3,4}{
                    \tkzDrawCircle[R,dashed,color=gray!50](O,\x)
                }

                \foreach \y in {0,1,...,11}{
                    \tkzDefPointsBy[rotation= center O angle 30*\y](O,P){P1,P2}
                \draw[->, line width=1.0pt] (O) -- (P2);}

                \tkzDefPoints{3/0/a,4/0/b,0/4/c,0/3/d}

                \tkzDrawPoints[color=red,fill=red,size=0.3cm](a,b,c,d)

                \tkzDrawPoints(O)
                \tkzLabelPoints[above right,font=\Large](a,b,c,d)

                \node[circle, radius=0.25, ball color=gray!50] (n1) at (0,0) {Q};

                \end{tikzpicture}
            \end{minipage}
        \end{center}
        
        

\begin{choices}
\choice $b\rightarrow a\rightarrow d\rightarrow c$ ou $c\rightarrow d\rightarrow a\rightarrow b$ 
\choice $b\rightarrow a$ ou $c\rightarrow d$ 
\choice $b\rightarrow c$ ou $a\rightarrow d$ 
\choice $a\rightarrow b$ ou $d\rightarrow c$ 
\choice $c\rightarrow b$ ou $d\rightarrow a$ 
\end{choices}
\question Uma corrente elétrica de    1.17 A percorre um fio de cobre. Sabendo-se que a carga de um elétron é igual a $1,6\times 10^{-19}\;C$, qual é o número de elétrons que atravessa, por minuto, a seção reta desse fio?

\begin{oneparchoices}
\choice 8.0e+19 \choice 7.3e+18 \choice 4.4e+20 \choice 1.0e+19 \choice 1.1e-17 \choice 8.2e+19 \choice 1.9e-19 \choice 2.3e+19 \choice 1.3e+19 \choice 2.5e+19 
\end{oneparchoices}\question Uma diferença de potencial de 120 V é aplicada a uma bomba d’água. Sabe-se que em funcionamento, o motor da bomba é percorrido por uma corrente de    2.98 A. Qual é a potência desenvolvida nesse motor?

\begin{oneparchoices}
\choice 3.9e+03 W\choice 1.1e+03 W\choice 1.2e+04 W\choice 1.9e+04 W\choice 8.7e+03 W\choice 357.015 W\choice 3.4e+04 W\choice  40.334 W\choice 2.9e+03 W\choice   0.025 W
\end{oneparchoices}\question A figura abaixo mostra a trajetória de uma partícula eletricamente carregada. $\vec{{v}}$ representa a velocidade atravessando um campo magnético $\vec{{B}}$. Determine a sua trajetória devido a ação da força magnética atuando sobre ela.
        
        \begin{center}
            \begin{minipage}[c]{0.5\linewidth}
                \begin{tikzpicture}[scale=0.5,transform shape, font=\Large]

                    \tkzInit[xmin=-3,xmax=3,ymin=-3,ymax=3]
                %	\tkzGrid[color=gray!20]
                    \tkzClip[space=1.0]

                    \tkzDefPoints{0/0/O,4/0/P}

                    \foreach \x in {-2.5,-1.5,...,2.5}{
                        \foreach \y in {-2.5,-1.5,...,2.5}{
                        \tkzDefPoint(\x,\y){B}
                        \tkzText(B){x}
                }
                }

                \draw[->, line width=1pt, color=red] (0,0) --++ (0,1.5) node [above] {$\vec{v}$};

                    \node[circle, radius=0.25, ball color=gray!50] (n1) at (0,0) {+};

                    \tkzText[above right=0.25cm](B){$\vec{B}$}

                \end{tikzpicture}
            \end{minipage}
        \end{center}

        

\begin{choices}
\choice Paralelo ao papel e da esquerda para a direita. 
\choice Paralelo ao papel e da direita para a esquerda. 
\choice Paralelo ao papel e na vertical. 
\choice Paralelo ao papel e circular no sentido horário. 
\choice Paralelo ao papel e circular no sentido anti-horário. 
\end{choices}
\question Uma partícula de carga 9.47e-06 C é lançada em um campo magnético uniforme de    0.96 T , com uma velocidade de 249.06 m/s. Calcule o valor da força magnética atuando na carga se o ângulo entre a velocidade e o campo magnético for   68.17 graus.

\begin{oneparchoices}
\choice -1.8e-03 N\choice 5.2e-04 N\choice 4.3e-04 N\choice   0.154 N\choice 3.2e-05 N\choice 1.1e-03 N\choice 2.1e-03 N\choice 1.5e-04 N\choice 8.4e-04 N\choice 5.6e-03 N
\end{oneparchoices}\end{multicols*}
\end{questions}
\newpage

    \begin{minipage}[b]{0.75\linewidth}
        \begin{flushleft}
            {\bf \large Prova bimestral}
        \end{flushleft}
        \begin{flushleft}
            {\bf \large LQ2N (2B), 31 de outubro de 2022}
        \end{flushleft}
    \end{minipage}
    \begin{minipage}[b]{0.20\linewidth}
        \begin{flushright}
            {\bf \large Code: 0}
        \end{flushright}
    \end{minipage}
    \vspace{0.5cm} \hrule \vspace{0.5cm}
    \begin{minipage}{0.55\linewidth}
        \begin{flushleft}
            Student: Flaviano W. Fernandes
        \end{flushleft}
    \end{minipage}
    \begin{minipage}{0.20\linewidth}
        \begin{center}
            Date: 2022-11-14
        \end{center}
    \end{minipage}
    \begin{minipage}{0.20\linewidth}
        \begin{flushright}
            Class: LQ2N
        \end{flushright}
    \end{minipage}
    \vspace{0.5cm} \hrule \vspace{0.5cm}
    \begin{questions}
\begin{multicols*}{2}
\question Considere a figura abaixo onde as linhas trajeçadas representam superfícies equipotenciais Se colocarmos um elétron próximo a carga Q, quais trechos possíveis o elétron poderá se deslocar?
        
        \begin{center}
            \begin{minipage}[c]{0.5\linewidth}
                \begin{tikzpicture}[scale=0.5,transform shape, font=\Large]

                \tkzInit[xmin=-4,xmax=4,ymin=-4,ymax=4]
                \tkzClip[space=0.5]

                \tkzDefPoints{0/0/O,4/0/P}

                \foreach \x in {0.5,1.25,2.25,3,4}{
                    \tkzDrawCircle[R,dashed,color=gray!50](O,\x)
                }

                \foreach \y in {0,1,...,11}{
                    \tkzDefPointsBy[rotation= center O angle 30*\y](O,P){P1,P2}
                \draw[->, line width=1.0pt] (O) -- (P2);}

                \tkzDefPoints{3/0/a,4/0/b,0/4/c,0/3/d}

                \tkzDrawPoints[color=red,fill=red,size=0.3cm](a,b,c,d)

                \tkzDrawPoints(O)
                \tkzLabelPoints[above right,font=\Large](a,b,c,d)

                \node[circle, radius=0.25, ball color=gray!50] (n1) at (0,0) {Q};

                \end{tikzpicture}
            \end{minipage}
        \end{center}
        
        

\begin{choices}
\choice $a\rightarrow b$ ou $d\rightarrow c$ 
\choice $c\rightarrow b$ ou $d\rightarrow a$ 
\choice $b\rightarrow a\rightarrow d\rightarrow c$ ou $c\rightarrow d\rightarrow a\rightarrow b$ 
\choice $b\rightarrow a$ ou $c\rightarrow d$ 
\choice $b\rightarrow c$ ou $a\rightarrow d$ 
\end{choices}
\question Uma corrente elétrica de    8.77 A percorre um fio de cobre. Sabendo-se que a carga de um elétron é igual a $1,6\times 10^{-19}\;C$, qual é o número de elétrons que atravessa, por minuto, a seção reta desse fio?

\begin{oneparchoices}
\choice 1.4e-18 \choice 7.7e+19 \choice 5.5e+19 \choice 8.4e-17 \choice 2.9e+19 \choice 7.0e+19 \choice 9.5e+19 \choice 8.8e+19 \choice 3.3e+21 \choice 9.9e+19 
\end{oneparchoices}\question Uma diferença de potencial de 120 V é aplicada a uma bomba d’água. Sabe-se que em funcionamento, o motor da bomba é percorrido por uma corrente de    4.14 A. Qual é a potência desenvolvida nesse motor?

\begin{oneparchoices}
\choice  29.020 W\choice 496.205 W\choice 1.2e+04 W\choice 1.8e+03 W\choice   0.034 W\choice 2.8e+04 W\choice 2.1e+03 W\choice 885.819 W\choice 1.3e+03 W\choice 1.2e+04 W
\end{oneparchoices}\question A figura abaixo mostra a trajetória de uma partícula eletricamente carregada. $\vec{{v}}$ representa a velocidade atravessando um campo magnético $\vec{{B}}$. Determine a sua trajetória devido a ação da força magnética atuando sobre ela.
        
        \begin{center}
            \begin{minipage}[c]{0.5\linewidth}
                \begin{tikzpicture}[scale=0.5,transform shape, font=\Large]

                    \tkzInit[xmin=-3,xmax=3,ymin=-3,ymax=3]
                %	\tkzGrid[color=gray!20]
                    \tkzClip[space=1.0]

                    \tkzDefPoints{0/0/O,4/0/P}

                    \foreach \x in {-2.5,-1.5,...,2.5}{
                        \foreach \y in {-2.5,-1.5,...,2.5}{
                        \tkzDefPoint(\x,\y){B}
                        \tkzText(B){x}
                }
                }

                \draw[->, line width=1pt, color=red] (0,0) --++ (0,1.5) node [above] {$\vec{v}$};

                    \node[circle, radius=0.25, ball color=gray!50] (n1) at (0,0) {+};

                    \tkzText[above right=0.25cm](B){$\vec{B}$}

                \end{tikzpicture}
            \end{minipage}
        \end{center}

        

\begin{choices}
\choice Paralelo ao papel e circular no sentido horário. 
\choice Paralelo ao papel e da esquerda para a direita. 
\choice Paralelo ao papel e circular no sentido anti-horário. 
\choice Paralelo ao papel e da direita para a esquerda. 
\choice Paralelo ao papel e na vertical. 
\end{choices}
\question Uma partícula de carga 5.70e-06 C é lançada em um campo magnético uniforme de    0.73 T , com uma velocidade de 434.68 m/s. Calcule o valor da força magnética atuando na carga se o ângulo entre a velocidade e o campo magnético for   21.76 graus.

\begin{oneparchoices}
\choice 9.1e-04 N\choice 7.7e-04 N\choice 8.6e-04 N\choice 1.1e-03 N\choice 1.7e-03 N\choice 4.1e-04 N\choice 1.7e-03 N\choice   0.039 N\choice 6.7e-04 N\choice 5.6e-04 N
\end{oneparchoices}\end{multicols*}
\end{questions}
\newpage

    \begin{minipage}[b]{0.75\linewidth}
        \begin{flushleft}
            {\bf \large Prova bimestral}
        \end{flushleft}
        \begin{flushleft}
            {\bf \large LQ2N (2B), 31 de outubro de 2022}
        \end{flushleft}
    \end{minipage}
    \begin{minipage}[b]{0.20\linewidth}
        \begin{flushright}
            {\bf \large Code: 1}
        \end{flushright}
    \end{minipage}
    \vspace{0.5cm} \hrule \vspace{0.5cm}
    \begin{minipage}{0.55\linewidth}
        \begin{flushleft}
            Student: Flaviano - IFPR
        \end{flushleft}
    \end{minipage}
    \begin{minipage}{0.20\linewidth}
        \begin{center}
            Date: 2022-11-14
        \end{center}
    \end{minipage}
    \begin{minipage}{0.20\linewidth}
        \begin{flushright}
            Class: LQ2N
        \end{flushright}
    \end{minipage}
    \vspace{0.5cm} \hrule \vspace{0.5cm}
    \begin{questions}
\begin{multicols*}{2}
\question Considere a figura abaixo onde as linhas trajeçadas representam superfícies equipotenciais Se colocarmos um elétron próximo a carga Q, quais trechos possíveis o elétron poderá se deslocar?
        
        \begin{center}
            \begin{minipage}[c]{0.5\linewidth}
                \begin{tikzpicture}[scale=0.5,transform shape, font=\Large]

                \tkzInit[xmin=-4,xmax=4,ymin=-4,ymax=4]
                \tkzClip[space=0.5]

                \tkzDefPoints{0/0/O,4/0/P}

                \foreach \x in {0.5,1.25,2.25,3,4}{
                    \tkzDrawCircle[R,dashed,color=gray!50](O,\x)
                }

                \foreach \y in {0,1,...,11}{
                    \tkzDefPointsBy[rotation= center O angle 30*\y](O,P){P1,P2}
                \draw[->, line width=1.0pt] (O) -- (P2);}

                \tkzDefPoints{3/0/a,4/0/b,0/4/c,0/3/d}

                \tkzDrawPoints[color=red,fill=red,size=0.3cm](a,b,c,d)

                \tkzDrawPoints(O)
                \tkzLabelPoints[above right,font=\Large](a,b,c,d)

                \node[circle, radius=0.25, ball color=gray!50] (n1) at (0,0) {Q};

                \end{tikzpicture}
            \end{minipage}
        \end{center}
        
        

\begin{choices}
\choice $b\rightarrow c$ ou $a\rightarrow d$ 
\choice $b\rightarrow a\rightarrow d\rightarrow c$ ou $c\rightarrow d\rightarrow a\rightarrow b$ 
\choice $a\rightarrow b$ ou $d\rightarrow c$ 
\choice $b\rightarrow a$ ou $c\rightarrow d$ 
\choice $c\rightarrow b$ ou $d\rightarrow a$ 
\end{choices}
\question Uma corrente elétrica de    8.34 A percorre um fio de cobre. Sabendo-se que a carga de um elétron é igual a $1,6\times 10^{-19}\;C$, qual é o número de elétrons que atravessa, por minuto, a seção reta desse fio?

\begin{oneparchoices}
\choice 1.4e+19 \choice 3.1e+21 \choice 4.5e+19 \choice 7.7e+19 \choice 7.3e+19 \choice 5.2e+19 \choice 7.4e+19 \choice 8.0e-17 \choice 1.3e-18 \choice 7.6e+19 
\end{oneparchoices}\question Uma diferença de potencial de 120 V é aplicada a uma bomba d’água. Sabe-se que em funcionamento, o motor da bomba é percorrido por uma corrente de    2.01 A. Qual é a potência desenvolvida nesse motor?

\begin{oneparchoices}
\choice 3.3e+04 W\choice 2.8e+04 W\choice 2.9e+04 W\choice 247.072 W\choice 1.9e+04 W\choice 2.5e+04 W\choice   0.017 W\choice 485.329 W\choice 241.329 W\choice  59.670 W
\end{oneparchoices}\question A figura abaixo mostra a trajetória de uma partícula eletricamente carregada. $\vec{{v}}$ representa a velocidade atravessando um campo magnético $\vec{{B}}$. Determine a sua trajetória devido a ação da força magnética atuando sobre ela.
        
        \begin{center}
            \begin{minipage}[c]{0.5\linewidth}
                \begin{tikzpicture}[scale=0.5,transform shape, font=\Large]

                    \tkzInit[xmin=-3,xmax=3,ymin=-3,ymax=3]
                %	\tkzGrid[color=gray!20]
                    \tkzClip[space=1.0]

                    \tkzDefPoints{0/0/O,4/0/P}

                    \foreach \x in {-2.5,-1.5,...,2.5}{
                        \foreach \y in {-2.5,-1.5,...,2.5}{
                        \tkzDefPoint(\x,\y){B}
                        \tkzText(B){x}
                }
                }

                \draw[->, line width=1pt, color=red] (0,0) --++ (0,1.5) node [above] {$\vec{v}$};

                    \node[circle, radius=0.25, ball color=gray!50] (n1) at (0,0) {+};

                    \tkzText[above right=0.25cm](B){$\vec{B}$}

                \end{tikzpicture}
            \end{minipage}
        \end{center}

        

\begin{choices}
\choice Paralelo ao papel e circular no sentido anti-horário. 
\choice Paralelo ao papel e na vertical. 
\choice Paralelo ao papel e da esquerda para a direita. 
\choice Paralelo ao papel e circular no sentido horário. 
\choice Paralelo ao papel e da direita para a esquerda. 
\end{choices}
\question Uma partícula de carga 6.44e-06 C é lançada em um campo magnético uniforme de    0.63 T , com uma velocidade de 971.35 m/s. Calcule o valor da força magnética atuando na carga se o ângulo entre a velocidade e o campo magnético for   74.82 graus.

\begin{oneparchoices}
\choice   0.295 N\choice 3.8e-03 N\choice 9.6e-04 N\choice 8.1e-04 N\choice 1.3e-04 N\choice -2.2e-03 N\choice 1.0e-03 N\choice 6.8e-05 N\choice 1.9e-04 N\choice 2.3e-04 N
\end{oneparchoices}\end{multicols*}
\end{questions}
\newpage
\end{document}