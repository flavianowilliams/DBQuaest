
\documentclass[12pt, addpoints]{exam}
\usepackage[utf8]{inputenc}
\usepackage[portuguese]{babel}
\usepackage{multicol}
\usepackage{graphicx}
\usepackage{amsmath}
\usepackage{xcolor}
\usepackage{tikz,pgfplots,tikz-3dplot,bm}
\usepackage{circuitikz}
\usepackage{tkz-base}
\usepackage{tkz-fct}
\usepackage{tkz-euclide}
\usepackage[a4paper, portrait, margin=2cm]{geometry}

\usetikzlibrary{arrows,3d,calc,automata,positioning,shadows,math,fit,shapes}
\usetikzlibrary{patterns,hobby,optics,calc}
\tikzset{>=stealth, thick, global scale/.style={scale=#1,every node/.style={scale=#1}}}
\setlength{\columnsep}{1cm}
\renewcommand{\choiceshook}{\setlength{\leftmargin}{0pt}}

\begin{document}

    \begin{minipage}[b]{0.75\linewidth}
        \begin{flushleft}
            {\bf \large Prova bimestral}
        \end{flushleft}
        \begin{flushleft}
            {\bf \large LQ2N (2B), 31 de outubro de 2022}
        \end{flushleft}
    \end{minipage}
    \begin{minipage}[b]{0.20\linewidth}
        \begin{flushright}
            {\bf \large Code: 0}
        \end{flushright}
    \end{minipage}
    \vspace{0.5cm} \hrule \vspace{0.5cm}
    \begin{minipage}{0.75\linewidth}
        \begin{flushleft}
            Student: Alessandra C. A. dos Santos
        \end{flushleft}
    \end{minipage}
    \begin{minipage}{0.20\linewidth}
        \begin{flushright}
            Class: LQ2N
        \end{flushright}
    \end{minipage}
    \vspace{0.5cm} \hrule \vspace{0.5cm}
    \begin{questions}
\begin{multicols*}{2}
\question Considere a figura abaixo onde as linhas trajeçadas representam superfícies equipotenciais Se colocarmos um elétron próximo a carga Q, quais trechos possíveis o elétron poderá se deslocar?
        
        \begin{center}
            \begin{minipage}[c]{0.5\linewidth}
                \begin{tikzpicture}[scale=0.5,transform shape, font=\Large]

                \tkzInit[xmin=-4,xmax=4,ymin=-4,ymax=4]
                \tkzClip[space=0.5]

                \tkzDefPoints{0/0/O,4/0/P}

                \foreach \x in {0.5,1.25,2.25,3,4}{
                    \tkzDrawCircle[R,dashed,color=gray!50](O,\x)
                }

                \foreach \y in {0,1,...,11}{
                    \tkzDefPointsBy[rotation= center O angle 30*\y](O,P){P1,P2}
                \draw[->, line width=1.0pt] (O) -- (P2);}

                \tkzDefPoints{3/0/a,4/0/b,0/4/c,0/3/d}

                \tkzDrawPoints[color=red,fill=red,size=0.3cm](a,b,c,d)

                \tkzDrawPoints(O)
                \tkzLabelPoints[above right,font=\Large](a,b,c,d)

                \node[circle, radius=0.25, ball color=gray!50] (n1) at (0,0) {Q};

                \end{tikzpicture}
            \end{minipage}
        \end{center}
        
        

\begin{choices}
\choice $a\rightarrow b$ ou $d\rightarrow c$ 
\choice $c\rightarrow b$ ou $d\rightarrow a$ 
\choice $b\rightarrow c$ ou $a\rightarrow d$ 
\choice $b\rightarrow a$ ou $c\rightarrow d$ 
\choice $b\rightarrow a\rightarrow d\rightarrow c$ ou $c\rightarrow d\rightarrow a\rightarrow b$ 
\end{choices}
\question Uma corrente elétrica de    8.72 A percorre um fio de cobre. Sabendo-se que a carga de um elétron é igual a $1,6\times 10^{-19}\;C$, qual é o número de elétrons que atravessa, por minuto, a seção reta desse fio?

\begin{oneparchoices}
\choice 4.2e+19 \choice 5.5e+19 \choice 1.4e-18 \choice 4.5e+19 \choice 7.3e+19 \choice 6.9e+19 \choice 3.3e+21 \choice 8.4e-17 \choice 6.6e+19 \choice 4.4e+19 
\end{oneparchoices}\question Uma diferença de potencial de 120 V é aplicada a uma bomba d’água. Sabe-se que em funcionamento, o motor da bomba é percorrido por uma corrente de    1.28 A. Qual é a potência desenvolvida nesse motor?

\begin{oneparchoices}
\choice   0.011 W\choice 9.9e+03 W\choice 197.285 W\choice 153.864 W\choice 2.7e+04 W\choice 3.5e+04 W\choice  93.589 W\choice 1.3e+04 W\choice 1.1e+03 W\choice 2.6e+04 W
\end{oneparchoices}\question A figura abaixo mostra a trajetória de uma partícula eletricamente carregada. $\vec{{v}}$ representa a velocidade atravessando um campo magnético $\vec{{B}}$. Determine a sua trajetória devido a ação da força magnética atuando sobre ela.
        
        \begin{center}
            \begin{minipage}[c]{0.5\linewidth}
                \begin{tikzpicture}[scale=0.5,transform shape, font=\Large]

                    \tkzInit[xmin=-3,xmax=3,ymin=-3,ymax=3]
                %	\tkzGrid[color=gray!20]
                    \tkzClip[space=1.0]

                    \tkzDefPoints{0/0/O,4/0/P}

                    \foreach \x in {-2.5,-1.5,...,2.5}{
                        \foreach \y in {-2.5,-1.5,...,2.5}{
                        \tkzDefPoint(\x,\y){B}
                        \tkzText(B){x}
                }
                }

                \draw[->, line width=1pt, color=red] (0,0) --++ (0,1.5) node [above] {$\vec{v}$};

                    \node[circle, radius=0.25, ball color=gray!50] (n1) at (0,0) {+};

                    \tkzText[above right=0.25cm](B){$\vec{B}$}

                \end{tikzpicture}
            \end{minipage}
        \end{center}

        

\begin{choices}
\choice Paralelo ao papel e circular no sentido horário. 
\choice Paralelo ao papel e na vertical. 
\choice Paralelo ao papel e circular no sentido anti-horário. 
\choice Paralelo ao papel e da esquerda para a direita. 
\choice Paralelo ao papel e da direita para a esquerda. 
\end{choices}
\question Uma partícula de carga 6.59e-06 C é lançada em um campo magnético uniforme de    0.96 T , com uma velocidade de 398.41 m/s. Calcule o valor da força magnética atuando na carga se o ângulo entre a velocidade e o campo magnético for   42.05 graus.

\begin{oneparchoices}
\choice 3.8e-04 N\choice 3.3e-03 N\choice 1.7e-03 N\choice 1.9e-03 N\choice 4.2e-04 N\choice -2.4e-03 N\choice 9.7e-04 N\choice   0.106 N\choice 3.3e-04 N\choice 1.2e-03 N
\end{oneparchoices}\end{multicols*}
\end{questions}
\newpage

    \begin{minipage}[b]{0.75\linewidth}
        \begin{flushleft}
            {\bf \large Prova bimestral}
        \end{flushleft}
        \begin{flushleft}
            {\bf \large LQ2N (2B), 31 de outubro de 2022}
        \end{flushleft}
    \end{minipage}
    \begin{minipage}[b]{0.20\linewidth}
        \begin{flushright}
            {\bf \large Code: 1}
        \end{flushright}
    \end{minipage}
    \vspace{0.5cm} \hrule \vspace{0.5cm}
    \begin{minipage}{0.75\linewidth}
        \begin{flushleft}
            Student: Joivana F. R. Lau
        \end{flushleft}
    \end{minipage}
    \begin{minipage}{0.20\linewidth}
        \begin{flushright}
            Class: LQ2N
        \end{flushright}
    \end{minipage}
    \vspace{0.5cm} \hrule \vspace{0.5cm}
    \begin{questions}
\begin{multicols*}{2}
\question Considere a figura abaixo onde as linhas trajeçadas representam superfícies equipotenciais Se colocarmos um elétron próximo a carga Q, quais trechos possíveis o elétron poderá se deslocar?
        
        \begin{center}
            \begin{minipage}[c]{0.5\linewidth}
                \begin{tikzpicture}[scale=0.5,transform shape, font=\Large]

                \tkzInit[xmin=-4,xmax=4,ymin=-4,ymax=4]
                \tkzClip[space=0.5]

                \tkzDefPoints{0/0/O,4/0/P}

                \foreach \x in {0.5,1.25,2.25,3,4}{
                    \tkzDrawCircle[R,dashed,color=gray!50](O,\x)
                }

                \foreach \y in {0,1,...,11}{
                    \tkzDefPointsBy[rotation= center O angle 30*\y](O,P){P1,P2}
                \draw[->, line width=1.0pt] (O) -- (P2);}

                \tkzDefPoints{3/0/a,4/0/b,0/4/c,0/3/d}

                \tkzDrawPoints[color=red,fill=red,size=0.3cm](a,b,c,d)

                \tkzDrawPoints(O)
                \tkzLabelPoints[above right,font=\Large](a,b,c,d)

                \node[circle, radius=0.25, ball color=gray!50] (n1) at (0,0) {Q};

                \end{tikzpicture}
            \end{minipage}
        \end{center}
        
        

\begin{choices}
\choice $b\rightarrow a\rightarrow d\rightarrow c$ ou $c\rightarrow d\rightarrow a\rightarrow b$ 
\choice $c\rightarrow b$ ou $d\rightarrow a$ 
\choice $b\rightarrow c$ ou $a\rightarrow d$ 
\choice $a\rightarrow b$ ou $d\rightarrow c$ 
\choice $b\rightarrow a$ ou $c\rightarrow d$ 
\end{choices}
\question Uma corrente elétrica de    2.17 A percorre um fio de cobre. Sabendo-se que a carga de um elétron é igual a $1,6\times 10^{-19}\;C$, qual é o número de elétrons que atravessa, por minuto, a seção reta desse fio?

\begin{oneparchoices}
\choice 6.3e+19 \choice 2.7e+19 \choice 7.5e+19 \choice 1.4e+19 \choice 3.5e-19 \choice 2.1e-17 \choice 8.0e+19 \choice 8.1e+20 \choice 9.7e+19 \choice 8.2e+19 
\end{oneparchoices}\question Uma diferença de potencial de 120 V é aplicada a uma bomba d’água. Sabe-se que em funcionamento, o motor da bomba é percorrido por uma corrente de    3.23 A. Qual é a potência desenvolvida nesse motor?

\begin{oneparchoices}
\choice 387.185 W\choice 1.2e+03 W\choice 2.7e+04 W\choice 4.4e+03 W\choice  37.191 W\choice 2.0e+04 W\choice 1.4e+04 W\choice   0.027 W\choice 2.4e+04 W\choice 806.793 W
\end{oneparchoices}\question A figura abaixo mostra a trajetória de uma partícula eletricamente carregada. $\vec{{v}}$ representa a velocidade atravessando um campo magnético $\vec{{B}}$. Determine a sua trajetória devido a ação da força magnética atuando sobre ela.
        
        \begin{center}
            \begin{minipage}[c]{0.5\linewidth}
                \begin{tikzpicture}[scale=0.5,transform shape, font=\Large]

                    \tkzInit[xmin=-3,xmax=3,ymin=-3,ymax=3]
                %	\tkzGrid[color=gray!20]
                    \tkzClip[space=1.0]

                    \tkzDefPoints{0/0/O,4/0/P}

                    \foreach \x in {-2.5,-1.5,...,2.5}{
                        \foreach \y in {-2.5,-1.5,...,2.5}{
                        \tkzDefPoint(\x,\y){B}
                        \tkzText(B){x}
                }
                }

                \draw[->, line width=1pt, color=red] (0,0) --++ (0,1.5) node [above] {$\vec{v}$};

                    \node[circle, radius=0.25, ball color=gray!50] (n1) at (0,0) {+};

                    \tkzText[above right=0.25cm](B){$\vec{B}$}

                \end{tikzpicture}
            \end{minipage}
        \end{center}

        

\begin{choices}
\choice Paralelo ao papel e circular no sentido horário. 
\choice Paralelo ao papel e na vertical. 
\choice Paralelo ao papel e da direita para a esquerda. 
\choice Paralelo ao papel e circular no sentido anti-horário. 
\choice Paralelo ao papel e da esquerda para a direita. 
\end{choices}
\question Uma partícula de carga 9.22e-06 C é lançada em um campo magnético uniforme de    0.55 T , com uma velocidade de 858.59 m/s. Calcule o valor da força magnética atuando na carga se o ângulo entre a velocidade e o campo magnético for    6.31 graus.

\begin{oneparchoices}
\choice 3.4e-03 N\choice   0.027 N\choice 1.3e-03 N\choice 6.2e-03 N\choice 6.2e-04 N\choice 4.5e-04 N\choice 4.7e-04 N\choice 1.1e-04 N\choice 4.3e-03 N\choice 1.3e-03 N
\end{oneparchoices}\end{multicols*}
\end{questions}
\newpage

    \begin{minipage}[b]{0.75\linewidth}
        \begin{flushleft}
            {\bf \large Prova bimestral}
        \end{flushleft}
        \begin{flushleft}
            {\bf \large LQ2N (2B), 31 de outubro de 2022}
        \end{flushleft}
    \end{minipage}
    \begin{minipage}[b]{0.20\linewidth}
        \begin{flushright}
            {\bf \large Code: 2}
        \end{flushright}
    \end{minipage}
    \vspace{0.5cm} \hrule \vspace{0.5cm}
    \begin{minipage}{0.75\linewidth}
        \begin{flushleft}
            Student: Nicoly T. Trentin
        \end{flushleft}
    \end{minipage}
    \begin{minipage}{0.20\linewidth}
        \begin{flushright}
            Class: LQ2N
        \end{flushright}
    \end{minipage}
    \vspace{0.5cm} \hrule \vspace{0.5cm}
    \begin{questions}
\begin{multicols*}{2}
\question Considere a figura abaixo onde as linhas trajeçadas representam superfícies equipotenciais Se colocarmos um elétron próximo a carga Q, quais trechos possíveis o elétron poderá se deslocar?
        
        \begin{center}
            \begin{minipage}[c]{0.5\linewidth}
                \begin{tikzpicture}[scale=0.5,transform shape, font=\Large]

                \tkzInit[xmin=-4,xmax=4,ymin=-4,ymax=4]
                \tkzClip[space=0.5]

                \tkzDefPoints{0/0/O,4/0/P}

                \foreach \x in {0.5,1.25,2.25,3,4}{
                    \tkzDrawCircle[R,dashed,color=gray!50](O,\x)
                }

                \foreach \y in {0,1,...,11}{
                    \tkzDefPointsBy[rotation= center O angle 30*\y](O,P){P1,P2}
                \draw[->, line width=1.0pt] (O) -- (P2);}

                \tkzDefPoints{3/0/a,4/0/b,0/4/c,0/3/d}

                \tkzDrawPoints[color=red,fill=red,size=0.3cm](a,b,c,d)

                \tkzDrawPoints(O)
                \tkzLabelPoints[above right,font=\Large](a,b,c,d)

                \node[circle, radius=0.25, ball color=gray!50] (n1) at (0,0) {Q};

                \end{tikzpicture}
            \end{minipage}
        \end{center}
        
        

\begin{choices}
\choice $b\rightarrow c$ ou $a\rightarrow d$ 
\choice $c\rightarrow b$ ou $d\rightarrow a$ 
\choice $b\rightarrow a$ ou $c\rightarrow d$ 
\choice $b\rightarrow a\rightarrow d\rightarrow c$ ou $c\rightarrow d\rightarrow a\rightarrow b$ 
\choice $a\rightarrow b$ ou $d\rightarrow c$ 
\end{choices}
\question Uma corrente elétrica de    9.32 A percorre um fio de cobre. Sabendo-se que a carga de um elétron é igual a $1,6\times 10^{-19}\;C$, qual é o número de elétrons que atravessa, por minuto, a seção reta desse fio?

\begin{oneparchoices}
\choice 2.9e+19 \choice 5.8e+19 \choice 4.0e+19 \choice 9.5e+19 \choice 3.5e+21 \choice 7.4e+19 \choice 3.5e+19 \choice 1.5e-18 \choice 8.9e-17 \choice 9.6e+19 
\end{oneparchoices}\question Uma diferença de potencial de 120 V é aplicada a uma bomba d’água. Sabe-se que em funcionamento, o motor da bomba é percorrido por uma corrente de    2.35 A. Qual é a potência desenvolvida nesse motor?

\begin{oneparchoices}
\choice 4.9e+03 W\choice 8.2e+03 W\choice 664.893 W\choice 282.466 W\choice 1.3e+04 W\choice 1.9e+04 W\choice 2.3e+04 W\choice 1.2e+04 W\choice  50.980 W\choice   0.020 W
\end{oneparchoices}\question A figura abaixo mostra a trajetória de uma partícula eletricamente carregada. $\vec{{v}}$ representa a velocidade atravessando um campo magnético $\vec{{B}}$. Determine a sua trajetória devido a ação da força magnética atuando sobre ela.
        
        \begin{center}
            \begin{minipage}[c]{0.5\linewidth}
                \begin{tikzpicture}[scale=0.5,transform shape, font=\Large]

                    \tkzInit[xmin=-3,xmax=3,ymin=-3,ymax=3]
                %	\tkzGrid[color=gray!20]
                    \tkzClip[space=1.0]

                    \tkzDefPoints{0/0/O,4/0/P}

                    \foreach \x in {-2.5,-1.5,...,2.5}{
                        \foreach \y in {-2.5,-1.5,...,2.5}{
                        \tkzDefPoint(\x,\y){B}
                        \tkzText(B){x}
                }
                }

                \draw[->, line width=1pt, color=red] (0,0) --++ (0,1.5) node [above] {$\vec{v}$};

                    \node[circle, radius=0.25, ball color=gray!50] (n1) at (0,0) {+};

                    \tkzText[above right=0.25cm](B){$\vec{B}$}

                \end{tikzpicture}
            \end{minipage}
        \end{center}

        

\begin{choices}
\choice Paralelo ao papel e da direita para a esquerda. 
\choice Paralelo ao papel e da esquerda para a direita. 
\choice Paralelo ao papel e circular no sentido anti-horário. 
\choice Paralelo ao papel e circular no sentido horário. 
\choice Paralelo ao papel e na vertical. 
\end{choices}
\question Uma partícula de carga 5.22e-06 C é lançada em um campo magnético uniforme de    0.63 T , com uma velocidade de 822.68 m/s. Calcule o valor da força magnética atuando na carga se o ângulo entre a velocidade e o campo magnético for   44.49 graus.

\begin{oneparchoices}
\choice 1.8e-04 N\choice 2.1e-03 N\choice 1.7e-03 N\choice   0.120 N\choice 4.3e-04 N\choice 1.9e-03 N\choice 7.3e-04 N\choice 1.3e-03 N\choice 1.1e-03 N\choice 1.9e-03 N
\end{oneparchoices}\end{multicols*}
\end{questions}
\newpage

    \begin{minipage}[b]{0.75\linewidth}
        \begin{flushleft}
            {\bf \large Prova bimestral}
        \end{flushleft}
        \begin{flushleft}
            {\bf \large LQ2N (2B), 31 de outubro de 2022}
        \end{flushleft}
    \end{minipage}
    \begin{minipage}[b]{0.20\linewidth}
        \begin{flushright}
            {\bf \large Code: 3}
        \end{flushright}
    \end{minipage}
    \vspace{0.5cm} \hrule \vspace{0.5cm}
    \begin{minipage}{0.75\linewidth}
        \begin{flushleft}
            Student: Bruno A. Vienc
        \end{flushleft}
    \end{minipage}
    \begin{minipage}{0.20\linewidth}
        \begin{flushright}
            Class: LQ2N
        \end{flushright}
    \end{minipage}
    \vspace{0.5cm} \hrule \vspace{0.5cm}
    \begin{questions}
\begin{multicols*}{2}
\question Considere a figura abaixo onde as linhas trajeçadas representam superfícies equipotenciais Se colocarmos um elétron próximo a carga Q, quais trechos possíveis o elétron poderá se deslocar?
        
        \begin{center}
            \begin{minipage}[c]{0.5\linewidth}
                \begin{tikzpicture}[scale=0.5,transform shape, font=\Large]

                \tkzInit[xmin=-4,xmax=4,ymin=-4,ymax=4]
                \tkzClip[space=0.5]

                \tkzDefPoints{0/0/O,4/0/P}

                \foreach \x in {0.5,1.25,2.25,3,4}{
                    \tkzDrawCircle[R,dashed,color=gray!50](O,\x)
                }

                \foreach \y in {0,1,...,11}{
                    \tkzDefPointsBy[rotation= center O angle 30*\y](O,P){P1,P2}
                \draw[->, line width=1.0pt] (O) -- (P2);}

                \tkzDefPoints{3/0/a,4/0/b,0/4/c,0/3/d}

                \tkzDrawPoints[color=red,fill=red,size=0.3cm](a,b,c,d)

                \tkzDrawPoints(O)
                \tkzLabelPoints[above right,font=\Large](a,b,c,d)

                \node[circle, radius=0.25, ball color=gray!50] (n1) at (0,0) {Q};

                \end{tikzpicture}
            \end{minipage}
        \end{center}
        
        

\begin{choices}
\choice $c\rightarrow b$ ou $d\rightarrow a$ 
\choice $b\rightarrow c$ ou $a\rightarrow d$ 
\choice $b\rightarrow a$ ou $c\rightarrow d$ 
\choice $a\rightarrow b$ ou $d\rightarrow c$ 
\choice $b\rightarrow a\rightarrow d\rightarrow c$ ou $c\rightarrow d\rightarrow a\rightarrow b$ 
\end{choices}
\question Uma corrente elétrica de    6.81 A percorre um fio de cobre. Sabendo-se que a carga de um elétron é igual a $1,6\times 10^{-19}\;C$, qual é o número de elétrons que atravessa, por minuto, a seção reta desse fio?

\begin{oneparchoices}
\choice 4.7e+19 \choice 4.3e+19 \choice 8.2e+19 \choice 2.6e+21 \choice 6.6e+19 \choice 1.1e+19 \choice 1.1e-18 \choice 6.0e+19 \choice 6.5e-17 \choice 9.1e+19 
\end{oneparchoices}\question Uma diferença de potencial de 120 V é aplicada a uma bomba d’água. Sabe-se que em funcionamento, o motor da bomba é percorrido por uma corrente de    3.64 A. Qual é a potência desenvolvida nesse motor?

\begin{oneparchoices}
\choice  33.007 W\choice 1.8e+04 W\choice 1.9e+04 W\choice 2.6e+04 W\choice 2.1e+04 W\choice 1.6e+03 W\choice 3.4e+04 W\choice 436.274 W\choice   0.030 W\choice 5.3e+03 W
\end{oneparchoices}\question A figura abaixo mostra a trajetória de uma partícula eletricamente carregada. $\vec{{v}}$ representa a velocidade atravessando um campo magnético $\vec{{B}}$. Determine a sua trajetória devido a ação da força magnética atuando sobre ela.
        
        \begin{center}
            \begin{minipage}[c]{0.5\linewidth}
                \begin{tikzpicture}[scale=0.5,transform shape, font=\Large]

                    \tkzInit[xmin=-3,xmax=3,ymin=-3,ymax=3]
                %	\tkzGrid[color=gray!20]
                    \tkzClip[space=1.0]

                    \tkzDefPoints{0/0/O,4/0/P}

                    \foreach \x in {-2.5,-1.5,...,2.5}{
                        \foreach \y in {-2.5,-1.5,...,2.5}{
                        \tkzDefPoint(\x,\y){B}
                        \tkzText(B){x}
                }
                }

                \draw[->, line width=1pt, color=red] (0,0) --++ (0,1.5) node [above] {$\vec{v}$};

                    \node[circle, radius=0.25, ball color=gray!50] (n1) at (0,0) {+};

                    \tkzText[above right=0.25cm](B){$\vec{B}$}

                \end{tikzpicture}
            \end{minipage}
        \end{center}

        

\begin{choices}
\choice Paralelo ao papel e da direita para a esquerda. 
\choice Paralelo ao papel e na vertical. 
\choice Paralelo ao papel e da esquerda para a direita. 
\choice Paralelo ao papel e circular no sentido anti-horário. 
\choice Paralelo ao papel e circular no sentido horário. 
\end{choices}
\question Uma partícula de carga 9.78e-06 C é lançada em um campo magnético uniforme de    0.57 T , com uma velocidade de 398.72 m/s. Calcule o valor da força magnética atuando na carga se o ângulo entre a velocidade e o campo magnético for   11.27 graus.

\begin{oneparchoices}
\choice   0.025 N\choice 1.6e-04 N\choice 1.2e-03 N\choice 2.2e-03 N\choice 3.5e-04 N\choice -2.1e-03 N\choice 4.3e-04 N\choice 7.6e-04 N\choice 8.8e-04 N\choice 1.0e-04 N
\end{oneparchoices}\end{multicols*}
\end{questions}
\newpage

    \begin{minipage}[b]{0.75\linewidth}
        \begin{flushleft}
            {\bf \large Prova bimestral}
        \end{flushleft}
        \begin{flushleft}
            {\bf \large LQ2N (2B), 31 de outubro de 2022}
        \end{flushleft}
    \end{minipage}
    \begin{minipage}[b]{0.20\linewidth}
        \begin{flushright}
            {\bf \large Code: 4}
        \end{flushright}
    \end{minipage}
    \vspace{0.5cm} \hrule \vspace{0.5cm}
    \begin{minipage}{0.75\linewidth}
        \begin{flushleft}
            Student: Margarete do R. de A. Soares
        \end{flushleft}
    \end{minipage}
    \begin{minipage}{0.20\linewidth}
        \begin{flushright}
            Class: LQ2N
        \end{flushright}
    \end{minipage}
    \vspace{0.5cm} \hrule \vspace{0.5cm}
    \begin{questions}
\begin{multicols*}{2}
\question Considere a figura abaixo onde as linhas trajeçadas representam superfícies equipotenciais Se colocarmos um elétron próximo a carga Q, quais trechos possíveis o elétron poderá se deslocar?
        
        \begin{center}
            \begin{minipage}[c]{0.5\linewidth}
                \begin{tikzpicture}[scale=0.5,transform shape, font=\Large]

                \tkzInit[xmin=-4,xmax=4,ymin=-4,ymax=4]
                \tkzClip[space=0.5]

                \tkzDefPoints{0/0/O,4/0/P}

                \foreach \x in {0.5,1.25,2.25,3,4}{
                    \tkzDrawCircle[R,dashed,color=gray!50](O,\x)
                }

                \foreach \y in {0,1,...,11}{
                    \tkzDefPointsBy[rotation= center O angle 30*\y](O,P){P1,P2}
                \draw[->, line width=1.0pt] (O) -- (P2);}

                \tkzDefPoints{3/0/a,4/0/b,0/4/c,0/3/d}

                \tkzDrawPoints[color=red,fill=red,size=0.3cm](a,b,c,d)

                \tkzDrawPoints(O)
                \tkzLabelPoints[above right,font=\Large](a,b,c,d)

                \node[circle, radius=0.25, ball color=gray!50] (n1) at (0,0) {Q};

                \end{tikzpicture}
            \end{minipage}
        \end{center}
        
        

\begin{choices}
\choice $b\rightarrow c$ ou $a\rightarrow d$ 
\choice $b\rightarrow a$ ou $c\rightarrow d$ 
\choice $a\rightarrow b$ ou $d\rightarrow c$ 
\choice $c\rightarrow b$ ou $d\rightarrow a$ 
\choice $b\rightarrow a\rightarrow d\rightarrow c$ ou $c\rightarrow d\rightarrow a\rightarrow b$ 
\end{choices}
\question Uma corrente elétrica de    5.87 A percorre um fio de cobre. Sabendo-se que a carga de um elétron é igual a $1,6\times 10^{-19}\;C$, qual é o número de elétrons que atravessa, por minuto, a seção reta desse fio?

\begin{oneparchoices}
\choice 2.2e+21 \choice 1.9e+19 \choice 7.4e+19 \choice 9.4e-19 \choice 6.7e+19 \choice 5.6e-17 \choice 4.1e+19 \choice 5.5e+19 \choice 7.3e+19 \choice 3.7e+19 
\end{oneparchoices}\question Uma diferença de potencial de 120 V é aplicada a uma bomba d’água. Sabe-se que em funcionamento, o motor da bomba é percorrido por uma corrente de    4.16 A. Qual é a potência desenvolvida nesse motor?

\begin{oneparchoices}
\choice  15.805 W\choice 1.4e+04 W\choice 2.6e+04 W\choice 2.1e+03 W\choice 3.2e+04 W\choice  28.845 W\choice   0.035 W\choice 185.637 W\choice 499.218 W\choice 2.8e+04 W
\end{oneparchoices}\question A figura abaixo mostra a trajetória de uma partícula eletricamente carregada. $\vec{{v}}$ representa a velocidade atravessando um campo magnético $\vec{{B}}$. Determine a sua trajetória devido a ação da força magnética atuando sobre ela.
        
        \begin{center}
            \begin{minipage}[c]{0.5\linewidth}
                \begin{tikzpicture}[scale=0.5,transform shape, font=\Large]

                    \tkzInit[xmin=-3,xmax=3,ymin=-3,ymax=3]
                %	\tkzGrid[color=gray!20]
                    \tkzClip[space=1.0]

                    \tkzDefPoints{0/0/O,4/0/P}

                    \foreach \x in {-2.5,-1.5,...,2.5}{
                        \foreach \y in {-2.5,-1.5,...,2.5}{
                        \tkzDefPoint(\x,\y){B}
                        \tkzText(B){x}
                }
                }

                \draw[->, line width=1pt, color=red] (0,0) --++ (0,1.5) node [above] {$\vec{v}$};

                    \node[circle, radius=0.25, ball color=gray!50] (n1) at (0,0) {+};

                    \tkzText[above right=0.25cm](B){$\vec{B}$}

                \end{tikzpicture}
            \end{minipage}
        \end{center}

        

\begin{choices}
\choice Paralelo ao papel e circular no sentido horário. 
\choice Paralelo ao papel e circular no sentido anti-horário. 
\choice Paralelo ao papel e da direita para a esquerda. 
\choice Paralelo ao papel e na vertical. 
\choice Paralelo ao papel e da esquerda para a direita. 
\end{choices}
\question Uma partícula de carga 8.73e-06 C é lançada em um campo magnético uniforme de    0.59 T , com uma velocidade de 609.07 m/s. Calcule o valor da força magnética atuando na carga se o ângulo entre a velocidade e o campo magnético for   16.02 graus.

\begin{oneparchoices}
\choice -9.6e-04 N\choice 1.1e-03 N\choice 5.7e-04 N\choice 1.6e-04 N\choice   0.050 N\choice 8.6e-04 N\choice 3.0e-03 N\choice 3.6e-04 N\choice 1.3e-04 N\choice 6.5e-05 N
\end{oneparchoices}\end{multicols*}
\end{questions}
\newpage

    \begin{minipage}[b]{0.75\linewidth}
        \begin{flushleft}
            {\bf \large Prova bimestral}
        \end{flushleft}
        \begin{flushleft}
            {\bf \large LQ2N (2B), 31 de outubro de 2022}
        \end{flushleft}
    \end{minipage}
    \begin{minipage}[b]{0.20\linewidth}
        \begin{flushright}
            {\bf \large Code: 5}
        \end{flushright}
    \end{minipage}
    \vspace{0.5cm} \hrule \vspace{0.5cm}
    \begin{minipage}{0.75\linewidth}
        \begin{flushleft}
            Student: Marcelo A. de O. Teixeira
        \end{flushleft}
    \end{minipage}
    \begin{minipage}{0.20\linewidth}
        \begin{flushright}
            Class: LQ2N
        \end{flushright}
    \end{minipage}
    \vspace{0.5cm} \hrule \vspace{0.5cm}
    \begin{questions}
\begin{multicols*}{2}
\question Considere a figura abaixo onde as linhas trajeçadas representam superfícies equipotenciais Se colocarmos um elétron próximo a carga Q, quais trechos possíveis o elétron poderá se deslocar?
        
        \begin{center}
            \begin{minipage}[c]{0.5\linewidth}
                \begin{tikzpicture}[scale=0.5,transform shape, font=\Large]

                \tkzInit[xmin=-4,xmax=4,ymin=-4,ymax=4]
                \tkzClip[space=0.5]

                \tkzDefPoints{0/0/O,4/0/P}

                \foreach \x in {0.5,1.25,2.25,3,4}{
                    \tkzDrawCircle[R,dashed,color=gray!50](O,\x)
                }

                \foreach \y in {0,1,...,11}{
                    \tkzDefPointsBy[rotation= center O angle 30*\y](O,P){P1,P2}
                \draw[->, line width=1.0pt] (O) -- (P2);}

                \tkzDefPoints{3/0/a,4/0/b,0/4/c,0/3/d}

                \tkzDrawPoints[color=red,fill=red,size=0.3cm](a,b,c,d)

                \tkzDrawPoints(O)
                \tkzLabelPoints[above right,font=\Large](a,b,c,d)

                \node[circle, radius=0.25, ball color=gray!50] (n1) at (0,0) {Q};

                \end{tikzpicture}
            \end{minipage}
        \end{center}
        
        

\begin{choices}
\choice $b\rightarrow c$ ou $a\rightarrow d$ 
\choice $a\rightarrow b$ ou $d\rightarrow c$ 
\choice $c\rightarrow b$ ou $d\rightarrow a$ 
\choice $b\rightarrow a$ ou $c\rightarrow d$ 
\choice $b\rightarrow a\rightarrow d\rightarrow c$ ou $c\rightarrow d\rightarrow a\rightarrow b$ 
\end{choices}
\question Uma corrente elétrica de    3.81 A percorre um fio de cobre. Sabendo-se que a carga de um elétron é igual a $1,6\times 10^{-19}\;C$, qual é o número de elétrons que atravessa, por minuto, a seção reta desse fio?

\begin{oneparchoices}
\choice 4.9e+19 \choice 5.8e+19 \choice 8.8e+19 \choice 2.4e+19 \choice 7.9e+19 \choice 6.1e-19 \choice 1.4e+21 \choice 3.7e-17 \choice 3.4e+19 \choice 3.0e+19 
\end{oneparchoices}\question Uma diferença de potencial de 120 V é aplicada a uma bomba d’água. Sabe-se que em funcionamento, o motor da bomba é percorrido por uma corrente de    3.75 A. Qual é a potência desenvolvida nesse motor?

\begin{oneparchoices}
\choice 6.0e+03 W\choice 3.3e+04 W\choice 450.265 W\choice 2.1e+04 W\choice 2.0e+03 W\choice 2.7e+04 W\choice  31.981 W\choice 1.5e+04 W\choice   0.031 W\choice 1.7e+03 W
\end{oneparchoices}\question A figura abaixo mostra a trajetória de uma partícula eletricamente carregada. $\vec{{v}}$ representa a velocidade atravessando um campo magnético $\vec{{B}}$. Determine a sua trajetória devido a ação da força magnética atuando sobre ela.
        
        \begin{center}
            \begin{minipage}[c]{0.5\linewidth}
                \begin{tikzpicture}[scale=0.5,transform shape, font=\Large]

                    \tkzInit[xmin=-3,xmax=3,ymin=-3,ymax=3]
                %	\tkzGrid[color=gray!20]
                    \tkzClip[space=1.0]

                    \tkzDefPoints{0/0/O,4/0/P}

                    \foreach \x in {-2.5,-1.5,...,2.5}{
                        \foreach \y in {-2.5,-1.5,...,2.5}{
                        \tkzDefPoint(\x,\y){B}
                        \tkzText(B){x}
                }
                }

                \draw[->, line width=1pt, color=red] (0,0) --++ (0,1.5) node [above] {$\vec{v}$};

                    \node[circle, radius=0.25, ball color=gray!50] (n1) at (0,0) {+};

                    \tkzText[above right=0.25cm](B){$\vec{B}$}

                \end{tikzpicture}
            \end{minipage}
        \end{center}

        

\begin{choices}
\choice Paralelo ao papel e circular no sentido horário. 
\choice Paralelo ao papel e circular no sentido anti-horário. 
\choice Paralelo ao papel e na vertical. 
\choice Paralelo ao papel e da direita para a esquerda. 
\choice Paralelo ao papel e da esquerda para a direita. 
\end{choices}
\question Uma partícula de carga 8.94e-06 C é lançada em um campo magnético uniforme de    0.59 T , com uma velocidade de 540.24 m/s. Calcule o valor da força magnética atuando na carga se o ângulo entre a velocidade e o campo magnético for   33.24 graus.

\begin{oneparchoices}
\choice 2.4e-03 N\choice 2.7e-03 N\choice 3.0e-04 N\choice 7.1e-05 N\choice 1.3e-03 N\choice 2.6e-04 N\choice   0.094 N\choice 6.2e-04 N\choice 5.6e-04 N\choice 1.6e-03 N
\end{oneparchoices}\end{multicols*}
\end{questions}
\newpage

    \begin{minipage}[b]{0.75\linewidth}
        \begin{flushleft}
            {\bf \large Prova bimestral}
        \end{flushleft}
        \begin{flushleft}
            {\bf \large LQ2N (2B), 31 de outubro de 2022}
        \end{flushleft}
    \end{minipage}
    \begin{minipage}[b]{0.20\linewidth}
        \begin{flushright}
            {\bf \large Code: 6}
        \end{flushright}
    \end{minipage}
    \vspace{0.5cm} \hrule \vspace{0.5cm}
    \begin{minipage}{0.75\linewidth}
        \begin{flushleft}
            Student: Gabrieli dos Santos
        \end{flushleft}
    \end{minipage}
    \begin{minipage}{0.20\linewidth}
        \begin{flushright}
            Class: LQ2N
        \end{flushright}
    \end{minipage}
    \vspace{0.5cm} \hrule \vspace{0.5cm}
    \begin{questions}
\begin{multicols*}{2}
\question Considere a figura abaixo onde as linhas trajeçadas representam superfícies equipotenciais Se colocarmos um elétron próximo a carga Q, quais trechos possíveis o elétron poderá se deslocar?
        
        \begin{center}
            \begin{minipage}[c]{0.5\linewidth}
                \begin{tikzpicture}[scale=0.5,transform shape, font=\Large]

                \tkzInit[xmin=-4,xmax=4,ymin=-4,ymax=4]
                \tkzClip[space=0.5]

                \tkzDefPoints{0/0/O,4/0/P}

                \foreach \x in {0.5,1.25,2.25,3,4}{
                    \tkzDrawCircle[R,dashed,color=gray!50](O,\x)
                }

                \foreach \y in {0,1,...,11}{
                    \tkzDefPointsBy[rotation= center O angle 30*\y](O,P){P1,P2}
                \draw[->, line width=1.0pt] (O) -- (P2);}

                \tkzDefPoints{3/0/a,4/0/b,0/4/c,0/3/d}

                \tkzDrawPoints[color=red,fill=red,size=0.3cm](a,b,c,d)

                \tkzDrawPoints(O)
                \tkzLabelPoints[above right,font=\Large](a,b,c,d)

                \node[circle, radius=0.25, ball color=gray!50] (n1) at (0,0) {Q};

                \end{tikzpicture}
            \end{minipage}
        \end{center}
        
        

\begin{choices}
\choice $b\rightarrow a\rightarrow d\rightarrow c$ ou $c\rightarrow d\rightarrow a\rightarrow b$ 
\choice $b\rightarrow a$ ou $c\rightarrow d$ 
\choice $c\rightarrow b$ ou $d\rightarrow a$ 
\choice $b\rightarrow c$ ou $a\rightarrow d$ 
\choice $a\rightarrow b$ ou $d\rightarrow c$ 
\end{choices}
\question Uma corrente elétrica de    1.67 A percorre um fio de cobre. Sabendo-se que a carga de um elétron é igual a $1,6\times 10^{-19}\;C$, qual é o número de elétrons que atravessa, por minuto, a seção reta desse fio?

\begin{oneparchoices}
\choice 1.6e-17 \choice 4.8e+19 \choice 8.2e+19 \choice 1.0e+19 \choice 2.7e-19 \choice 6.2e+20 \choice 3.2e+19 \choice 9.8e+19 \choice 7.7e+19 \choice 6.5e+19 
\end{oneparchoices}\question Uma diferença de potencial de 120 V é aplicada a uma bomba d’água. Sabe-se que em funcionamento, o motor da bomba é percorrido por uma corrente de    2.20 A. Qual é a potência desenvolvida nesse motor?

\begin{oneparchoices}
\choice 580.254 W\choice 3.3e+04 W\choice 1.3e+04 W\choice 2.0e+04 W\choice 5.2e+03 W\choice   0.018 W\choice 1.6e+04 W\choice 7.3e+03 W\choice 263.876 W\choice  54.571 W
\end{oneparchoices}\question A figura abaixo mostra a trajetória de uma partícula eletricamente carregada. $\vec{{v}}$ representa a velocidade atravessando um campo magnético $\vec{{B}}$. Determine a sua trajetória devido a ação da força magnética atuando sobre ela.
        
        \begin{center}
            \begin{minipage}[c]{0.5\linewidth}
                \begin{tikzpicture}[scale=0.5,transform shape, font=\Large]

                    \tkzInit[xmin=-3,xmax=3,ymin=-3,ymax=3]
                %	\tkzGrid[color=gray!20]
                    \tkzClip[space=1.0]

                    \tkzDefPoints{0/0/O,4/0/P}

                    \foreach \x in {-2.5,-1.5,...,2.5}{
                        \foreach \y in {-2.5,-1.5,...,2.5}{
                        \tkzDefPoint(\x,\y){B}
                        \tkzText(B){x}
                }
                }

                \draw[->, line width=1pt, color=red] (0,0) --++ (0,1.5) node [above] {$\vec{v}$};

                    \node[circle, radius=0.25, ball color=gray!50] (n1) at (0,0) {+};

                    \tkzText[above right=0.25cm](B){$\vec{B}$}

                \end{tikzpicture}
            \end{minipage}
        \end{center}

        

\begin{choices}
\choice Paralelo ao papel e circular no sentido horário. 
\choice Paralelo ao papel e na vertical. 
\choice Paralelo ao papel e da esquerda para a direita. 
\choice Paralelo ao papel e circular no sentido anti-horário. 
\choice Paralelo ao papel e da direita para a esquerda. 
\end{choices}
\question Uma partícula de carga 8.69e-06 C é lançada em um campo magnético uniforme de    0.49 T , com uma velocidade de 744.18 m/s. Calcule o valor da força magnética atuando na carga se o ângulo entre a velocidade e o campo magnético for   20.58 graus.

\begin{oneparchoices}
\choice 8.7e-04 N\choice 1.1e-03 N\choice 8.9e-05 N\choice 3.0e-03 N\choice 3.2e-03 N\choice 2.1e-05 N\choice 1.6e-03 N\choice   0.066 N\choice 4.1e-04 N\choice 5.1e-03 N
\end{oneparchoices}\end{multicols*}
\end{questions}
\newpage
\end{document}