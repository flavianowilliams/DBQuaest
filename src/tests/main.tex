
\documentclass[12pt, addpoints]{exam}
\usepackage[utf8]{inputenc}
\usepackage[portuguese]{babel}
\usepackage{multicol}
\usepackage{graphicx}
\usepackage{amsmath}
\usepackage{xcolor}
\usepackage{tikz,pgfplots,tikz-3dplot,bm}
\usepackage{circuitikz}
\usepackage{tkz-base}
\usepackage{tkz-fct}
\usepackage{tkz-euclide}
\usepackage[a4paper, portrait, margin=2cm]{geometry}

\usetikzlibrary{arrows,3d,calc,automata,positioning,shadows,math,fit,shapes}
\usetikzlibrary{patterns,hobby,optics,calc}
\tikzset{>=stealth, thick, global scale/.style={scale=#1,every node/.style={scale=#1}}}
\setlength{\columnsep}{1cm}
\renewcommand{\choiceshook}{\setlength{\leftmargin}{0pt}}

\begin{document}

    \begin{minipage}[b]{0.75\linewidth}
        \begin{flushleft}
            {\bf \large Prova bimestral}
        \end{flushleft}
        \begin{flushleft}
            {\bf \large LQ2N (2B), 31 de outubro de 2022}
        \end{flushleft}
    \end{minipage}
    \begin{minipage}[b]{0.20\linewidth}
        \begin{flushright}
            {\bf \large Code: 0}
        \end{flushright}
    \end{minipage}
    \vspace{0.5cm} \hrule \vspace{0.5cm}
    \begin{minipage}{0.50\linewidth}
        \begin{flushleft}
            Student: Flaviano W. Fernandes
        \end{flushleft}
    \end{minipage}
    \begin{minipage}{0.20\linewidth}
        \begin{center}
            Date: 2022-11-14
        \end{center}
    \end{minipage}
    \begin{minipage}{0.20\linewidth}
        \begin{flushright}
            Class: LQ2N
        \end{flushright}
    \end{minipage}
    \vspace{0.5cm} \hrule \vspace{0.5cm}
    \begin{questions}
\begin{multicols*}{2}
\question Considere a figura abaixo onde as linhas trajeçadas representam superfícies equipotenciais Se colocarmos um elétron próximo a carga Q, quais trechos possíveis o elétron poderá se deslocar?
        
        \begin{center}
            \begin{minipage}[c]{0.5\linewidth}
                \begin{tikzpicture}[scale=0.5,transform shape, font=\Large]

                \tkzInit[xmin=-4,xmax=4,ymin=-4,ymax=4]
                \tkzClip[space=0.5]

                \tkzDefPoints{0/0/O,4/0/P}

                \foreach \x in {0.5,1.25,2.25,3,4}{
                    \tkzDrawCircle[R,dashed,color=gray!50](O,\x)
                }

                \foreach \y in {0,1,...,11}{
                    \tkzDefPointsBy[rotation= center O angle 30*\y](O,P){P1,P2}
                \draw[->, line width=1.0pt] (O) -- (P2);}

                \tkzDefPoints{3/0/a,4/0/b,0/4/c,0/3/d}

                \tkzDrawPoints[color=red,fill=red,size=0.3cm](a,b,c,d)

                \tkzDrawPoints(O)
                \tkzLabelPoints[above right,font=\Large](a,b,c,d)

                \node[circle, radius=0.25, ball color=gray!50] (n1) at (0,0) {Q};

                \end{tikzpicture}
            \end{minipage}
        \end{center}
        
        

\begin{choices}
\choice $b\rightarrow c$ ou $a\rightarrow d$ 
\choice $a\rightarrow b$ ou $d\rightarrow c$ 
\choice $b\rightarrow a\rightarrow d\rightarrow c$ ou $c\rightarrow d\rightarrow a\rightarrow b$ 
\choice $c\rightarrow b$ ou $d\rightarrow a$ 
\choice $b\rightarrow a$ ou $c\rightarrow d$ 
\end{choices}
\question Uma corrente elétrica de    9.29 A percorre um fio de cobre. Sabendo-se que a carga de um elétron é igual a $1,6\times 10^{-19}\;C$, qual é o número de elétrons que atravessa, por minuto, a seção reta desse fio?

\begin{oneparchoices}
\choice 3.5e+21 \choice 8.1e+19 \choice 8.9e-17 \choice 8.2e+19 \choice 4.7e+19 \choice 2.2e+19 \choice 8.1e+19 \choice 1.5e-18 \choice 6.0e+19 \choice 5.8e+19 
\end{oneparchoices}\question Uma diferença de potencial de 120 V é aplicada a uma bomba d’água. Sabe-se que em funcionamento, o motor da bomba é percorrido por uma corrente de    2.15 A. Qual é a potência desenvolvida nesse motor?

\begin{oneparchoices}
\choice 6.7e+03 W\choice  55.933 W\choice 2.8e+04 W\choice 2.2e+04 W\choice 552.332 W\choice   0.018 W\choice 1.0e+04 W\choice 1.1e+04 W\choice 5.3e+03 W\choice 257.449 W
\end{oneparchoices}\question A figura abaixo mostra a trajetória de uma partícula eletricamente carregada. $\vec{{v}}$ representa a velocidade atravessando um campo magnético $\vec{{B}}$. Determine a sua trajetória devido a ação da força magnética atuando sobre ela.
        
        \begin{center}
            \begin{minipage}[c]{0.5\linewidth}
                \begin{tikzpicture}[scale=0.5,transform shape, font=\Large]

                    \tkzInit[xmin=-3,xmax=3,ymin=-3,ymax=3]
                %	\tkzGrid[color=gray!20]
                    \tkzClip[space=1.0]

                    \tkzDefPoints{0/0/O,4/0/P}

                    \foreach \x in {-2.5,-1.5,...,2.5}{
                        \foreach \y in {-2.5,-1.5,...,2.5}{
                        \tkzDefPoint(\x,\y){B}
                        \tkzText(B){x}
                }
                }

                \draw[->, line width=1pt, color=red] (0,0) --++ (0,1.5) node [above] {$\vec{v}$};

                    \node[circle, radius=0.25, ball color=gray!50] (n1) at (0,0) {+};

                    \tkzText[above right=0.25cm](B){$\vec{B}$}

                \end{tikzpicture}
            \end{minipage}
        \end{center}

        

\begin{choices}
\choice Paralelo ao papel e circular no sentido anti-horário. 
\choice Paralelo ao papel e circular no sentido horário. 
\choice Paralelo ao papel e da direita para a esquerda. 
\choice Paralelo ao papel e na vertical. 
\choice Paralelo ao papel e da esquerda para a direita. 
\end{choices}
\question Uma partícula de carga 8.15e-06 C é lançada em um campo magnético uniforme de    0.17 T , com uma velocidade de 926.97 m/s. Calcule o valor da força magnética atuando na carga se o ângulo entre a velocidade e o campo magnético for   54.84 graus.

\begin{oneparchoices}
\choice 2.6e-04 N\choice 2.3e-04 N\choice   0.069 N\choice 1.0e-03 N\choice 1.5e-03 N\choice 7.2e-04 N\choice -1.2e-03 N\choice 1.2e-04 N\choice 5.7e-04 N\choice 4.5e-04 N
\end{oneparchoices}\end{multicols*}
\end{questions}
\newpage

    \begin{minipage}[b]{0.75\linewidth}
        \begin{flushleft}
            {\bf \large Prova bimestral}
        \end{flushleft}
        \begin{flushleft}
            {\bf \large LQ2N (2B), 31 de outubro de 2022}
        \end{flushleft}
    \end{minipage}
    \begin{minipage}[b]{0.20\linewidth}
        \begin{flushright}
            {\bf \large Code: 1}
        \end{flushright}
    \end{minipage}
    \vspace{0.5cm} \hrule \vspace{0.5cm}
    \begin{minipage}{0.50\linewidth}
        \begin{flushleft}
            Student: Flaviano - IFPR
        \end{flushleft}
    \end{minipage}
    \begin{minipage}{0.20\linewidth}
        \begin{center}
            Date: 2022-11-14
        \end{center}
    \end{minipage}
    \begin{minipage}{0.20\linewidth}
        \begin{flushright}
            Class: LQ2N
        \end{flushright}
    \end{minipage}
    \vspace{0.5cm} \hrule \vspace{0.5cm}
    \begin{questions}
\begin{multicols*}{2}
\question Considere a figura abaixo onde as linhas trajeçadas representam superfícies equipotenciais Se colocarmos um elétron próximo a carga Q, quais trechos possíveis o elétron poderá se deslocar?
        
        \begin{center}
            \begin{minipage}[c]{0.5\linewidth}
                \begin{tikzpicture}[scale=0.5,transform shape, font=\Large]

                \tkzInit[xmin=-4,xmax=4,ymin=-4,ymax=4]
                \tkzClip[space=0.5]

                \tkzDefPoints{0/0/O,4/0/P}

                \foreach \x in {0.5,1.25,2.25,3,4}{
                    \tkzDrawCircle[R,dashed,color=gray!50](O,\x)
                }

                \foreach \y in {0,1,...,11}{
                    \tkzDefPointsBy[rotation= center O angle 30*\y](O,P){P1,P2}
                \draw[->, line width=1.0pt] (O) -- (P2);}

                \tkzDefPoints{3/0/a,4/0/b,0/4/c,0/3/d}

                \tkzDrawPoints[color=red,fill=red,size=0.3cm](a,b,c,d)

                \tkzDrawPoints(O)
                \tkzLabelPoints[above right,font=\Large](a,b,c,d)

                \node[circle, radius=0.25, ball color=gray!50] (n1) at (0,0) {Q};

                \end{tikzpicture}
            \end{minipage}
        \end{center}
        
        

\begin{choices}
\choice $a\rightarrow b$ ou $d\rightarrow c$ 
\choice $c\rightarrow b$ ou $d\rightarrow a$ 
\choice $b\rightarrow a$ ou $c\rightarrow d$ 
\choice $b\rightarrow c$ ou $a\rightarrow d$ 
\choice $b\rightarrow a\rightarrow d\rightarrow c$ ou $c\rightarrow d\rightarrow a\rightarrow b$ 
\end{choices}
\question Uma corrente elétrica de    1.52 A percorre um fio de cobre. Sabendo-se que a carga de um elétron é igual a $1,6\times 10^{-19}\;C$, qual é o número de elétrons que atravessa, por minuto, a seção reta desse fio?

\begin{oneparchoices}
\choice 3.0e+19 \choice 5.7e+20 \choice 1.5e-17 \choice 2.4e-19 \choice 9.5e+18 \choice 3.7e+19 \choice 5.9e+19 \choice 7.0e+19 \choice 7.6e+19 \choice 4.7e+19 
\end{oneparchoices}\question Uma diferença de potencial de 120 V é aplicada a uma bomba d’água. Sabe-se que em funcionamento, o motor da bomba é percorrido por uma corrente de    2.17 A. Qual é a potência desenvolvida nesse motor?

\begin{oneparchoices}
\choice  55.383 W\choice 1.8e+04 W\choice 3.0e+04 W\choice 3.4e+04 W\choice   0.018 W\choice 3.5e+03 W\choice 563.368 W\choice 260.008 W\choice 3.3e+04 W\choice 2.5e+04 W
\end{oneparchoices}\question A figura abaixo mostra a trajetória de uma partícula eletricamente carregada. $\vec{{v}}$ representa a velocidade atravessando um campo magnético $\vec{{B}}$. Determine a sua trajetória devido a ação da força magnética atuando sobre ela.
        
        \begin{center}
            \begin{minipage}[c]{0.5\linewidth}
                \begin{tikzpicture}[scale=0.5,transform shape, font=\Large]

                    \tkzInit[xmin=-3,xmax=3,ymin=-3,ymax=3]
                %	\tkzGrid[color=gray!20]
                    \tkzClip[space=1.0]

                    \tkzDefPoints{0/0/O,4/0/P}

                    \foreach \x in {-2.5,-1.5,...,2.5}{
                        \foreach \y in {-2.5,-1.5,...,2.5}{
                        \tkzDefPoint(\x,\y){B}
                        \tkzText(B){x}
                }
                }

                \draw[->, line width=1pt, color=red] (0,0) --++ (0,1.5) node [above] {$\vec{v}$};

                    \node[circle, radius=0.25, ball color=gray!50] (n1) at (0,0) {+};

                    \tkzText[above right=0.25cm](B){$\vec{B}$}

                \end{tikzpicture}
            \end{minipage}
        \end{center}

        

\begin{choices}
\choice Paralelo ao papel e circular no sentido horário. 
\choice Paralelo ao papel e da direita para a esquerda. 
\choice Paralelo ao papel e da esquerda para a direita. 
\choice Paralelo ao papel e na vertical. 
\choice Paralelo ao papel e circular no sentido anti-horário. 
\end{choices}
\question Uma partícula de carga 7.03e-06 C é lançada em um campo magnético uniforme de    0.73 T , com uma velocidade de 881.83 m/s. Calcule o valor da força magnética atuando na carga se o ângulo entre a velocidade e o campo magnético for   72.25 graus.

\begin{oneparchoices}
\choice 7.9e-04 N\choice 5.4e-06 N\choice 1.2e-05 N\choice 1.4e-03 N\choice 1.5e-03 N\choice 4.3e-03 N\choice 4.0e-04 N\choice 8.9e-04 N\choice 7.3e-04 N\choice   0.326 N
\end{oneparchoices}\end{multicols*}
\end{questions}
\newpage
\end{document}