
\documentclass[12pt, addpoints]{exam}
\usepackage[utf8]{inputenc}
\usepackage[portuguese]{babel}
\usepackage{multicol}
\usepackage{graphicx}
\usepackage{amsmath}
\usepackage{xcolor}
\usepackage{tikz,pgfplots,tikz-3dplot,bm}
\usepackage{circuitikz}
\usepackage{tkz-base}
\usepackage{tkz-fct}
\usepackage{tkz-euclide}
\usepackage[a4paper, portrait, margin=2cm]{geometry}

\usetikzlibrary{arrows,3d,calc,automata,positioning,shadows,math,fit,shapes}
\usetikzlibrary{patterns,hobby,optics,calc}
\tikzset{>=stealth, thick, global scale/.style={scale=#1,every node/.style={scale=#1}}}
\setlength{\columnsep}{1cm}
\renewcommand{\choiceshook}{\setlength{\leftmargin}{0pt}}

        \begin{document}

        \begin{minipage}[b]{0.75\linewidth}
            \begin{flushleft}
                {\bf \large Prova bimestral}
            \end{flushleft}
            \begin{flushleft}
                {\bf \large LQ2N (2B), 31 de outubro de 2022}
            \end{flushleft}
        \end{minipage}
        \begin{minipage}[b]{0.20\linewidth}
            \begin{flushright}
                {\bf \large Code: 0}
            \end{flushright}
        \end{minipage}
        \vspace{0.5cm} \hrule \vspace{0.5cm}
        \begin{minipage}{0.75\linewidth}
            \begin{flushleft}
                Student: Alessandra Alves dos Santos
            \end{flushleft}
        \end{minipage}
        \begin{minipage}{0.20\linewidth}
            \begin{flushright}
                Class: LQ2N
            \end{flushright}
        \end{minipage}
        \vspace{0.5cm} \hrule \vspace{0.5cm}
        \begin{questions}
\begin{multicols*}{2}
\question A função trabalho do sódio é 2,28 eV, determine a energia cinética dos elétrons que são emitidos desse material quando ele é bombardeado por radiação com comprimento de onda de  319.66 nm.

\begin{oneparchoices}
\choice  -2.280 eV\choice   0.041 eV\choice   2.352 eV\choice   3.333 eV\choice   0.167 eV\choice   1.599 eV\choice   0.643 eV\choice   0.629 eV\choice  -2.280 eV\choice   6.159 eV
\end{oneparchoices}\end{multicols*}
\end{questions}
\newpage\end{document}