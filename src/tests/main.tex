
\documentclass[12pt, addpoints]{exam}
\usepackage[utf8]{inputenc}
\usepackage[portuguese]{babel}
\usepackage{multicol}
\usepackage{graphicx}
\usepackage{amsmath}
\usepackage{xcolor}
\usepackage{tikz,pgfplots,tikz-3dplot,bm}
\usepackage{circuitikz}
\usepackage{tkz-base}
\usepackage{tkz-fct}
\usepackage{tkz-euclide}
\usepackage[a4paper, portrait, margin=2cm]{geometry}

\usetikzlibrary{arrows,3d,calc,automata,positioning,shadows,math,fit,shapes}
\usetikzlibrary{patterns,hobby,optics,calc}
\tikzset{>=stealth, thick, global scale/.style={scale=#1,every node/.style={scale=#1}}}
\setlength{\columnsep}{1cm}
\renewcommand{\choiceshook}{\setlength{\leftmargin}{0pt}}

        \begin{document}

        \begin{minipage}[b]{0.75\linewidth}
            \begin{flushleft}
                {\bf \large Prova bimestral}
            \end{flushleft}
            \begin{flushleft}
                {\bf \large LQ2N (2B), 31 de outubro de 2022}
            \end{flushleft}
        \end{minipage}
        \begin{minipage}[b]{0.20\linewidth}
            \begin{flushright}
                {\bf \large Code: 0}
            \end{flushright}
        \end{minipage}
        \vspace{0.5cm} \hrule \vspace{0.5cm}
        \begin{minipage}{0.75\linewidth}
            \begin{flushleft}
                Student: Alessandra Alves dos Santos
            \end{flushleft}
        \end{minipage}
        \begin{minipage}{0.20\linewidth}
            \begin{flushright}
                Class: LQ2N
            \end{flushright}
        \end{minipage}
        \vspace{0.5cm} \hrule \vspace{0.5cm}
        \begin{questions}
\begin{multicols*}{2}
\question A função trabalho do sódio é 2,28 eV, determine a energia cinética dos elétrons que são emitidos desse material quando ele é bombardeado por radiação com comprimento de onda de  228.51 nm.

\begin{oneparchoices}
\choice   0.553 eV\choice   0.873 eV\choice   3.665 eV\choice   3.146 eV\choice   3.259 eV\choice  -2.280 eV\choice   7.706 eV\choice   0.920 eV\choice   3.633 eV\choice  -2.280 eV
\end{oneparchoices}\end{multicols*}
\end{questions}
\newpage
        \begin{minipage}[b]{0.75\linewidth}
            \begin{flushleft}
                {\bf \large Prova bimestral}
            \end{flushleft}
            \begin{flushleft}
                {\bf \large LQ2N (2B), 31 de outubro de 2022}
            \end{flushleft}
        \end{minipage}
        \begin{minipage}[b]{0.20\linewidth}
            \begin{flushright}
                {\bf \large Code: 0}
            \end{flushright}
        \end{minipage}
        \vspace{0.5cm} \hrule \vspace{0.5cm}
        \begin{minipage}{0.75\linewidth}
            \begin{flushleft}
                Student: Alessandra Alves dos Santos
            \end{flushleft}
        \end{minipage}
        \begin{minipage}{0.20\linewidth}
            \begin{flushright}
                Class: LQ2N
            \end{flushright}
        \end{minipage}
        \vspace{0.5cm} \hrule \vspace{0.5cm}
        \begin{questions}
\begin{multicols*}{2}
\question A função trabalho do sódio é 2,28 eV, determine a energia cinética dos elétrons que são emitidos desse material quando ele é bombardeado por radiação com comprimento de onda de  239.85 nm.

\begin{oneparchoices}
\choice   2.889 eV\choice   0.649 eV\choice   1.297 eV\choice   7.449 eV\choice   2.158 eV\choice   1.571 eV\choice   1.512 eV\choice   1.322 eV\choice  -2.280 eV\choice  -2.280 eV
\end{oneparchoices}\end{multicols*}
\end{questions}
\newpage
        \begin{minipage}[b]{0.75\linewidth}
            \begin{flushleft}
                {\bf \large Prova bimestral}
            \end{flushleft}
            \begin{flushleft}
                {\bf \large LQ2N (2B), 31 de outubro de 2022}
            \end{flushleft}
        \end{minipage}
        \begin{minipage}[b]{0.20\linewidth}
            \begin{flushright}
                {\bf \large Code: 0}
            \end{flushright}
        \end{minipage}
        \vspace{0.5cm} \hrule \vspace{0.5cm}
        \begin{minipage}{0.75\linewidth}
            \begin{flushleft}
                Student: Alessandra Alves dos Santos
            \end{flushleft}
        \end{minipage}
        \begin{minipage}{0.20\linewidth}
            \begin{flushright}
                Class: LQ2N
            \end{flushright}
        \end{minipage}
        \vspace{0.5cm} \hrule \vspace{0.5cm}
        \begin{questions}
\begin{multicols*}{2}
\question A função trabalho do sódio é 2,28 eV, determine a energia cinética dos elétrons que são emitidos desse material quando ele é bombardeado por radiação com comprimento de onda de  283.93 nm.

\begin{oneparchoices}
\choice   0.573 eV\choice   0.967 eV\choice   0.353 eV\choice   2.087 eV\choice   0.369 eV\choice  -2.280 eV\choice  -2.280 eV\choice   6.647 eV\choice   1.446 eV\choice   2.015 eV
\end{oneparchoices}\end{multicols*}
\end{questions}
\newpage
        \begin{minipage}[b]{0.75\linewidth}
            \begin{flushleft}
                {\bf \large Prova bimestral}
            \end{flushleft}
            \begin{flushleft}
                {\bf \large LQ2N (2B), 31 de outubro de 2022}
            \end{flushleft}
        \end{minipage}
        \begin{minipage}[b]{0.20\linewidth}
            \begin{flushright}
                {\bf \large Code: 0}
            \end{flushright}
        \end{minipage}
        \vspace{0.5cm} \hrule \vspace{0.5cm}
        \begin{minipage}{0.75\linewidth}
            \begin{flushleft}
                Student: Alessandra Alves dos Santos
            \end{flushleft}
        \end{minipage}
        \begin{minipage}{0.20\linewidth}
            \begin{flushright}
                Class: LQ2N
            \end{flushright}
        \end{minipage}
        \vspace{0.5cm} \hrule \vspace{0.5cm}
        \begin{questions}
\begin{multicols*}{2}
\question A função trabalho do sódio é 2,28 eV, determine a energia cinética dos elétrons que são emitidos desse material quando ele é bombardeado por radiação com comprimento de onda de  335.59 nm.

\begin{oneparchoices}
\choice   1.414 eV\choice   5.974 eV\choice   0.386 eV\choice  -2.280 eV\choice   0.364 eV\choice   0.451 eV\choice   1.228 eV\choice  -2.280 eV\choice   2.044 eV\choice   0.590 eV
\end{oneparchoices}\end{multicols*}
\end{questions}
\newpage
        \begin{minipage}[b]{0.75\linewidth}
            \begin{flushleft}
                {\bf \large Prova bimestral}
            \end{flushleft}
            \begin{flushleft}
                {\bf \large LQ2N (2B), 31 de outubro de 2022}
            \end{flushleft}
        \end{minipage}
        \begin{minipage}[b]{0.20\linewidth}
            \begin{flushright}
                {\bf \large Code: 0}
            \end{flushright}
        \end{minipage}
        \vspace{0.5cm} \hrule \vspace{0.5cm}
        \begin{minipage}{0.75\linewidth}
            \begin{flushleft}
                Student: Alessandra Alves dos Santos
            \end{flushleft}
        \end{minipage}
        \begin{minipage}{0.20\linewidth}
            \begin{flushright}
                Class: LQ2N
            \end{flushright}
        \end{minipage}
        \vspace{0.5cm} \hrule \vspace{0.5cm}
        \begin{questions}
\begin{multicols*}{2}
\question A função trabalho do sódio é 2,28 eV, determine a energia cinética dos elétrons que são emitidos desse material quando ele é bombardeado por radiação com comprimento de onda de  536.01 nm.

\begin{oneparchoices}
\choice   1.757 eV\choice  -2.280 eV\choice   1.085 eV\choice   0.112 eV\choice   0.033 eV\choice   4.593 eV\choice   1.325 eV\choice   0.757 eV\choice   0.020 eV\choice  -2.280 eV
\end{oneparchoices}\end{multicols*}
\end{questions}
\newpage
        \begin{minipage}[b]{0.75\linewidth}
            \begin{flushleft}
                {\bf \large Prova bimestral}
            \end{flushleft}
            \begin{flushleft}
                {\bf \large LQ2N (2B), 31 de outubro de 2022}
            \end{flushleft}
        \end{minipage}
        \begin{minipage}[b]{0.20\linewidth}
            \begin{flushright}
                {\bf \large Code: 0}
            \end{flushright}
        \end{minipage}
        \vspace{0.5cm} \hrule \vspace{0.5cm}
        \begin{minipage}{0.75\linewidth}
            \begin{flushleft}
                Student: Alessandra Alves dos Santos
            \end{flushleft}
        \end{minipage}
        \begin{minipage}{0.20\linewidth}
            \begin{flushright}
                Class: LQ2N
            \end{flushright}
        \end{minipage}
        \vspace{0.5cm} \hrule \vspace{0.5cm}
        \begin{questions}
\begin{multicols*}{2}
\question A função trabalho do sódio é 2,28 eV, determine a energia cinética dos elétrons que são emitidos desse material quando ele é bombardeado por radiação com comprimento de onda de  538.31 nm.

\begin{oneparchoices}
\choice   1.714 eV\choice   1.006 eV\choice   0.648 eV\choice  -2.280 eV\choice  -2.280 eV\choice   0.127 eV\choice   2.772 eV\choice   0.124 eV\choice   4.583 eV\choice   0.023 eV
\end{oneparchoices}\end{multicols*}
\end{questions}
\newpage\end{document}