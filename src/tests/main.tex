
\documentclass[12pt, addpoints]{exam}
\usepackage[utf8]{inputenc}
\usepackage[portuguese]{babel}
\usepackage{multicol}
\usepackage{graphicx}
\usepackage{amsmath}
\usepackage{xcolor}
\usepackage{tikz,pgfplots,tikz-3dplot,bm}
\usepackage{circuitikz}
\usepackage{tkz-base}
\usepackage{tkz-fct}
\usepackage{tkz-euclide}
\usepackage[a4paper, portrait, margin=2cm]{geometry}

\usetikzlibrary{arrows,3d,calc,automata,positioning,shadows,math,fit,shapes}
\usetikzlibrary{patterns,hobby,optics,calc}
\tikzset{>=stealth, thick, global scale/.style={scale=#1,every node/.style={scale=#1}}}
\setlength{\columnsep}{1cm}
\renewcommand{\choiceshook}{\setlength{\leftmargin}{0pt}}

        \begin{document}

        \begin{minipage}[b]{0.75\linewidth}
            \begin{flushleft}
                {\bf \large Prova bimestral}
            \end{flushleft}
            \begin{flushleft}
                {\bf \large LQ2N (2B), 31 de outubro de 2022}
            \end{flushleft}
        \end{minipage}
        \begin{minipage}[b]{0.20\linewidth}
            \begin{flushright}
                {\bf \large Código: 0}
            \end{flushright}
        \end{minipage}
        \vspace{0.5cm} \hrule \vspace{0.5cm}
        \begin{minipage}{0.75\linewidth}
            Aluno:
        \end{minipage}
        \vspace{0.5cm} \hrule \vspace{0.5cm}

        \begin{questions}
\begin{multicols*}{2}
\question[20] Determine a energia E da partícula que possui a função de onda $\Psi(x,t)=Ae^{i(kx-\omega t)}$

\begin{choices}
\choice $\infty$ \choice $E=\hbar k$ \choice $E = \frac{mk^2}{2}$ \choice $E=\frac{\hbar^2k^2}{2m}$ \choice Zero \end{choices}
\end{multicols*}
\end{questions}
\newpage
        \begin{minipage}[b]{0.75\linewidth}
            \begin{flushleft}
                {\bf \large Prova bimestral}
            \end{flushleft}
            \begin{flushleft}
                {\bf \large LQ2N (2B), 31 de outubro de 2022}
            \end{flushleft}
        \end{minipage}
        \begin{minipage}[b]{0.20\linewidth}
            \begin{flushright}
                {\bf \large Código: 1}
            \end{flushright}
        \end{minipage}
        \vspace{0.5cm} \hrule \vspace{0.5cm}
        \begin{minipage}{0.75\linewidth}
            Aluno:
        \end{minipage}
        \vspace{0.5cm} \hrule \vspace{0.5cm}

        \begin{questions}
\begin{multicols*}{2}
\question[20] Determine a energia E da partícula que possui a função de onda $\Psi(x,t)=Ae^{i(kx-\omega t)}$

\begin{choices}
\choice $E=\hbar k$ \choice $E = \frac{mk^2}{2}$ \choice $E=\frac{\hbar^2k^2}{2m}$ \choice Zero \choice $\infty$ \end{choices}
\end{multicols*}
\end{questions}
\newpage
        \begin{minipage}[b]{0.75\linewidth}
            \begin{flushleft}
                {\bf \large Prova bimestral}
            \end{flushleft}
            \begin{flushleft}
                {\bf \large LQ2N (2B), 31 de outubro de 2022}
            \end{flushleft}
        \end{minipage}
        \begin{minipage}[b]{0.20\linewidth}
            \begin{flushright}
                {\bf \large Código: 2}
            \end{flushright}
        \end{minipage}
        \vspace{0.5cm} \hrule \vspace{0.5cm}
        \begin{minipage}{0.75\linewidth}
            Aluno:
        \end{minipage}
        \vspace{0.5cm} \hrule \vspace{0.5cm}

        \begin{questions}
\begin{multicols*}{2}
\question[20] Determine a energia E da partícula que possui a função de onda $\Psi(x,t)=Ae^{i(kx-\omega t)}$

\begin{choices}
\choice $E=\frac{\hbar^2k^2}{2m}$ \choice $E = \frac{mk^2}{2}$ \choice $E=\hbar k$ \choice $\infty$ \choice Zero \end{choices}
\end{multicols*}
\end{questions}
\newpage
        \begin{minipage}[b]{0.75\linewidth}
            \begin{flushleft}
                {\bf \large Prova bimestral}
            \end{flushleft}
            \begin{flushleft}
                {\bf \large LQ2N (2B), 31 de outubro de 2022}
            \end{flushleft}
        \end{minipage}
        \begin{minipage}[b]{0.20\linewidth}
            \begin{flushright}
                {\bf \large Código: 3}
            \end{flushright}
        \end{minipage}
        \vspace{0.5cm} \hrule \vspace{0.5cm}
        \begin{minipage}{0.75\linewidth}
            Aluno:
        \end{minipage}
        \vspace{0.5cm} \hrule \vspace{0.5cm}

        \begin{questions}
\begin{multicols*}{2}
\question[20] Determine a energia E da partícula que possui a função de onda $\Psi(x,t)=Ae^{i(kx-\omega t)}$

\begin{choices}
\choice $E=\frac{\hbar^2k^2}{2m}$ \choice $E=\hbar k$ \choice $\infty$ \choice Zero \choice $E = \frac{mk^2}{2}$ \end{choices}
\end{multicols*}
\end{questions}
\newpage
        \begin{minipage}[b]{0.75\linewidth}
            \begin{flushleft}
                {\bf \large Prova bimestral}
            \end{flushleft}
            \begin{flushleft}
                {\bf \large LQ2N (2B), 31 de outubro de 2022}
            \end{flushleft}
        \end{minipage}
        \begin{minipage}[b]{0.20\linewidth}
            \begin{flushright}
                {\bf \large Código: 4}
            \end{flushright}
        \end{minipage}
        \vspace{0.5cm} \hrule \vspace{0.5cm}
        \begin{minipage}{0.75\linewidth}
            Aluno:
        \end{minipage}
        \vspace{0.5cm} \hrule \vspace{0.5cm}

        \begin{questions}
\begin{multicols*}{2}
\question[20] Determine a energia E da partícula que possui a função de onda $\Psi(x,t)=Ae^{i(kx-\omega t)}$

\begin{choices}
\choice $E = \frac{mk^2}{2}$ \choice Zero \choice $E=\frac{\hbar^2k^2}{2m}$ \choice $E=\hbar k$ \choice $\infty$ \end{choices}
\end{multicols*}
\end{questions}
\newpage
        \begin{minipage}[b]{0.75\linewidth}
            \begin{flushleft}
                {\bf \large Prova bimestral}
            \end{flushleft}
            \begin{flushleft}
                {\bf \large LQ2N (2B), 31 de outubro de 2022}
            \end{flushleft}
        \end{minipage}
        \begin{minipage}[b]{0.20\linewidth}
            \begin{flushright}
                {\bf \large Código: 5}
            \end{flushright}
        \end{minipage}
        \vspace{0.5cm} \hrule \vspace{0.5cm}
        \begin{minipage}{0.75\linewidth}
            Aluno:
        \end{minipage}
        \vspace{0.5cm} \hrule \vspace{0.5cm}

        \begin{questions}
\begin{multicols*}{2}
\question[20] Determine a energia E da partícula que possui a função de onda $\Psi(x,t)=Ae^{i(kx-\omega t)}$

\begin{choices}
\choice Zero \choice $\infty$ \choice $E=\hbar k$ \choice $E=\frac{\hbar^2k^2}{2m}$ \choice $E = \frac{mk^2}{2}$ \end{choices}
\end{multicols*}
\end{questions}
\newpage
        \begin{minipage}[b]{0.75\linewidth}
            \begin{flushleft}
                {\bf \large Prova bimestral}
            \end{flushleft}
            \begin{flushleft}
                {\bf \large LQ2N (2B), 31 de outubro de 2022}
            \end{flushleft}
        \end{minipage}
        \begin{minipage}[b]{0.20\linewidth}
            \begin{flushright}
                {\bf \large Código: 6}
            \end{flushright}
        \end{minipage}
        \vspace{0.5cm} \hrule \vspace{0.5cm}
        \begin{minipage}{0.75\linewidth}
            Aluno:
        \end{minipage}
        \vspace{0.5cm} \hrule \vspace{0.5cm}

        \begin{questions}
\begin{multicols*}{2}
\question[20] Determine a energia E da partícula que possui a função de onda $\Psi(x,t)=Ae^{i(kx-\omega t)}$

\begin{choices}
\choice Zero \choice $E=\hbar k$ \choice $\infty$ \choice $E=\frac{\hbar^2k^2}{2m}$ \choice $E = \frac{mk^2}{2}$ \end{choices}
\end{multicols*}
\end{questions}
\newpage
        \begin{minipage}[b]{0.75\linewidth}
            \begin{flushleft}
                {\bf \large Prova bimestral}
            \end{flushleft}
            \begin{flushleft}
                {\bf \large LQ2N (2B), 31 de outubro de 2022}
            \end{flushleft}
        \end{minipage}
        \begin{minipage}[b]{0.20\linewidth}
            \begin{flushright}
                {\bf \large Código: 7}
            \end{flushright}
        \end{minipage}
        \vspace{0.5cm} \hrule \vspace{0.5cm}
        \begin{minipage}{0.75\linewidth}
            Aluno:
        \end{minipage}
        \vspace{0.5cm} \hrule \vspace{0.5cm}

        \begin{questions}
\begin{multicols*}{2}
\question[20] Determine a energia E da partícula que possui a função de onda $\Psi(x,t)=Ae^{i(kx-\omega t)}$

\begin{choices}
\choice Zero \choice $\infty$ \choice $E=\hbar k$ \choice $E=\frac{\hbar^2k^2}{2m}$ \choice $E = \frac{mk^2}{2}$ \end{choices}
\end{multicols*}
\end{questions}
\newpage
        \begin{minipage}[b]{0.75\linewidth}
            \begin{flushleft}
                {\bf \large Prova bimestral}
            \end{flushleft}
            \begin{flushleft}
                {\bf \large LQ2N (2B), 31 de outubro de 2022}
            \end{flushleft}
        \end{minipage}
        \begin{minipage}[b]{0.20\linewidth}
            \begin{flushright}
                {\bf \large Código: 8}
            \end{flushright}
        \end{minipage}
        \vspace{0.5cm} \hrule \vspace{0.5cm}
        \begin{minipage}{0.75\linewidth}
            Aluno:
        \end{minipage}
        \vspace{0.5cm} \hrule \vspace{0.5cm}

        \begin{questions}
\begin{multicols*}{2}
\question[20] Determine a energia E da partícula que possui a função de onda $\Psi(x,t)=Ae^{i(kx-\omega t)}$

\begin{choices}
\choice $\infty$ \choice $E = \frac{mk^2}{2}$ \choice Zero \choice $E=\frac{\hbar^2k^2}{2m}$ \choice $E=\hbar k$ \end{choices}
\end{multicols*}
\end{questions}
\newpage
        \begin{minipage}[b]{0.75\linewidth}
            \begin{flushleft}
                {\bf \large Prova bimestral}
            \end{flushleft}
            \begin{flushleft}
                {\bf \large LQ2N (2B), 31 de outubro de 2022}
            \end{flushleft}
        \end{minipage}
        \begin{minipage}[b]{0.20\linewidth}
            \begin{flushright}
                {\bf \large Código: 9}
            \end{flushright}
        \end{minipage}
        \vspace{0.5cm} \hrule \vspace{0.5cm}
        \begin{minipage}{0.75\linewidth}
            Aluno:
        \end{minipage}
        \vspace{0.5cm} \hrule \vspace{0.5cm}

        \begin{questions}
\begin{multicols*}{2}
\question[20] Determine a energia E da partícula que possui a função de onda $\Psi(x,t)=Ae^{i(kx-\omega t)}$

\begin{choices}
\choice $E = \frac{mk^2}{2}$ \choice $E=\hbar k$ \choice $E=\frac{\hbar^2k^2}{2m}$ \choice $\infty$ \choice Zero \end{choices}
\end{multicols*}
\end{questions}
\newpage\end{document}