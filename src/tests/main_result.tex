
\documentclass[12pt, addpoints]{exam}
\usepackage[utf8]{inputenc}
\usepackage[portuguese]{babel}
\usepackage{multicol}
\usepackage{graphicx}
\usepackage{amsmath}
\usepackage{xcolor}
\usepackage{tikz,pgfplots,tikz-3dplot,bm}
\usepackage{circuitikz}
\usepackage{tkz-base}
\usepackage{tkz-fct}
\usepackage{tkz-euclide}
\usepackage[a4paper, portrait, margin=2cm]{geometry}

\usetikzlibrary{arrows,3d,calc,automata,positioning,shadows,math,fit,shapes}
\usetikzlibrary{patterns,hobby,optics,calc}
\tikzset{>=stealth, thick, global scale/.style={scale=#1,every node/.style={scale=#1}}}
\setlength{\columnsep}{1cm}
\renewcommand{\choiceshook}{\setlength{\leftmargin}{0pt}}

        \begin{document}

        \begin{minipage}[b]{0.75\linewidth}
            \begin{flushleft}
                {\bf \large Prova bimestral}
            \end{flushleft}
            \begin{flushleft}
                {\bf \large LQ2N (2B), 31 de outubro de 2022}
            \end{flushleft}
        \end{minipage}
        \begin{minipage}[b]{0.20\linewidth}
            \begin{flushright}
                {\bf \large Código: 0}
            \end{flushright}
        \end{minipage}
        \vspace{0.5cm} \hrule \vspace{0.5cm}
        \begin{minipage}{0.75\linewidth}
            Aluno:
        \end{minipage}
        \vspace{0.5cm} \hrule \vspace{0.5cm}

        \begin{center}
\textcolor{red}{\emph\Large Correcting version}\end{center}
\begin{questions}
\begin{multicols*}{2}
\question[20] A figura abaixo mostra a trajetória de uma partícula eletricamente carregada. $\\vec{{v}}$ representa a velocidade atravessando um campo magnético $\\vec{{B}}$. Determine a sua trajetória devido a ação da força magnética atuando sobre ela.
        
        \begin{center}
            \begin{minipage}[c]{0.5\linewidth}
                \begin{tikzpicture}[scale=0.5,transform shape, font=\Large]

                    \tkzInit[xmin=-3,xmax=3,ymin=-3,ymax=3]
                %	\tkzGrid[color=gray!20]
                    \tkzClip[space=1.0]

                    \tkzDefPoints{0/0/O,4/0/P}

                    \foreach \x in {-2.5,-1.5,...,2.5}{
                        \foreach \y in {-2.5,-1.5,...,2.5}{
                        \tkzDefPoint(\x,\y){B}
                        \tkzText(B){x}
                }
                }

                \draw[->, line width=1pt, color=red] (0,0) --++ (0,1.5) node [above] {$\vec{v}$};

                    \node[circle, radius=0.25, ball color=gray!50] (n1) at (0,0) {+};

                    \tkzText[above right=0.25cm](B){$\vec{B}$}

                \end{tikzpicture}
            \end{minipage}
        \end{center}

        

\begin{oneparchoices}
\choice 10.0\choice 20\choice 0.0\choice 0.0\choice 0.0\end{oneparchoices}
\question[20] Uma corrente elétrica de    6.34 A percorre um fio de cobre. Sabendo-se que a carga de um elétron é igual a $1,6\times 10^{-19}\;C$, qual é o número de elétrons que atravessa, por minuto, a seção reta desse fio?

\begin{oneparchoices}
\choice 10.0\choice 0.0\choice 0.0\choice 10.0\choice 15.0\choice 0.0\choice 0.0\choice 0.0\choice 0.0\choice 20\end{oneparchoices}
\question[20] Considere a figura abaixo onde as linhas trajeçadas representam superfícies equipotenciais Se colocarmos um elétron próximo a carga Q, quais trechos possíveis o elétron poderá se deslocar?
        
        \begin{center}
            \begin{minipage}[c]{0.5\linewidth}
                \begin{tikzpicture}[scale=0.5,transform shape, font=\Large]

                \tkzInit[xmin=-4,xmax=4,ymin=-4,ymax=4]
                \tkzClip[space=0.5]

                \tkzDefPoints{0/0/O,4/0/P}

                \foreach \x in {0.5,1.25,2.25,3,4}{
                    \tkzDrawCircle[R,dashed,color=gray!50](O,\x)
                }

                \foreach \y in {0,1,...,11}{
                    \tkzDefPointsBy[rotation= center O angle 30*\y](O,P){P1,P2}
                \draw[->, line width=1.0pt] (O) -- (P2);}

                \tkzDefPoints{3/0/a,4/0/b,0/4/c,0/3/d}

                \tkzDrawPoints[color=red,fill=red,size=0.3cm](a,b,c,d)

                \tkzDrawPoints(O)
                \tkzLabelPoints[above right,font=\Large](a,b,c,d)

                \node[circle, radius=0.25, ball color=gray!50] (n1) at (0,0) {Q};

                \end{tikzpicture}
            \end{minipage}
        \end{center}
        
        

\begin{oneparchoices}
\choice 0.0\choice 20\choice 0.0\choice 0.0\choice 0.0\end{oneparchoices}
\question[20] Uma diferença de potencial de 120 V é aplicada a uma bomba d’água. Sabe-se que em funcionamento, o motor da bomba é percorrido por uma corrente de    4.32 A. Qual é a potência desenvolvida nesse motor?

\begin{oneparchoices}
\choice 0.0\choice 0.0\choice 0.0\choice 20\choice 0.0\choice 0.0\choice 0.0\choice 0.0\choice 0.0\choice 0.0\end{oneparchoices}
\question[20] Uma partícula de carga 3.74e-06 C é lançada em um campo magnético uniforme de    0.49 T , com uma velocidade de 211.89 m/s. Calcule o valor da força magnética atuando na carga se o ângulo entre a velocidade e o campo magnético for   83.54 graus.

\begin{oneparchoices}
\choice 5.0\choice 0.0\choice 20\choice 15.0\choice 0.0\choice 0.0\choice 5.0\choice 0.0\choice 0.0\choice 0.0\end{oneparchoices}
\end{multicols*}
\end{questions}
\newpage
        \begin{minipage}[b]{0.75\linewidth}
            \begin{flushleft}
                {\bf \large Prova bimestral}
            \end{flushleft}
            \begin{flushleft}
                {\bf \large LQ2N (2B), 31 de outubro de 2022}
            \end{flushleft}
        \end{minipage}
        \begin{minipage}[b]{0.20\linewidth}
            \begin{flushright}
                {\bf \large Código: 1}
            \end{flushright}
        \end{minipage}
        \vspace{0.5cm} \hrule \vspace{0.5cm}
        \begin{minipage}{0.75\linewidth}
            Aluno:
        \end{minipage}
        \vspace{0.5cm} \hrule \vspace{0.5cm}

        \begin{center}
\textcolor{red}{\emph\Large Correcting version}\end{center}
\begin{questions}
\begin{multicols*}{2}
\question[20] A figura abaixo mostra a trajetória de uma partícula eletricamente carregada. $\\vec{{v}}$ representa a velocidade atravessando um campo magnético $\\vec{{B}}$. Determine a sua trajetória devido a ação da força magnética atuando sobre ela.
        
        \begin{center}
            \begin{minipage}[c]{0.5\linewidth}
                \begin{tikzpicture}[scale=0.5,transform shape, font=\Large]

                    \tkzInit[xmin=-3,xmax=3,ymin=-3,ymax=3]
                %	\tkzGrid[color=gray!20]
                    \tkzClip[space=1.0]

                    \tkzDefPoints{0/0/O,4/0/P}

                    \foreach \x in {-2.5,-1.5,...,2.5}{
                        \foreach \y in {-2.5,-1.5,...,2.5}{
                        \tkzDefPoint(\x,\y){B}
                        \tkzText(B){x}
                }
                }

                \draw[->, line width=1pt, color=red] (0,0) --++ (0,1.5) node [above] {$\vec{v}$};

                    \node[circle, radius=0.25, ball color=gray!50] (n1) at (0,0) {+};

                    \tkzText[above right=0.25cm](B){$\vec{B}$}

                \end{tikzpicture}
            \end{minipage}
        \end{center}

        

\begin{oneparchoices}
\choice 10.0\choice 0.0\choice 0.0\choice 20\choice 0.0\end{oneparchoices}
\question[20] Uma corrente elétrica de    8.10 A percorre um fio de cobre. Sabendo-se que a carga de um elétron é igual a $1,6\times 10^{-19}\;C$, qual é o número de elétrons que atravessa, por minuto, a seção reta desse fio?

\begin{oneparchoices}
\choice 0.0\choice 20\choice 15.0\choice 0.0\choice 0.0\choice 10.0\choice 0.0\choice 0.0\choice 10.0\choice 0.0\end{oneparchoices}
\question[20] Considere a figura abaixo onde as linhas trajeçadas representam superfícies equipotenciais Se colocarmos um elétron próximo a carga Q, quais trechos possíveis o elétron poderá se deslocar?
        
        \begin{center}
            \begin{minipage}[c]{0.5\linewidth}
                \begin{tikzpicture}[scale=0.5,transform shape, font=\Large]

                \tkzInit[xmin=-4,xmax=4,ymin=-4,ymax=4]
                \tkzClip[space=0.5]

                \tkzDefPoints{0/0/O,4/0/P}

                \foreach \x in {0.5,1.25,2.25,3,4}{
                    \tkzDrawCircle[R,dashed,color=gray!50](O,\x)
                }

                \foreach \y in {0,1,...,11}{
                    \tkzDefPointsBy[rotation= center O angle 30*\y](O,P){P1,P2}
                \draw[->, line width=1.0pt] (O) -- (P2);}

                \tkzDefPoints{3/0/a,4/0/b,0/4/c,0/3/d}

                \tkzDrawPoints[color=red,fill=red,size=0.3cm](a,b,c,d)

                \tkzDrawPoints(O)
                \tkzLabelPoints[above right,font=\Large](a,b,c,d)

                \node[circle, radius=0.25, ball color=gray!50] (n1) at (0,0) {Q};

                \end{tikzpicture}
            \end{minipage}
        \end{center}
        
        

\begin{oneparchoices}
\choice 0.0\choice 0.0\choice 20\choice 0.0\choice 0.0\end{oneparchoices}
\question[20] Uma diferença de potencial de 120 V é aplicada a uma bomba d’água. Sabe-se que em funcionamento, o motor da bomba é percorrido por uma corrente de    3.52 A. Qual é a potência desenvolvida nesse motor?

\begin{oneparchoices}
\choice 0.0\choice 0.0\choice 0.0\choice 0.0\choice 0.0\choice 0.0\choice 0.0\choice 0.0\choice 0.0\choice 20\end{oneparchoices}
\question[20] Uma partícula de carga 4.04e-06 C é lançada em um campo magnético uniforme de    0.98 T , com uma velocidade de 971.00 m/s. Calcule o valor da força magnética atuando na carga se o ângulo entre a velocidade e o campo magnético for   41.79 graus.

\begin{oneparchoices}
\choice 0.0\choice 0.0\choice 0.0\choice 0.0\choice 0.0\choice 0.0\choice 5.0\choice 15.0\choice 20\choice 5.0\end{oneparchoices}
\end{multicols*}
\end{questions}
\newpage
        \begin{minipage}[b]{0.75\linewidth}
            \begin{flushleft}
                {\bf \large Prova bimestral}
            \end{flushleft}
            \begin{flushleft}
                {\bf \large LQ2N (2B), 31 de outubro de 2022}
            \end{flushleft}
        \end{minipage}
        \begin{minipage}[b]{0.20\linewidth}
            \begin{flushright}
                {\bf \large Código: 2}
            \end{flushright}
        \end{minipage}
        \vspace{0.5cm} \hrule \vspace{0.5cm}
        \begin{minipage}{0.75\linewidth}
            Aluno:
        \end{minipage}
        \vspace{0.5cm} \hrule \vspace{0.5cm}

        \begin{center}
\textcolor{red}{\emph\Large Correcting version}\end{center}
\begin{questions}
\begin{multicols*}{2}
\question[20] A figura abaixo mostra a trajetória de uma partícula eletricamente carregada. $\\vec{{v}}$ representa a velocidade atravessando um campo magnético $\\vec{{B}}$. Determine a sua trajetória devido a ação da força magnética atuando sobre ela.
        
        \begin{center}
            \begin{minipage}[c]{0.5\linewidth}
                \begin{tikzpicture}[scale=0.5,transform shape, font=\Large]

                    \tkzInit[xmin=-3,xmax=3,ymin=-3,ymax=3]
                %	\tkzGrid[color=gray!20]
                    \tkzClip[space=1.0]

                    \tkzDefPoints{0/0/O,4/0/P}

                    \foreach \x in {-2.5,-1.5,...,2.5}{
                        \foreach \y in {-2.5,-1.5,...,2.5}{
                        \tkzDefPoint(\x,\y){B}
                        \tkzText(B){x}
                }
                }

                \draw[->, line width=1pt, color=red] (0,0) --++ (0,1.5) node [above] {$\vec{v}$};

                    \node[circle, radius=0.25, ball color=gray!50] (n1) at (0,0) {+};

                    \tkzText[above right=0.25cm](B){$\vec{B}$}

                \end{tikzpicture}
            \end{minipage}
        \end{center}

        

\begin{oneparchoices}
\choice 0.0\choice 10.0\choice 20\choice 0.0\choice 0.0\end{oneparchoices}
\question[20] Uma corrente elétrica de    6.33 A percorre um fio de cobre. Sabendo-se que a carga de um elétron é igual a $1,6\times 10^{-19}\;C$, qual é o número de elétrons que atravessa, por minuto, a seção reta desse fio?

\begin{oneparchoices}
\choice 0.0\choice 0.0\choice 0.0\choice 0.0\choice 10.0\choice 15.0\choice 10.0\choice 0.0\choice 0.0\choice 20\end{oneparchoices}
\question[20] Considere a figura abaixo onde as linhas trajeçadas representam superfícies equipotenciais Se colocarmos um elétron próximo a carga Q, quais trechos possíveis o elétron poderá se deslocar?
        
        \begin{center}
            \begin{minipage}[c]{0.5\linewidth}
                \begin{tikzpicture}[scale=0.5,transform shape, font=\Large]

                \tkzInit[xmin=-4,xmax=4,ymin=-4,ymax=4]
                \tkzClip[space=0.5]

                \tkzDefPoints{0/0/O,4/0/P}

                \foreach \x in {0.5,1.25,2.25,3,4}{
                    \tkzDrawCircle[R,dashed,color=gray!50](O,\x)
                }

                \foreach \y in {0,1,...,11}{
                    \tkzDefPointsBy[rotation= center O angle 30*\y](O,P){P1,P2}
                \draw[->, line width=1.0pt] (O) -- (P2);}

                \tkzDefPoints{3/0/a,4/0/b,0/4/c,0/3/d}

                \tkzDrawPoints[color=red,fill=red,size=0.3cm](a,b,c,d)

                \tkzDrawPoints(O)
                \tkzLabelPoints[above right,font=\Large](a,b,c,d)

                \node[circle, radius=0.25, ball color=gray!50] (n1) at (0,0) {Q};

                \end{tikzpicture}
            \end{minipage}
        \end{center}
        
        

\begin{oneparchoices}
\choice 0.0\choice 0.0\choice 20\choice 0.0\choice 0.0\end{oneparchoices}
\question[20] Uma diferença de potencial de 120 V é aplicada a uma bomba d’água. Sabe-se que em funcionamento, o motor da bomba é percorrido por uma corrente de    3.65 A. Qual é a potência desenvolvida nesse motor?

\begin{oneparchoices}
\choice 0.0\choice 0.0\choice 0.0\choice 0.0\choice 0.0\choice 20\choice 0.0\choice 0.0\choice 0.0\choice 0.0\end{oneparchoices}
\question[20] Uma partícula de carga 7.06e-06 C é lançada em um campo magnético uniforme de    0.48 T , com uma velocidade de 175.55 m/s. Calcule o valor da força magnética atuando na carga se o ângulo entre a velocidade e o campo magnético for   13.21 graus.

\begin{oneparchoices}
\choice 0.0\choice 0.0\choice 5.0\choice 0.0\choice 0.0\choice 0.0\choice 20\choice 5.0\choice 0.0\choice 15.0\end{oneparchoices}
\end{multicols*}
\end{questions}
\newpage
        \begin{minipage}[b]{0.75\linewidth}
            \begin{flushleft}
                {\bf \large Prova bimestral}
            \end{flushleft}
            \begin{flushleft}
                {\bf \large LQ2N (2B), 31 de outubro de 2022}
            \end{flushleft}
        \end{minipage}
        \begin{minipage}[b]{0.20\linewidth}
            \begin{flushright}
                {\bf \large Código: 3}
            \end{flushright}
        \end{minipage}
        \vspace{0.5cm} \hrule \vspace{0.5cm}
        \begin{minipage}{0.75\linewidth}
            Aluno:
        \end{minipage}
        \vspace{0.5cm} \hrule \vspace{0.5cm}

        \begin{center}
\textcolor{red}{\emph\Large Correcting version}\end{center}
\begin{questions}
\begin{multicols*}{2}
\question[20] A figura abaixo mostra a trajetória de uma partícula eletricamente carregada. $\\vec{{v}}$ representa a velocidade atravessando um campo magnético $\\vec{{B}}$. Determine a sua trajetória devido a ação da força magnética atuando sobre ela.
        
        \begin{center}
            \begin{minipage}[c]{0.5\linewidth}
                \begin{tikzpicture}[scale=0.5,transform shape, font=\Large]

                    \tkzInit[xmin=-3,xmax=3,ymin=-3,ymax=3]
                %	\tkzGrid[color=gray!20]
                    \tkzClip[space=1.0]

                    \tkzDefPoints{0/0/O,4/0/P}

                    \foreach \x in {-2.5,-1.5,...,2.5}{
                        \foreach \y in {-2.5,-1.5,...,2.5}{
                        \tkzDefPoint(\x,\y){B}
                        \tkzText(B){x}
                }
                }

                \draw[->, line width=1pt, color=red] (0,0) --++ (0,1.5) node [above] {$\vec{v}$};

                    \node[circle, radius=0.25, ball color=gray!50] (n1) at (0,0) {+};

                    \tkzText[above right=0.25cm](B){$\vec{B}$}

                \end{tikzpicture}
            \end{minipage}
        \end{center}

        

\begin{oneparchoices}
\choice 0.0\choice 10.0\choice 0.0\choice 0.0\choice 20\end{oneparchoices}
\question[20] Uma corrente elétrica de    7.47 A percorre um fio de cobre. Sabendo-se que a carga de um elétron é igual a $1,6\times 10^{-19}\;C$, qual é o número de elétrons que atravessa, por minuto, a seção reta desse fio?

\begin{oneparchoices}
\choice 10.0\choice 15.0\choice 0.0\choice 20\choice 10.0\choice 0.0\choice 0.0\choice 0.0\choice 0.0\choice 0.0\end{oneparchoices}
\question[20] Considere a figura abaixo onde as linhas trajeçadas representam superfícies equipotenciais Se colocarmos um elétron próximo a carga Q, quais trechos possíveis o elétron poderá se deslocar?
        
        \begin{center}
            \begin{minipage}[c]{0.5\linewidth}
                \begin{tikzpicture}[scale=0.5,transform shape, font=\Large]

                \tkzInit[xmin=-4,xmax=4,ymin=-4,ymax=4]
                \tkzClip[space=0.5]

                \tkzDefPoints{0/0/O,4/0/P}

                \foreach \x in {0.5,1.25,2.25,3,4}{
                    \tkzDrawCircle[R,dashed,color=gray!50](O,\x)
                }

                \foreach \y in {0,1,...,11}{
                    \tkzDefPointsBy[rotation= center O angle 30*\y](O,P){P1,P2}
                \draw[->, line width=1.0pt] (O) -- (P2);}

                \tkzDefPoints{3/0/a,4/0/b,0/4/c,0/3/d}

                \tkzDrawPoints[color=red,fill=red,size=0.3cm](a,b,c,d)

                \tkzDrawPoints(O)
                \tkzLabelPoints[above right,font=\Large](a,b,c,d)

                \node[circle, radius=0.25, ball color=gray!50] (n1) at (0,0) {Q};

                \end{tikzpicture}
            \end{minipage}
        \end{center}
        
        

\begin{oneparchoices}
\choice 0.0\choice 0.0\choice 0.0\choice 0.0\choice 20\end{oneparchoices}
\question[20] Uma diferença de potencial de 120 V é aplicada a uma bomba d’água. Sabe-se que em funcionamento, o motor da bomba é percorrido por uma corrente de    4.60 A. Qual é a potência desenvolvida nesse motor?

\begin{oneparchoices}
\choice 0.0\choice 20\choice 0.0\choice 0.0\choice 0.0\choice 0.0\choice 0.0\choice 0.0\choice 0.0\choice 0.0\end{oneparchoices}
\question[20] Uma partícula de carga 6.66e-06 C é lançada em um campo magnético uniforme de    0.33 T , com uma velocidade de 528.77 m/s. Calcule o valor da força magnética atuando na carga se o ângulo entre a velocidade e o campo magnético for   83.36 graus.

\begin{oneparchoices}
\choice 20\choice 0.0\choice 0.0\choice 0.0\choice 0.0\choice 5.0\choice 0.0\choice 15.0\choice 0.0\choice 5.0\end{oneparchoices}
\end{multicols*}
\end{questions}
\newpage
        \begin{minipage}[b]{0.75\linewidth}
            \begin{flushleft}
                {\bf \large Prova bimestral}
            \end{flushleft}
            \begin{flushleft}
                {\bf \large LQ2N (2B), 31 de outubro de 2022}
            \end{flushleft}
        \end{minipage}
        \begin{minipage}[b]{0.20\linewidth}
            \begin{flushright}
                {\bf \large Código: 4}
            \end{flushright}
        \end{minipage}
        \vspace{0.5cm} \hrule \vspace{0.5cm}
        \begin{minipage}{0.75\linewidth}
            Aluno:
        \end{minipage}
        \vspace{0.5cm} \hrule \vspace{0.5cm}

        \begin{center}
\textcolor{red}{\emph\Large Correcting version}\end{center}
\begin{questions}
\begin{multicols*}{2}
\question[20] A figura abaixo mostra a trajetória de uma partícula eletricamente carregada. $\\vec{{v}}$ representa a velocidade atravessando um campo magnético $\\vec{{B}}$. Determine a sua trajetória devido a ação da força magnética atuando sobre ela.
        
        \begin{center}
            \begin{minipage}[c]{0.5\linewidth}
                \begin{tikzpicture}[scale=0.5,transform shape, font=\Large]

                    \tkzInit[xmin=-3,xmax=3,ymin=-3,ymax=3]
                %	\tkzGrid[color=gray!20]
                    \tkzClip[space=1.0]

                    \tkzDefPoints{0/0/O,4/0/P}

                    \foreach \x in {-2.5,-1.5,...,2.5}{
                        \foreach \y in {-2.5,-1.5,...,2.5}{
                        \tkzDefPoint(\x,\y){B}
                        \tkzText(B){x}
                }
                }

                \draw[->, line width=1pt, color=red] (0,0) --++ (0,1.5) node [above] {$\vec{v}$};

                    \node[circle, radius=0.25, ball color=gray!50] (n1) at (0,0) {+};

                    \tkzText[above right=0.25cm](B){$\vec{B}$}

                \end{tikzpicture}
            \end{minipage}
        \end{center}

        

\begin{oneparchoices}
\choice 0.0\choice 20\choice 10.0\choice 0.0\choice 0.0\end{oneparchoices}
\question[20] Uma corrente elétrica de    3.90 A percorre um fio de cobre. Sabendo-se que a carga de um elétron é igual a $1,6\times 10^{-19}\;C$, qual é o número de elétrons que atravessa, por minuto, a seção reta desse fio?

\begin{oneparchoices}
\choice 15.0\choice 0.0\choice 0.0\choice 0.0\choice 0.0\choice 20\choice 0.0\choice 0.0\choice 10.0\choice 10.0\end{oneparchoices}
\question[20] Considere a figura abaixo onde as linhas trajeçadas representam superfícies equipotenciais Se colocarmos um elétron próximo a carga Q, quais trechos possíveis o elétron poderá se deslocar?
        
        \begin{center}
            \begin{minipage}[c]{0.5\linewidth}
                \begin{tikzpicture}[scale=0.5,transform shape, font=\Large]

                \tkzInit[xmin=-4,xmax=4,ymin=-4,ymax=4]
                \tkzClip[space=0.5]

                \tkzDefPoints{0/0/O,4/0/P}

                \foreach \x in {0.5,1.25,2.25,3,4}{
                    \tkzDrawCircle[R,dashed,color=gray!50](O,\x)
                }

                \foreach \y in {0,1,...,11}{
                    \tkzDefPointsBy[rotation= center O angle 30*\y](O,P){P1,P2}
                \draw[->, line width=1.0pt] (O) -- (P2);}

                \tkzDefPoints{3/0/a,4/0/b,0/4/c,0/3/d}

                \tkzDrawPoints[color=red,fill=red,size=0.3cm](a,b,c,d)

                \tkzDrawPoints(O)
                \tkzLabelPoints[above right,font=\Large](a,b,c,d)

                \node[circle, radius=0.25, ball color=gray!50] (n1) at (0,0) {Q};

                \end{tikzpicture}
            \end{minipage}
        \end{center}
        
        

\begin{oneparchoices}
\choice 0.0\choice 0.0\choice 0.0\choice 20\choice 0.0\end{oneparchoices}
\question[20] Uma diferença de potencial de 120 V é aplicada a uma bomba d’água. Sabe-se que em funcionamento, o motor da bomba é percorrido por uma corrente de    4.45 A. Qual é a potência desenvolvida nesse motor?

\begin{oneparchoices}
\choice 0.0\choice 0.0\choice 20\choice 0.0\choice 0.0\choice 0.0\choice 0.0\choice 0.0\choice 0.0\choice 0.0\end{oneparchoices}
\question[20] Uma partícula de carga 7.34e-06 C é lançada em um campo magnético uniforme de    0.46 T , com uma velocidade de 296.81 m/s. Calcule o valor da força magnética atuando na carga se o ângulo entre a velocidade e o campo magnético for   69.33 graus.

\begin{oneparchoices}
\choice 5.0\choice 0.0\choice 20\choice 5.0\choice 0.0\choice 15.0\choice 0.0\choice 0.0\choice 0.0\choice 0.0\end{oneparchoices}
\end{multicols*}
\end{questions}
\newpage
        \begin{minipage}[b]{0.75\linewidth}
            \begin{flushleft}
                {\bf \large Prova bimestral}
            \end{flushleft}
            \begin{flushleft}
                {\bf \large LQ2N (2B), 31 de outubro de 2022}
            \end{flushleft}
        \end{minipage}
        \begin{minipage}[b]{0.20\linewidth}
            \begin{flushright}
                {\bf \large Código: 5}
            \end{flushright}
        \end{minipage}
        \vspace{0.5cm} \hrule \vspace{0.5cm}
        \begin{minipage}{0.75\linewidth}
            Aluno:
        \end{minipage}
        \vspace{0.5cm} \hrule \vspace{0.5cm}

        \begin{center}
\textcolor{red}{\emph\Large Correcting version}\end{center}
\begin{questions}
\begin{multicols*}{2}
\question[20] A figura abaixo mostra a trajetória de uma partícula eletricamente carregada. $\\vec{{v}}$ representa a velocidade atravessando um campo magnético $\\vec{{B}}$. Determine a sua trajetória devido a ação da força magnética atuando sobre ela.
        
        \begin{center}
            \begin{minipage}[c]{0.5\linewidth}
                \begin{tikzpicture}[scale=0.5,transform shape, font=\Large]

                    \tkzInit[xmin=-3,xmax=3,ymin=-3,ymax=3]
                %	\tkzGrid[color=gray!20]
                    \tkzClip[space=1.0]

                    \tkzDefPoints{0/0/O,4/0/P}

                    \foreach \x in {-2.5,-1.5,...,2.5}{
                        \foreach \y in {-2.5,-1.5,...,2.5}{
                        \tkzDefPoint(\x,\y){B}
                        \tkzText(B){x}
                }
                }

                \draw[->, line width=1pt, color=red] (0,0) --++ (0,1.5) node [above] {$\vec{v}$};

                    \node[circle, radius=0.25, ball color=gray!50] (n1) at (0,0) {+};

                    \tkzText[above right=0.25cm](B){$\vec{B}$}

                \end{tikzpicture}
            \end{minipage}
        \end{center}

        

\begin{oneparchoices}
\choice 20\choice 10.0\choice 0.0\choice 0.0\choice 0.0\end{oneparchoices}
\question[20] Uma corrente elétrica de    2.17 A percorre um fio de cobre. Sabendo-se que a carga de um elétron é igual a $1,6\times 10^{-19}\;C$, qual é o número de elétrons que atravessa, por minuto, a seção reta desse fio?

\begin{oneparchoices}
\choice 10.0\choice 0.0\choice 20\choice 0.0\choice 0.0\choice 0.0\choice 10.0\choice 15.0\choice 0.0\choice 0.0\end{oneparchoices}
\question[20] Considere a figura abaixo onde as linhas trajeçadas representam superfícies equipotenciais Se colocarmos um elétron próximo a carga Q, quais trechos possíveis o elétron poderá se deslocar?
        
        \begin{center}
            \begin{minipage}[c]{0.5\linewidth}
                \begin{tikzpicture}[scale=0.5,transform shape, font=\Large]

                \tkzInit[xmin=-4,xmax=4,ymin=-4,ymax=4]
                \tkzClip[space=0.5]

                \tkzDefPoints{0/0/O,4/0/P}

                \foreach \x in {0.5,1.25,2.25,3,4}{
                    \tkzDrawCircle[R,dashed,color=gray!50](O,\x)
                }

                \foreach \y in {0,1,...,11}{
                    \tkzDefPointsBy[rotation= center O angle 30*\y](O,P){P1,P2}
                \draw[->, line width=1.0pt] (O) -- (P2);}

                \tkzDefPoints{3/0/a,4/0/b,0/4/c,0/3/d}

                \tkzDrawPoints[color=red,fill=red,size=0.3cm](a,b,c,d)

                \tkzDrawPoints(O)
                \tkzLabelPoints[above right,font=\Large](a,b,c,d)

                \node[circle, radius=0.25, ball color=gray!50] (n1) at (0,0) {Q};

                \end{tikzpicture}
            \end{minipage}
        \end{center}
        
        

\begin{oneparchoices}
\choice 20\choice 0.0\choice 0.0\choice 0.0\choice 0.0\end{oneparchoices}
\question[20] Uma diferença de potencial de 120 V é aplicada a uma bomba d’água. Sabe-se que em funcionamento, o motor da bomba é percorrido por uma corrente de    3.08 A. Qual é a potência desenvolvida nesse motor?

\begin{oneparchoices}
\choice 0.0\choice 0.0\choice 0.0\choice 0.0\choice 0.0\choice 20\choice 0.0\choice 0.0\choice 0.0\choice 0.0\end{oneparchoices}
\question[20] Uma partícula de carga 5.19e-06 C é lançada em um campo magnético uniforme de    0.90 T , com uma velocidade de 456.72 m/s. Calcule o valor da força magnética atuando na carga se o ângulo entre a velocidade e o campo magnético for   67.25 graus.

\begin{oneparchoices}
\choice 0.0\choice 0.0\choice 0.0\choice 15.0\choice 0.0\choice 0.0\choice 20\choice 0.0\choice 5.0\choice 5.0\end{oneparchoices}
\end{multicols*}
\end{questions}
\newpage
        \begin{minipage}[b]{0.75\linewidth}
            \begin{flushleft}
                {\bf \large Prova bimestral}
            \end{flushleft}
            \begin{flushleft}
                {\bf \large LQ2N (2B), 31 de outubro de 2022}
            \end{flushleft}
        \end{minipage}
        \begin{minipage}[b]{0.20\linewidth}
            \begin{flushright}
                {\bf \large Código: 6}
            \end{flushright}
        \end{minipage}
        \vspace{0.5cm} \hrule \vspace{0.5cm}
        \begin{minipage}{0.75\linewidth}
            Aluno:
        \end{minipage}
        \vspace{0.5cm} \hrule \vspace{0.5cm}

        \begin{center}
\textcolor{red}{\emph\Large Correcting version}\end{center}
\begin{questions}
\begin{multicols*}{2}
\question[20] A figura abaixo mostra a trajetória de uma partícula eletricamente carregada. $\\vec{{v}}$ representa a velocidade atravessando um campo magnético $\\vec{{B}}$. Determine a sua trajetória devido a ação da força magnética atuando sobre ela.
        
        \begin{center}
            \begin{minipage}[c]{0.5\linewidth}
                \begin{tikzpicture}[scale=0.5,transform shape, font=\Large]

                    \tkzInit[xmin=-3,xmax=3,ymin=-3,ymax=3]
                %	\tkzGrid[color=gray!20]
                    \tkzClip[space=1.0]

                    \tkzDefPoints{0/0/O,4/0/P}

                    \foreach \x in {-2.5,-1.5,...,2.5}{
                        \foreach \y in {-2.5,-1.5,...,2.5}{
                        \tkzDefPoint(\x,\y){B}
                        \tkzText(B){x}
                }
                }

                \draw[->, line width=1pt, color=red] (0,0) --++ (0,1.5) node [above] {$\vec{v}$};

                    \node[circle, radius=0.25, ball color=gray!50] (n1) at (0,0) {+};

                    \tkzText[above right=0.25cm](B){$\vec{B}$}

                \end{tikzpicture}
            \end{minipage}
        \end{center}

        

\begin{oneparchoices}
\choice 0.0\choice 0.0\choice 10.0\choice 0.0\choice 20\end{oneparchoices}
\question[20] Uma corrente elétrica de    3.16 A percorre um fio de cobre. Sabendo-se que a carga de um elétron é igual a $1,6\times 10^{-19}\;C$, qual é o número de elétrons que atravessa, por minuto, a seção reta desse fio?

\begin{oneparchoices}
\choice 0.0\choice 0.0\choice 20\choice 0.0\choice 10.0\choice 0.0\choice 10.0\choice 15.0\choice 0.0\choice 0.0\end{oneparchoices}
\question[20] Considere a figura abaixo onde as linhas trajeçadas representam superfícies equipotenciais Se colocarmos um elétron próximo a carga Q, quais trechos possíveis o elétron poderá se deslocar?
        
        \begin{center}
            \begin{minipage}[c]{0.5\linewidth}
                \begin{tikzpicture}[scale=0.5,transform shape, font=\Large]

                \tkzInit[xmin=-4,xmax=4,ymin=-4,ymax=4]
                \tkzClip[space=0.5]

                \tkzDefPoints{0/0/O,4/0/P}

                \foreach \x in {0.5,1.25,2.25,3,4}{
                    \tkzDrawCircle[R,dashed,color=gray!50](O,\x)
                }

                \foreach \y in {0,1,...,11}{
                    \tkzDefPointsBy[rotation= center O angle 30*\y](O,P){P1,P2}
                \draw[->, line width=1.0pt] (O) -- (P2);}

                \tkzDefPoints{3/0/a,4/0/b,0/4/c,0/3/d}

                \tkzDrawPoints[color=red,fill=red,size=0.3cm](a,b,c,d)

                \tkzDrawPoints(O)
                \tkzLabelPoints[above right,font=\Large](a,b,c,d)

                \node[circle, radius=0.25, ball color=gray!50] (n1) at (0,0) {Q};

                \end{tikzpicture}
            \end{minipage}
        \end{center}
        
        

\begin{oneparchoices}
\choice 0.0\choice 0.0\choice 20\choice 0.0\choice 0.0\end{oneparchoices}
\question[20] Uma diferença de potencial de 120 V é aplicada a uma bomba d’água. Sabe-se que em funcionamento, o motor da bomba é percorrido por uma corrente de    3.16 A. Qual é a potência desenvolvida nesse motor?

\begin{oneparchoices}
\choice 20\choice 0.0\choice 0.0\choice 0.0\choice 0.0\choice 0.0\choice 0.0\choice 0.0\choice 0.0\choice 0.0\end{oneparchoices}
\question[20] Uma partícula de carga 2.59e-06 C é lançada em um campo magnético uniforme de    0.87 T , com uma velocidade de 655.21 m/s. Calcule o valor da força magnética atuando na carga se o ângulo entre a velocidade e o campo magnético for   57.59 graus.

\begin{oneparchoices}
\choice 5.0\choice 0.0\choice 15.0\choice 0.0\choice 20\choice 0.0\choice 5.0\choice 0.0\choice 0.0\choice 0.0\end{oneparchoices}
\end{multicols*}
\end{questions}
\newpage
        \begin{minipage}[b]{0.75\linewidth}
            \begin{flushleft}
                {\bf \large Prova bimestral}
            \end{flushleft}
            \begin{flushleft}
                {\bf \large LQ2N (2B), 31 de outubro de 2022}
            \end{flushleft}
        \end{minipage}
        \begin{minipage}[b]{0.20\linewidth}
            \begin{flushright}
                {\bf \large Código: 7}
            \end{flushright}
        \end{minipage}
        \vspace{0.5cm} \hrule \vspace{0.5cm}
        \begin{minipage}{0.75\linewidth}
            Aluno:
        \end{minipage}
        \vspace{0.5cm} \hrule \vspace{0.5cm}

        \begin{center}
\textcolor{red}{\emph\Large Correcting version}\end{center}
\begin{questions}
\begin{multicols*}{2}
\question[20] A figura abaixo mostra a trajetória de uma partícula eletricamente carregada. $\\vec{{v}}$ representa a velocidade atravessando um campo magnético $\\vec{{B}}$. Determine a sua trajetória devido a ação da força magnética atuando sobre ela.
        
        \begin{center}
            \begin{minipage}[c]{0.5\linewidth}
                \begin{tikzpicture}[scale=0.5,transform shape, font=\Large]

                    \tkzInit[xmin=-3,xmax=3,ymin=-3,ymax=3]
                %	\tkzGrid[color=gray!20]
                    \tkzClip[space=1.0]

                    \tkzDefPoints{0/0/O,4/0/P}

                    \foreach \x in {-2.5,-1.5,...,2.5}{
                        \foreach \y in {-2.5,-1.5,...,2.5}{
                        \tkzDefPoint(\x,\y){B}
                        \tkzText(B){x}
                }
                }

                \draw[->, line width=1pt, color=red] (0,0) --++ (0,1.5) node [above] {$\vec{v}$};

                    \node[circle, radius=0.25, ball color=gray!50] (n1) at (0,0) {+};

                    \tkzText[above right=0.25cm](B){$\vec{B}$}

                \end{tikzpicture}
            \end{minipage}
        \end{center}

        

\begin{oneparchoices}
\choice 10.0\choice 20\choice 0.0\choice 0.0\choice 0.0\end{oneparchoices}
\question[20] Uma corrente elétrica de    9.29 A percorre um fio de cobre. Sabendo-se que a carga de um elétron é igual a $1,6\times 10^{-19}\;C$, qual é o número de elétrons que atravessa, por minuto, a seção reta desse fio?

\begin{oneparchoices}
\choice 0.0\choice 0.0\choice 0.0\choice 20\choice 10.0\choice 0.0\choice 10.0\choice 15.0\choice 0.0\choice 0.0\end{oneparchoices}
\question[20] Considere a figura abaixo onde as linhas trajeçadas representam superfícies equipotenciais Se colocarmos um elétron próximo a carga Q, quais trechos possíveis o elétron poderá se deslocar?
        
        \begin{center}
            \begin{minipage}[c]{0.5\linewidth}
                \begin{tikzpicture}[scale=0.5,transform shape, font=\Large]

                \tkzInit[xmin=-4,xmax=4,ymin=-4,ymax=4]
                \tkzClip[space=0.5]

                \tkzDefPoints{0/0/O,4/0/P}

                \foreach \x in {0.5,1.25,2.25,3,4}{
                    \tkzDrawCircle[R,dashed,color=gray!50](O,\x)
                }

                \foreach \y in {0,1,...,11}{
                    \tkzDefPointsBy[rotation= center O angle 30*\y](O,P){P1,P2}
                \draw[->, line width=1.0pt] (O) -- (P2);}

                \tkzDefPoints{3/0/a,4/0/b,0/4/c,0/3/d}

                \tkzDrawPoints[color=red,fill=red,size=0.3cm](a,b,c,d)

                \tkzDrawPoints(O)
                \tkzLabelPoints[above right,font=\Large](a,b,c,d)

                \node[circle, radius=0.25, ball color=gray!50] (n1) at (0,0) {Q};

                \end{tikzpicture}
            \end{minipage}
        \end{center}
        
        

\begin{oneparchoices}
\choice 0.0\choice 20\choice 0.0\choice 0.0\choice 0.0\end{oneparchoices}
\question[20] Uma diferença de potencial de 120 V é aplicada a uma bomba d’água. Sabe-se que em funcionamento, o motor da bomba é percorrido por uma corrente de    3.68 A. Qual é a potência desenvolvida nesse motor?

\begin{oneparchoices}
\choice 0.0\choice 0.0\choice 0.0\choice 0.0\choice 0.0\choice 20\choice 0.0\choice 0.0\choice 0.0\choice 0.0\end{oneparchoices}
\question[20] Uma partícula de carga 7.60e-06 C é lançada em um campo magnético uniforme de    0.33 T , com uma velocidade de 461.23 m/s. Calcule o valor da força magnética atuando na carga se o ângulo entre a velocidade e o campo magnético for   13.40 graus.

\begin{oneparchoices}
\choice 5.0\choice 0.0\choice 20\choice 0.0\choice 15.0\choice 5.0\choice 0.0\choice 0.0\choice 0.0\choice 0.0\end{oneparchoices}
\end{multicols*}
\end{questions}
\newpage
        \begin{minipage}[b]{0.75\linewidth}
            \begin{flushleft}
                {\bf \large Prova bimestral}
            \end{flushleft}
            \begin{flushleft}
                {\bf \large LQ2N (2B), 31 de outubro de 2022}
            \end{flushleft}
        \end{minipage}
        \begin{minipage}[b]{0.20\linewidth}
            \begin{flushright}
                {\bf \large Código: 8}
            \end{flushright}
        \end{minipage}
        \vspace{0.5cm} \hrule \vspace{0.5cm}
        \begin{minipage}{0.75\linewidth}
            Aluno:
        \end{minipage}
        \vspace{0.5cm} \hrule \vspace{0.5cm}

        \begin{center}
\textcolor{red}{\emph\Large Correcting version}\end{center}
\begin{questions}
\begin{multicols*}{2}
\question[20] A figura abaixo mostra a trajetória de uma partícula eletricamente carregada. $\\vec{{v}}$ representa a velocidade atravessando um campo magnético $\\vec{{B}}$. Determine a sua trajetória devido a ação da força magnética atuando sobre ela.
        
        \begin{center}
            \begin{minipage}[c]{0.5\linewidth}
                \begin{tikzpicture}[scale=0.5,transform shape, font=\Large]

                    \tkzInit[xmin=-3,xmax=3,ymin=-3,ymax=3]
                %	\tkzGrid[color=gray!20]
                    \tkzClip[space=1.0]

                    \tkzDefPoints{0/0/O,4/0/P}

                    \foreach \x in {-2.5,-1.5,...,2.5}{
                        \foreach \y in {-2.5,-1.5,...,2.5}{
                        \tkzDefPoint(\x,\y){B}
                        \tkzText(B){x}
                }
                }

                \draw[->, line width=1pt, color=red] (0,0) --++ (0,1.5) node [above] {$\vec{v}$};

                    \node[circle, radius=0.25, ball color=gray!50] (n1) at (0,0) {+};

                    \tkzText[above right=0.25cm](B){$\vec{B}$}

                \end{tikzpicture}
            \end{minipage}
        \end{center}

        

\begin{oneparchoices}
\choice 0.0\choice 0.0\choice 10.0\choice 20\choice 0.0\end{oneparchoices}
\question[20] Uma corrente elétrica de    3.84 A percorre um fio de cobre. Sabendo-se que a carga de um elétron é igual a $1,6\times 10^{-19}\;C$, qual é o número de elétrons que atravessa, por minuto, a seção reta desse fio?

\begin{oneparchoices}
\choice 20\choice 15.0\choice 10.0\choice 0.0\choice 0.0\choice 0.0\choice 0.0\choice 0.0\choice 0.0\choice 10.0\end{oneparchoices}
\question[20] Considere a figura abaixo onde as linhas trajeçadas representam superfícies equipotenciais Se colocarmos um elétron próximo a carga Q, quais trechos possíveis o elétron poderá se deslocar?
        
        \begin{center}
            \begin{minipage}[c]{0.5\linewidth}
                \begin{tikzpicture}[scale=0.5,transform shape, font=\Large]

                \tkzInit[xmin=-4,xmax=4,ymin=-4,ymax=4]
                \tkzClip[space=0.5]

                \tkzDefPoints{0/0/O,4/0/P}

                \foreach \x in {0.5,1.25,2.25,3,4}{
                    \tkzDrawCircle[R,dashed,color=gray!50](O,\x)
                }

                \foreach \y in {0,1,...,11}{
                    \tkzDefPointsBy[rotation= center O angle 30*\y](O,P){P1,P2}
                \draw[->, line width=1.0pt] (O) -- (P2);}

                \tkzDefPoints{3/0/a,4/0/b,0/4/c,0/3/d}

                \tkzDrawPoints[color=red,fill=red,size=0.3cm](a,b,c,d)

                \tkzDrawPoints(O)
                \tkzLabelPoints[above right,font=\Large](a,b,c,d)

                \node[circle, radius=0.25, ball color=gray!50] (n1) at (0,0) {Q};

                \end{tikzpicture}
            \end{minipage}
        \end{center}
        
        

\begin{oneparchoices}
\choice 20\choice 0.0\choice 0.0\choice 0.0\choice 0.0\end{oneparchoices}
\question[20] Uma diferença de potencial de 120 V é aplicada a uma bomba d’água. Sabe-se que em funcionamento, o motor da bomba é percorrido por uma corrente de    3.38 A. Qual é a potência desenvolvida nesse motor?

\begin{oneparchoices}
\choice 0.0\choice 20\choice 0.0\choice 0.0\choice 0.0\choice 0.0\choice 0.0\choice 0.0\choice 0.0\choice 0.0\end{oneparchoices}
\question[20] Uma partícula de carga 3.96e-06 C é lançada em um campo magnético uniforme de    0.89 T , com uma velocidade de 982.56 m/s. Calcule o valor da força magnética atuando na carga se o ângulo entre a velocidade e o campo magnético for   23.50 graus.

\begin{oneparchoices}
\choice 20\choice 0.0\choice 5.0\choice 0.0\choice 0.0\choice 0.0\choice 0.0\choice 5.0\choice 0.0\choice 15.0\end{oneparchoices}
\end{multicols*}
\end{questions}
\newpage
        \begin{minipage}[b]{0.75\linewidth}
            \begin{flushleft}
                {\bf \large Prova bimestral}
            \end{flushleft}
            \begin{flushleft}
                {\bf \large LQ2N (2B), 31 de outubro de 2022}
            \end{flushleft}
        \end{minipage}
        \begin{minipage}[b]{0.20\linewidth}
            \begin{flushright}
                {\bf \large Código: 9}
            \end{flushright}
        \end{minipage}
        \vspace{0.5cm} \hrule \vspace{0.5cm}
        \begin{minipage}{0.75\linewidth}
            Aluno:
        \end{minipage}
        \vspace{0.5cm} \hrule \vspace{0.5cm}

        \begin{center}
\textcolor{red}{\emph\Large Correcting version}\end{center}
\begin{questions}
\begin{multicols*}{2}
\question[20] A figura abaixo mostra a trajetória de uma partícula eletricamente carregada. $\\vec{{v}}$ representa a velocidade atravessando um campo magnético $\\vec{{B}}$. Determine a sua trajetória devido a ação da força magnética atuando sobre ela.
        
        \begin{center}
            \begin{minipage}[c]{0.5\linewidth}
                \begin{tikzpicture}[scale=0.5,transform shape, font=\Large]

                    \tkzInit[xmin=-3,xmax=3,ymin=-3,ymax=3]
                %	\tkzGrid[color=gray!20]
                    \tkzClip[space=1.0]

                    \tkzDefPoints{0/0/O,4/0/P}

                    \foreach \x in {-2.5,-1.5,...,2.5}{
                        \foreach \y in {-2.5,-1.5,...,2.5}{
                        \tkzDefPoint(\x,\y){B}
                        \tkzText(B){x}
                }
                }

                \draw[->, line width=1pt, color=red] (0,0) --++ (0,1.5) node [above] {$\vec{v}$};

                    \node[circle, radius=0.25, ball color=gray!50] (n1) at (0,0) {+};

                    \tkzText[above right=0.25cm](B){$\vec{B}$}

                \end{tikzpicture}
            \end{minipage}
        \end{center}

        

\begin{oneparchoices}
\choice 0.0\choice 0.0\choice 0.0\choice 20\choice 10.0\end{oneparchoices}
\question[20] Uma corrente elétrica de    1.39 A percorre um fio de cobre. Sabendo-se que a carga de um elétron é igual a $1,6\times 10^{-19}\;C$, qual é o número de elétrons que atravessa, por minuto, a seção reta desse fio?

\begin{oneparchoices}
\choice 20\choice 0.0\choice 0.0\choice 10.0\choice 10.0\choice 0.0\choice 0.0\choice 15.0\choice 0.0\choice 0.0\end{oneparchoices}
\question[20] Considere a figura abaixo onde as linhas trajeçadas representam superfícies equipotenciais Se colocarmos um elétron próximo a carga Q, quais trechos possíveis o elétron poderá se deslocar?
        
        \begin{center}
            \begin{minipage}[c]{0.5\linewidth}
                \begin{tikzpicture}[scale=0.5,transform shape, font=\Large]

                \tkzInit[xmin=-4,xmax=4,ymin=-4,ymax=4]
                \tkzClip[space=0.5]

                \tkzDefPoints{0/0/O,4/0/P}

                \foreach \x in {0.5,1.25,2.25,3,4}{
                    \tkzDrawCircle[R,dashed,color=gray!50](O,\x)
                }

                \foreach \y in {0,1,...,11}{
                    \tkzDefPointsBy[rotation= center O angle 30*\y](O,P){P1,P2}
                \draw[->, line width=1.0pt] (O) -- (P2);}

                \tkzDefPoints{3/0/a,4/0/b,0/4/c,0/3/d}

                \tkzDrawPoints[color=red,fill=red,size=0.3cm](a,b,c,d)

                \tkzDrawPoints(O)
                \tkzLabelPoints[above right,font=\Large](a,b,c,d)

                \node[circle, radius=0.25, ball color=gray!50] (n1) at (0,0) {Q};

                \end{tikzpicture}
            \end{minipage}
        \end{center}
        
        

\begin{oneparchoices}
\choice 20\choice 0.0\choice 0.0\choice 0.0\choice 0.0\end{oneparchoices}
\question[20] Uma diferença de potencial de 120 V é aplicada a uma bomba d’água. Sabe-se que em funcionamento, o motor da bomba é percorrido por uma corrente de    4.76 A. Qual é a potência desenvolvida nesse motor?

\begin{oneparchoices}
\choice 0.0\choice 0.0\choice 0.0\choice 0.0\choice 0.0\choice 0.0\choice 20\choice 0.0\choice 0.0\choice 0.0\end{oneparchoices}
\question[20] Uma partícula de carga 6.52e-06 C é lançada em um campo magnético uniforme de    0.33 T , com uma velocidade de 184.73 m/s. Calcule o valor da força magnética atuando na carga se o ângulo entre a velocidade e o campo magnético for   11.15 graus.

\begin{oneparchoices}
\choice 0.0\choice 0.0\choice 0.0\choice 15.0\choice 5.0\choice 5.0\choice 0.0\choice 0.0\choice 0.0\choice 20\end{oneparchoices}
\end{multicols*}
\end{questions}
\newpage\end{document}